% -*-cap9.tex-*- Este fichero es parte de la plantilla LaTeX para la realización
% de Proyectos Final de Carrera, protejido bajo los términos de la licencia
% GFDL.  Para más información, la licencia completa viene incluida en el fichero
% fdl-1.3.tex

% Copyright (C) 2009 Pablo Recio Quijano

Para terminar vamos a desarrollar este capítulo dónde hablaremos de nuestras
expectativas futuras en torno a la posible extensión del sistema software creado
y las conclusiones que hemos sacado a través del trabajo que hemos realizado en
torno al mismo.

\section{Conclusiones Generales}
Como resultado del desarrollo del sistema \cursiva{Simulator Football Online}
hemos sido capaces de desarrollar una aplicación web multidioma y multijugador
que será capaz de poner en contacto a diferentes usuarios para simular la
gestión que se realizan en los clubes de fútbol desde el entrenamiento de los
jugadores que conforman al equipo hasta la estipulación del precio de las
entradas del siguiente partido.

El resultado que hemos obtenido al término del desarrollo del sistema software
es el de un producto web al que podrán acceder un número ilimitado de personas
dada la naturaleza de libre acceso con la que ha sido diseñado. Gracias a ello
se han colmado varias expectativas personales y profesionales.


\subsection{Primer simulador de la gestión de clubes de fútbol libre}
\cursiva{Simulator Football Online} es el primer simulador de la gestión de
clubes de fútbol realizado sobre \cursiva{Ruby On Rails} licenciado mediante la fórmula de software libre, en concreto,
liberado bajo licencia \cursiva{GPL}.

Esto supone varias ventajas que otros sistemas no aportan, sobre todo las
inherentes a su naturaleza libre, entre ellas destacamos \cite{fsf:what_is}:

\begin{itemize}
\item Libertad de usar el programa con cualquier propósito. Aunque distribuyamos
  esta primera versión de manera gratuita, no habría ningún problema en hacer
  una explotación comercial del sistema.
\item Libertad para estudiar cómo funciona el programa y modificarlo,
  adaptándolo a cualquier necesidad. Esto supondría que podrían crearse varios
  \cursiva{forks} para adaptarlo a otro tipo de competiciones deportivas.
\item La libertad para distribuir copias del programa, con lo que gracias a ello
  se puede llegar a un mayor número de usuarios.
\item La libertad para mejorar el programa y hacer públicas esas mejoras a los
  demás, de modo que toda la comunidad se beneficie de ello. Gracias a esto,
  posibles errores serían modificados con mayor celeridad.
\end{itemize}

El licenciar este programa con una licencia de software libre supone devolver a
la comunidad todo lo que ella me ha aportado a mi. Sólo hace falta visionar en
el capítulo \ref{cap:implementacion} el número de herramientas con licencia de
software libre para darse cuenta que de no disponer de la licencia de software
libre pertinente, este proyecto no podría haberse llevado a cabo.

\subsection{Aplicación de lo aprendido durante la etapa educativa}
Este proyecto no podría haber sido llevado a buen término de no haber cursado
numerosas asignaturas de la carrera universitaria cursada, a destacar:

\begin{description}
\item [Ingeniería del Software] Impartida por Doña Elena García Orta. En ella
  aprendimos todo lo necesario para poder hacer un análisis y diseño software
  que hiciera la etapa de codificación prácticamente un mero trámite.
\item [Programación en Internet] Impartida por Don Manuel Palomo Duarte y actual
  tutor de este proyecto de fin de carrera. Gracias a la asignatura se sentaron
  las bases de cómo funciona la tecnología web, sus protocolos. Sin ella hubiera
  sido mucho más complicado el acercamiento al desarrollo de un portal web
  multiusuario.
\item [Inteligencia Artificial] Impartida por Don Ignacio Pérez
  Blanquer. Gracias a ella supimos evaluar la forma más sencilla para realizar
  un algoritmo lo suficientemente rápido y realista para la consecución de una
  simulación más o menos real del desarrollo de un partido de fútbol.
\item [Comercio electrónico] Aún acudiendo a esta asignatura como oyente, al
  pertenecer a la titulación de Ingeniería Informática, esta asignatura fue
  impartida en su primer año por Don Juan Manuel Dodero Beardo y precisamente
  utilizando como \cursiva{framework} el que hemos utilizado en este proyecto,
  \cursiva{Ruby On Rails}.
\end{description}

De todas formas, aún siendo estas asignaturas las que más competencias
académicas tenían en relación con el proyecto, no hay que menospreciar la
formación y conocimientos aportadas por todas las que componen la carrera como
son, Programación Orientada a Objetos, Programación Funcional, Sistemas
Operativos, Álgebra, y tantas otras que nos han aportado no solo los
conocimientos inherentes al temario de las asignaturas, sino a cómo pensar,
aprender y ser autosuficientes.

\subsection{Capacidad de iniciativa}
Este proyecto fue una propuesta propia. Dado que pensaba que tenía que orientar
el proyecto de fin de carrera a todo lo aprendido en la misma, y teniendo en
cuenta que la carrera pertenece a al rama de gestión de la informática, quise
orientar el desarrollo del proyecto al desarrollo de un sistema de gestión.

Teniendo en cuenta que los sistemas de gestión actuales suelen ir dirigidos a
colmar las necesidades de un programa integral que abarque toda la planificación
de recursos empresariales (\cursiva{ERP}), quisimos distanciarnos un poco de ese
tipo de sistemas y crear algo que nos reconfortara cada vez que estuviéramos
desarrollándolo, pero siempre teniendo como objetivo poner en práctica los
conocimientos adquiridos en la carrera.

Por ello propusimos a nuestro tutor de proyecto el desarrollo de un videojuego
integrado como una aplicación web de gestión de clubes de fútbol.

\subsection{Desarrollo íntegro}
Es una gran satisfacción, tanto en el plano personal como en el profesional el
desarrollo íntegro de un sistema software desde sus etapa de conceptualización
hasta la de su publicación. Además, ha supuesto un gran reto el enfrentarnos a
una tecnología relativamente nueva que hace un uso intensivo de la programación
funcional debido a la característica del lenguaje del \cursiva{framework} que
hemos escogido para el desarrollo.

Dado que hemos desarrollado este proyecto de forma individual, ha sido necesario
aplicar conocimientos y desarrollar algunas competencias de las que adolecíamos,
a destacar:

\begin{description}
\item [Usabilidad y compatibilidad web] Ha sido necesario aprender todo lo
  necesario respecto a usabilidad web para realizar una página web lo
  suficientemente sencilla y atractiva para que fuera usada por cualquier tipo
  de persona.

  Debido a nuestras limitaciones como grafistas, tuvimos que buscar un sistema
  que nos proporcionara una forma sencilla de diseñar y realizar una página web
  accesible y que cumpliera con los suficientes estándares para que funcionara
  con un amplio rango de navegadores.

  Conseguimos solventar nuestras carencias como grafistas gracias al uso del
  \cursiva{framework CSS}, \cursiva{Twitter Bootstrap}.
\item [Planificación y Gestión de proyectos] Dado que optamos por ser nosotros
  mismos quiénes delimitáramos los límites y ámbito de aplicación del proyecto
  tuvimos que embarcarnos en la necesidad de ser unos buenos gestores y
  planificadores del tiempo, así como prever cuáles serían los problemas que
  podría acarrear el desarrollo e implementación de algunas características a
  simple vista sencillas.

  Esta fue una experiencia muy enriquecedora que nos reveló que no es nada
  sencillo el cumplir con el calendario proyectado para la implementación de una
  funcionalidad que hayamos decidido introducir si no se ha analizado de forma
  concienzuda las posibles contingencias que nos pueden surgir.
\item [Herramientas de desarrollo] Gracias a tener que hacer un desarrollo desde
  cero, tanto de la documentación como de la aplicación en sí, se nos hizo
  necesario investigar cuáles serían las herramientas que más se adecuaran a
  nuestro flujo de desarrollo. Gracias a ello descubrimos nuevas herramientas,
  como \cursiva{SublimeText2} y aprendimos a utilizar nuevos controladores de la
  versión como es \cursiva{GIT}, además de tener que acostumbrarnos a seguir un
  flujo de desarrollo estándar para comunicarnos con el cliente, que en este
  caso sería nuestro tutor de proyecto.
\item [Aprendizaje de frameworks] A lo largo del desarrollo tuvimos que afrontar
  la decisión de codificar la aplicación desde cero, o utilizar
  \cursiva{frameworks}. Dado que uno de los objetivos que nos marcamos fue el
  aprendizaje de nuevas tecnologías nos decantamos por la utilización intensiva
  de \cursiva{frameworks}, esto nos llevó a concluir que en un principio es un
  ligero escollo el desarrollo a través de un marco de trabajo ya
  preestablecido, ya que hay que hacer una inversión en forma de tiempo para
  conocer cómo funciona y se estructura, y adaptar de esta forma tus hábitos de
  programación a la estructura que nos proporciona el \cursiva{framework} escogido.
\end{description}

\section{Trabajos futuros}
Dado que ha sido necesario imponer un límite de funcionalidades que desarrollar
para cumplir con la fecha de entrega y presentación del proyecto dentro de unos
límites razonables que se corresponden con las horas de estudio que representan
los 6 créditos que conforman la matrícula del proyecto, muchas de las ideas que
tuvimos a la hora de hacer la propuesta de proyecto fueron descartadas.

Hay que tener en cuenta la dificultad que supone que un sólo desarrollador sea
el que lleve a cabo todas las etapas de desarrollo de una página web, hay que
tener en cuenta además la dificultad añadida que supone que no tengamos muchos
conocimientos en cuanto a diseño gráfico para hacer del producto algo mucho más
atractivo de cara al usuario.

Debido a ello, lamentablemente, tuvimos que dejar fuera del pliego de
especificación de requisitos algunas funcionalidades que creemos que harían del
sistema desarrollado un simulador mucho más completo y atractivo para el
usuario. Aún así, y debido a que vamos a liberar el sistema bajo una licencia de
software libre, vamos a dejar constancia de cuáles son las características que
creemos que pueden ser mejorables y cuáles funcionalidades no han sido siquiera
implementadas, para que alguien, si lo desea, pueda implementarlas.

\subsection{Sistemas de Inteligencia artificial}

Creemos que la inteligencia artificial del sistema podría ser mejorada en varias
secciones, a saber:

\begin{itemize}
\item Subsistema de simulación de sanciones.
\item Subsistema de simulación de lesiones.
\item Perfeccionamiento del algoritmo de simulación de partidos.
\item Subsistema de simulación de mercado de agentes libres.
\item Subsistema de simulación de vejez.
\item Subsistema de simulación de junta directiva.
\end{itemize}

\subsection{Personalización de perfil de usuario y equipo}
Pensamos que sería muy interesante poder dejar al usuario la posibilidad de
personalizar las características de su club de fútbol. Como pueden ser

\begin{itemize}
\item Posibilidad de cambiar el nombre del estadio de su equipo.
\item Posibilidad de establecer un escudo para su equipo.
\item Posibilidad de establecer un nombre para su equipo.
\end{itemize}

\subsection{Sistema de estadísticas}
Sería interesante el desarrollar una sección web que mostrara estadísticas y
comparativas entre usuarios, futbolistas y clubes para que los jugadores puedan
perfeccionar su forma de gestionar los clubes y así conseguir una experiencia de
usuario más plena.