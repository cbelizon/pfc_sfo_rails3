% -*-memoria.tex-*-
% Este fichero es parte de la plantilla LaTeX para
% la realización de Proyectos Final de Carrera, protejido
% bajo los términos de la licencia GFDL.
% Para más información, la licencia completa viene incluida en el
% fichero fdl-1.3.tex

% Copyright (C) 2009 Pablo Recio Quijano 

%-------------------------------------------------------
% ---- Plantilla para libros / memorias PFC -----

% Realizada por Pablo Recio Quijano y Noelia Sales Montes 
% Formato de portada y primera página tomado del PFC de
% Francisco Javier Vázquez Púa, en su proyecto 'libgann'
% -------------------------------------------------------

\documentclass[a4paper,11pt]{book}

\usepackage{./estilos/estiloBase} % Basicamente son todas las
                                  % librerias usadas. En caso de que
                                  % falten librerias se van añadiendo
                                  % al fichero.
\usepackage{./estilos/colores}  % Algunos colores ya generados, para
                                % los algunos estilos más avanzados.
\usepackage{./estilos/comandos} % Algunos comandos personalizados

\graphicspath{{./imagenes/}} % Indicamos la ruta donde se encuentran
                             % las imagenes, para ahorrarnos la ruta
                             % completa, y solo modificar aquí si en
                             % un momento dado lo movemos

\begin{document}

% Renombramos las figuras y las tablas
\renewcommand{\figurename}{Figura}
\renewcommand{\listfigurename}{Índice de figuras}
\renewcommand{\tablename}{Tabla}
\renewcommand{\listtablename}{Índice de tablas}

\pagestyle{empty}
\input{portada.tex}
\cleardoublepage

\input{primerahoja.tex}
\cleardoublepage
\pagestyle{plain}

\frontmatter % Introducción, índices ...

\input{previo.tex}
\cleardoublepage

\tableofcontents
\listoffigures
\listoftables

\mainmatter % Contenido en si ...

\chapter{Introducción}
\label{cap:introduccion}
% -*-cap1.tex-*- Este fichero es parte de la plantilla LaTeX para la realización
% de Proyectos Final de Carrera, protejido bajo los términos de la licencia
% GFDL.  Para más información, la licencia completa viene incluida en el fichero
% fdl-1.3.tex

% Copyright (C) 2009 Pablo Recio Quijano

\section{Introducción}

Este documento describe el proceso que hemos seguido para la realización del
proyecto.  Comenzamos con la recopilación de requisitos obteniendo al tinal un
documento de Especifacación de Requisitos del Software, realizada a través de
las sucesivas entrevistas con los tutores del proyecto de fin de carrera.

Posteriormente seleccionamos la tecnología más adecuada para llevar a cabo el
proyecto. A continuación pasamos a desarrollar los componentes del software que
añadiremos al sistema para que cumpla con los requisitos que hemos recopilado
previamente.

Para finalizar, seguiremos un desarollo software ágil, por el cual vamos
desarollando partes del sistema de forma que sean totalmente funcionales una vez
terminadas hasta conformar el proyecto según las especificaciones que se han
recogido.

\section{Objetivos}

Los objetivos de este proyecto son por una parte poner en práctica los
conocimientos adquiridos a lo largo de la formación académica recibida en la
titulación de \cursiva{Ingeniería Técnica en Informática de Gestión} y a la vez
aprender a desarrollar bajo una novedosa tecnología web basada en un
\programa{framework} basado en el patrón Modelo-Vista-Controlador (\sigla{MVC})
denominado \programa{Ruby On Rails} (\sigla{MVC}).

Este \programa{framework}, a su vez, basa su simplicidad en el uso de un
lenguaje de \programa{script} llamado \programa{Ruby} que tiene características
propias de lenguajes funcionales; además, es uno de los exponentes de una
perspectiva de desarrollo en auge que se engloba dentro de las denominadas
metodologías ágiles.

El resultado de este proyecto es el desarrollo de una aplicación informática
\programa{web} de código abierto para realizar la simulación virtual de la
\cursiva{gestión computerizada} de un club de fútbol, la simulación de los
partidos que se jueguen, además de la interacción con otros usuarios. En
definitiva: un videojuego multijugador con interfaz web.

Pasamos a enumerar los objetivos de forma esquemática:

\begin{itemize}
\item \negrita{Gestión de usuarios:} todas las herramientas necesarias para
  identificar a cada usuario de nuestra aplicación web de forma unívoca.
\item \negrita{Gestión de clubes: } todas las herramientas necesarias para
  gestionar precios de entrada, venta, compra, entrenamiento de futbolistas y
  tipos de entrenamientos.
\item \negrita{Gestión de campeonatos: } todas las herramientas necesarias para
  que el administrador pueda controlar el calendario de competición, y la
  creación es de nuevas ligas en función de los usuarios que se vayan
  registrando en la web.
\item \negrita{Simulación de partidos: } desarrollaremos un simulador que
  mediante diferentes ponderaciones sea capaz de simular la realidad de un
  partido de fútbol 11.
\end{itemize}

\section{Alcance}
Se desea informatizar toda la gestión y simulación de un entorno de competición
de fútbol 11 de forma que acabemos obteniendo un videojuego multijugador
interactivo con interfaz web.

\subsection{Identificación del producto mediante un nombre}
El producto realizado se llama ``Football Simulator''.

A lo largo de toda la memoria nos referiremos a él como
\negrita{\cursiva{Simulador de fútbol online}}.

\subsection{Funcionalidades del producto}
El producto que deseamos obtener tal y como hemos mencionado anteriormente es
capaz de simular la gestión, interacción y desarrollo de una competición de liga
de fútbol 11, proporcionando funcionalidades de interacción entre usuarios y del
administrador para controlar el comportamiento de la competición.

Por otra parte, nuestro producto también es capaz de simular mediante diversas
ponderaciones los resultados de un partido de fútbol.

Para disfrutar de las funcionalidades de nuestro producto es necesario el
registro y participación de varios usuarios en él, además de la interacción de
un administrador que llevará a cabo acciones de control, supervisión y
seguimiento del desarrollo de la competición.

\subsection{Aplicaciones del software: beneficios, objetivos y metas}

Con este producto podemos obtener diversas ventajas ya descritas anteriormente,
entre ellas cabe destacar la gestión de los equipos de fútbol y la interacción
de usuarios.

El principal objetivo de la utilización del producto es divertir y entretener al
usuario, mediante la inmersión en un entorno virtual de gestión de clubes con
diversos usuarios que compiten entre sí para alcanzar la posición más alta en
clasificación y división que puedan.

Al fin y al cabo lo que se desea es formar la comunidad más grande posible y
para ello es necesario hacer que el usuario desee acceder asiduamente a nuestra
aplicación.

La mayor ventaja es que al ser una aplicación web, cualquier usuario puede
registrarse sin ningún tipo de restricción.

\section{Definiciones, acrónimos y abreviaturas}

\begin{itemize}
\item \negrita{Framework: } Estructura conceptual y tecnológica de soporte
  definida, normalmente con artefactos o módulos de sofware concretos, con base
  en la cual otro proyecto software puede ser organizado y desarrollado.
\item \negrita{Patrón Modelo Vista Controlador (MVC): } Es un estilo de
  arquitectura software que separa los datos de una aplicación, la interfaz de
  usuario y la lógica de control en tres componentes distintos.
\item \negrita{COC (Convenction over Configuration): } En español significa
  \cursiva{``Convención sobre configuración''}, quiere decir que toda pieza de
  código tiene su lugar y todas ellas interactúan según un camino estándar.
\item \negrita{DRY (Don't Repeat Yourself): } En español significa \cursiva{``No
    te repitas''}. Se refiera a que es innecesario repetir código si éste está
  bien diseñado y refactorizado.
\item \negrita{Programación Orientada a Objetos (POO): } . Es un paradigma de la
  programación que se basa en la utilización de clases de objetos.
\item \negrita{Simulador de Fútbol Online:} es el nombre utilizado comúnmente
  durante todo el documento para hacer referencia al nombre del proyecto.
\item \negrita{Ruby: } Lenguaje de programación orientado a objetos de script
  con características propias de lenguajes funcionales.
\item \negrita{Gema: }
\item \negrita{Ruby On Rails (RoR): } Framework de desarrollo rápido de
  aplicaciones web basado en el lenguaje de programación \programa{Ruby}.
\item \negrita{GIT: } Sistema de control de la configuración distribuido.
\item \negrita{Integrated Development Environment (IDE): } En español significa
  \cursiva{``Entorno integrado de desarrollo''}. Es un programa informático
  compuesto por un conjunto de herramientas y módulos de programación que ayudan
  al desarrollo de productos software.
\item \negrita{NetBeans: } IDE con funcionalidades que ayudan al desarrollo de
  aplicaciones \programa{Ruby on Rails}
\item \negrita{Sistema de Gestión de Base de Datos (SGBD)}:
\item \negrita{Oracle: } Es un sistema de Gestión de Base de Datos privativo.
\item \negrita{MySQL: } Sistema de gestión de base de datos de software libre.
\item \negrita{SQLite2: } Sistema de gestión de base de datos de software libre.
\item \negrita{SQLite3: } Sistema de gestión de base de datos de software libre.
\item \negrita{Postgre: } Sistema de gestión de base de datos de software libre.
\item \negrita{Cascade Style Sheet (CSS):} En español significa \cursiva{``Hojas
    de estilo en cascada''}.
\item \negrita{Hypertext Markup Language (HTML))} En español significa
  \cursiva{``Lenguaje de marcado de hipertexto''}.
\item \negrita{JavaScript: } Lenguaje de script que se ejecuta en el lado
  cliente.
\item \negrita{JQuery: } Librería de desarrollo rápido JavaScript.
\item \negrita{Algoritmo criptográfico: } En computación y criptografía un
  algoritmo criptográfico es un algoritmo que modifica los datos de un documento
  con el objeto de alcanzar algunas características de seguridad como
  autenticación, integridad y confidencialidad.
\item \negrita{Algoritmo MD5: } En criptografía, MD5 (abreviatura de
  Message-Digest Algorithm 5, Algoritmo de Resumen del Mensaje 5) es un
  \cursiva{algoritmo criptográfico} de 128 bits ampliamente usado.
\item \negrita{Log:} Archivo de texto plano que nos indica las todas las
  acciones que se realizan en nuestra aplicación con la finalidad de conseguir
  auditar el uso de la misma.


\end{itemize}

\section{Visión General}
Esta memoria sigue las pautas descritas en el documento de referencia
\cursiva{``Recomendaciones para la documentación del Proyecto de fin de
  Carrera''} realizada por varios profesores del departamento de lenguajes y
sistemas informáticos.

Las partes que componen el documento son:

\begin{enumerate}
\item \negrita{Introducción: } Visión general. Objetivos, alcance y estructura.
\item \negrita{Planificación: } Planificación temporal del desarrollo del
  proyecto y porcentaje de esfuerzo dedicado a las diversas tareas y fases que
  lo forman.
\item \negrita{Descripción general del proyecto: } Ampliación de la visión
  global del proyecto. Perspectiva, funciones y restricciones.
\item \negrita{Análisis: } Identificación de las metas globales, perspectivas
  del cliente y recogida de información necesaria para llevar a cabo el
  proyecto.
\item \negrita{Diseño: } Aplicamos diferentes técnicas y procedimientos para
  obtener un producto con el suficiente detalle que permita su realización
  física.
\item \negrita{Codificación: } Aspectos más relevantes de la implementación de
  la aplicación software.
\item \negrita{Pruebas: } Todo lo concerniente a las pruebas para que la
  aplicación logre su cometido sin fallos.
\item \negrita{Resumen: } Breve resumen de lo más destacable del proyecto
  realizado.
\item \negrita{Conclusiones: } Valoración global, posibles mejoras y
  ampliaciones del proyecto.
\item \negrita{Bibliografía: } Libros y referencias físicas y electrónicas
  consultadas para la elaboración del proyecto.
\item \negrita{Apéndices: }
  \begin{itemize}
  \item \negrita{Manual de usuario: } Manual del funcionamiento de la aplicación
    destinado a los usuarios de la misma.
  \item \negrita{Manual de instalación y puesta en funcionamiento: }
    Instrucciones para realizar la instalación de la aplicación, así como su
    puesta en funcionamiento.
  \item \negrita{Licencia: } Texto de la licencia adoptada para la divulgación
    del proyecto.
  \end{itemize}
\end{enumerate}



\chapter{Planificación}
\label{cap:planificacion}
% -*-cap2.tex-*- Este fichero es parte de la plantilla LaTeX para la realización
% de Proyectos Final de Carrera, protejido bajo los términos de la licencia
% GFDL.  Para más información, la licencia completa viene incluida en el fichero
% fdl-1.3.tex

% Copyright (C) 2009 Pablo Recio Quijano

\section{Metodología de desarrollo}
La distribución temporal seguida en líneas generales es la descrita a
continuación, de la cual se harán revisiones constantes al tratarse de una
aplicación en la que se desea aprender a desarrollar bajo una metodología ágil.
\begin{itemize}
\item Análisis: En esta fase llevaremos a cabo la recogida de todos los
  requisitos funcionales de nuestra aplicación. Al utilizar una metodología ágil
  la recogida de requisitos podrá ser más laxa que si hubiéramos seguido un
  modelo lineal.
\item Diseño: En esta fase recrearemos diseños conceptuales de cómo serán
  nuestras vistas, y a partir de ellas podremos obtener nuestros modelos de
  datos.
\item Programación: En esta fase nos dedicaremos a codificar de forma sinérgica
  las diferentes secciones en las que se compone un desarrollo basado en el
  patrón Modelo Vista Controlador
\item Pruebas: Las pruebas que realizaremos serán:
  \begin{enumerate}
  \item Test de usabilidad
  \item Pruebas de caja negra.
  \item Estudio y resolución de los resultados de las pruebas.
  \end{enumerate}
\item Documentación: Para la documentación del código usaremos la herramienta
  propia del lenguaje Ruby, Ruby-doc, gracias a esta aplicación podremos generar
  un documento web fácilmente extensible y manipulable. En esta fase también
  desarrollaremos la memoria de la aplicación, el manual de usuario y la
  presentación multimedia para difundir nuestra aplicación.
\end{itemize}

\section{Calendario}

En la figura \ref{fig:gantt} se puede ver un diagrama de Gantt con la división y
el tiempo tomado para realizar cada una de las tareas en las que hemos dividido
el proyecto. La unidad de tiempo básica está expresada en días laborables de 8
horas.

Aunque la etapa de Memoria se ha realizado de forma solidaria con las demás,
para simplificar se ha extraído en un sólo bloque.

\begin{itemize}
\item Entrevistas: Nos reunimos con nuestros tutores de proyecto para recopilar
  los requisitos que deberá tener nuestro proyecto de fin de carrera para que
  sea lo suficientemente completo para pasar la aprobación del tribunal y
  aprender de forma exhausta el framework \cursiva{Ruby On Rails} (5 días).
\item Análisis: Una vez recopilada toda la información en las entrevistas con
  nuestros tutores, extraemos la información importante y la esquematizamos (7
  días).
\item Especificación de Requisitos: Definimos, organizamos, cada uno de los
  requisitos que hayamos extraído y los desarrollamos de forma exhausta. En esta
  fase también describiremos los casos de uso, evitando en la medida de lo
  posible la redundancia de los mismos y maximizando la posibilidad de
  reutilizarlos (10 días).
\item Diseño: En esta fase estudiaremos la mejor forma para incorporar la
  funcionalidad del sistema teniendo en cuenta que usaremos el framework de
  desarrollo rápido de aplicaciones \cursiva{Ruby On Rails}, con especial
  atención a que haremos uso del patrón Modelo-Vista-Controlador (25 días).
\item Implementación: En esta fase codificaremos la aplicación para que sea
  funcional según hemos especificado en la fase de diseño; además, será
  necesario aprender la forma de utilizar de forma correcta el framework
  \cursiva{Ruby On Rails)}. En esta fase va incluida la creación de la
  estructuras necesarias para la interacción con el Sistema de Gestión de Bases
  de Datos (\cursiva{SGBD}) a través de un sistema \cursiva{ORM} (90 días).
\item Pruebas: En este paso comprobaremos que la corrección del sistema que
  hemos desarrollado. En caso de encontrar algún fallo es el momento de
  solucionarlo y repetir las pruebas (5 días).
\item Memoria: Toda la documentación que conforma esta memoria es el resultado
  de detallar cada uno de los documentos que hemos ido generando a través de la
  realización de la misma (40 días).
\item Presentación: Creación y preparación de la presentación que mostraremos en
  la etapa de defensa ante el tribunal (10 días).
\end{itemize}

\section{Presupuesto}

Para realizar una estimación del montante económico del proyecto realizado será
necesario calcular y dirimir todos los recursos que han sido necesarios para
llevarlo a cabo. En dicho presupuesto no se incluirán los materiales de oficina
como serían escritorio, folios, ordenador, impresora, y demás recursos
ofimáticos.

Sabiendo que excluiremos todos los gastos que podríamos considerar como
indirectos, sólo tendremos que fijarnos en el gasto que supondría el coste de
mantener a un \cursiva{Analista-programador y Diseñador de página Web} en la
empresa. Para ello consultaremos las tablas salariales de dicha categoría
profesional en el \cursiva{XVI CONVENIO COLECTIVO ESTATAL DE EMPRESAS
  CONSULTORAS, DE PLANIFICACIÓN, ORGANIZACIÓN DE EMPRESA Y CONTABLE, EMPRESAS DE
  SERVICIOS DE INFORMÁTICA Y DE ESTUDIOS DE MERCADO Y DE LA OPINIÓN PÚBLICA} tal
y como vemos en la tabla \ref{tab:salario}. Para el cálculo del pago de la
Seguridad Social aplicaremos un 33\% del coste del salario bruto del trabajador.

\begin{table}[H]
  \begin{center}
    \begin{tabular}{| c | c | c | c |}
      \hline
      Nombre & Coste mensual & Coste anual & Coste jornada (25 días por mes)\\ \hline
      Analista-programador y Diseñador Web & 1.468,54 \euro & 20.559,56
      & 58,75 \euro\\ \hline
      Seguridad Social & 484,82 \euro & 6.784,65 \euro & No aplica \\
      \hline
    \end{tabular}
  \end{center}
  \caption{Salario según último convenio de las TIC}
  \label{tab:salario}
\end{table}


Teniendo en cuenta que el proyecto lo hemos realizado en solitario y nos ha
llevado un total de 192 jornadas laborales, lo que equivaldría a 7 meses y 3
semanas naturales, tendremos que el coste total del proyecto es el referido en
la tabla \ref{tab:coste}

\begin{table}[H]
  \begin{center}
    \begin{tabular}{| c | c | c | c |}
      \hline
      Cantidad & Descripción & Coste unitario & Coste Total\\ \hline
      192 & Jornada Analista-Programador y Diseñador Web & 58,75 \euro &
      11.280 \euro\\ \hline
      8 & Seguridad Social Analista-Programador y Diseñador Web & 484,82
      \euro & 3.878,56 \euro \\ \hline
      \multicolumn{3}{|r|}{Total} &  15.158,56 \euro\\
      \hline
    \end{tabular}
  \end{center}
  \caption{Coste total del proyecto realizado}
  \label{tab:coste}
\end{table}

\chapter{Descripción General del proyecto}
\label{cap:descripcion}
\section{Perspectiva del Producto}
En este capítulo explicaremos qué pretendemos conseguir con la realización de
este proyecto y qué características generales ha de cumplir.

\subsection{Dependencias del producto}
El proyecto es independiente, no es un subsistema de un proyecto de mayor
envergadura, tampoco es continuación de ningún otro proyecto, pero al estar
liberado con una licencia de \cursiva{software libre} podría ser continuado,
mejorado y actualizado por su propio creador o terceras personas ajenas a éste.

\subsection{Interfaces de usuario}
La interfaz de usuario se basa en una página web dinámica que lleva asociada
varias opciones para poder usar las herramientas proporcionadas.

Para el uso de la aplicación, la página principal dispone de varios menús, entre
ellos la de selección de idiomas y un \cursiva{hipervínculo} al subsistema de
registro e inicio de sesión.

Todo el diseño de la aplicación respeta los estándares \cursiva{HTML 4.0} y
\cursiva{CSS 2.0 } así como el uso de la librería de \cursiva{JavaScript JQuery}
lo que nos proporciona una gran compatibilidad con cualquier navegador web del
mercado.

Al usar \cursiva{hojas de estilo CSS} la apariencia de nuestra aplicación puede
ser fácilmente modificada para adaptarla a nuevas tendencias de diseño de forma
sencilla y eficiente.

Siempre que se termine cualquier operación, ya sea exitosa o no, el sistema nos
informará a través de un sistema de notificación embebido dentro del navegador
muy parecido al sistema usado en \cursiva{Mac OS X } y últimas versiones de la
\cursiva{distribución Linux Ubuntu}.

Como mejora para un uso mucho más cómodo y rápido por parte de cualquier
usuario, la aplicación dispone de un menú estático para acceder a cualquier tipo
de herramienta o funcionalidad según el usuario sea un jugador de la aplicación
o sea el administrador.

\subsection{Interfaces software}
El producto interactúa con el sistema operativo en el lado del servidor en dónde
está instalado y con el navegador usado en el lado del cliente.

De hecho, el producto es totalmente multiplataforma tanto en el lado servidor
como en el lado cliente, gracias a que el \cursiva{framework RoR} está diseñado
para ser instalado en entornos \cursiva{Windows, Linux y Mac OS X}; mientras que
en el lado cliente disponemos de varios navegadores web cualquiera que sea el
sistema operativo en el que se ejecute.

Debemos destacar que el mencionado \cursiva{framework} tiene una gran comunidad
que lo sustenta, que da como resultado la disponibilidad de una gran cantidad de
documentación y un buen soporte de la plataforma, repercutiendo directamente en
la calidad del software que desarrollemos.

Hemos de indicar que \cursiva{RoR} es agnóstico en cuanto al \cursiva{SGBD} que
utilicemos, con lo que el producto podrá ser desplegado con cualquiera de los
\cursiva{SGBD} más comunes como son \cursiva{Oracle, MySQL, SQLite2, SQLite3 y
  Postgre}, a través de una \cursiva{gema} específica y una simple modificación
en uno de los archivos de configuración.

\subsection{Operaciones}
Existen dos tipos de usuarios dentro de la aplicación:

\begin{itemize}
\item \negrita{Superusuario: } Es el responsable de instalar, mantener,
  interconectar con la \cursiva{base de datos} y optimizar el servidor dónde se
  encuentra la aplicación.
\item \negrita{Administrador: } Es el responsable de toda la gestión de
  campeonatos, así como de dar de alta a los nuevos usuarios para que puedan
  empezar a interactuar con las funcionalidades de la aplicación.
\item \negrita{Usuario: } Conforma el resto de usuarios del sistema que
  participan en la simulación del entorno de gestión de equipos de fútbol, a
  excepción de las funcionalidades exclusivas del administrador.
\end{itemize}

\subsection{Requisitos de adaptación a la ubicación}
Para poder adaptar la aplicación al entorno dónde se instalará será necesario
disponer de un \cursiva{SGBD} compatible con \cursiva{RoR} además de modificar
el fichero de configuración para conectarlo con nuestro sistema.

\subsection{Funciones del producto}
La función principal del producto que estamos presentando es la de desarrollar
un videojuego multijugador dónde diversos usuarios puedan interactuar en un
entorno que simule ser real a través de \cursiva{Internet}.  Para ello ha sido
necesario crear una aplicación informática encargada de poder registrar a
cualquier usuario que acceda a nuestra web y proporcionarles las herramientas
necesarias para gestionar su propio club; además de simular los partidos que en
una competición de fútbol se desarrollan.

Las características más importantes de esta aplicación se describen a
continuación:

\begin{itemize}
\item Posibilidad de registro y control de acceso a una cuenta de usuario
  personal e intransferible que distinga al usuario unívocamente de los demás a
  través de una simple interfaz web.
\item Capacidad para la gestión de todos los aspectos económicos de los clubes
  de los usuarios registrados.
\item Capacidad para la visualización del calendario y clasificación de todos
  los clubes y usuarios registrados en el sistema.
\item Capacidad para modificar la estructura deportiva y técnica de los clubes
  de los usuarios registrados.
\item Capacidad para simular un partido de forma realista y mostrarlo a los
  usuarios como si se estuviera desarrollando en directo.
\item Capacidad para gestionar el estado de la competición.
\item Capacidad para escoger el idioma en el que se mostrará la información de
  nuestra aplicación.
\end{itemize}

Aquí sólo se han descrito las características generales que realiza la
aplicación, se podrá comprobar toda la funcionalidad en los puntos siguientes de
esta memoria, en especial en el \cursiva{Manual de usuario}.

\section{Características de los usuarios}

Para el correcto manejo de la aplicación \cursiva{Simulador de fútbol online}
existen dos perfiles distintos de manejo.

Podríamos dividir los perfiles en:

\begin{itemize}
\item \negrita{Usuario (Jugador):} Necesitan unos conocimientos básicos de
  informática para su manejo, sobre todo orientado al manejo de aplicaciones
  web. También será necesario un conocimiento meramente básico de cuáles son las
  reglas de negocio y deportivas de un entorno de gestión de clubes deportivos
  de fútbol 11.
\item \negrita{Usuario (Administrador): } Necesitará unos conocimientos básicos
  en informático para su manejo, sobre todo orientado al manejo de aplicaciones
  web. También será necesario conocer cómo es el discurrir de las competiciones
  en el diseño de nuestra aplicación.
\item \negrita{Superusuario (Administrador): } Serán necesarios unos
  conocimientos avanzados de informática, incidiendo en conocimientos de
  despliegue de aplicaciones web.
\end{itemize}

\section{Restricciones generales}
\label{sec:restricciones_generales}
El proyecto se ha desarrollado bajo la metodología \cursiva{Rational Unified
  Process (RUP)}, proceso de desarrollo orientado a objetos, al estar basado
todo el proyecto en un entorno de desarrollo bajo el paradigma de programación
orientada a objetos.

El lenguaje de programación que usaremos será \cursiva{Ruby}, que es libre así
como el \cursiva{framework} sobre el que apoyaremos nuestra aplicación web
\cursiva{RoR}.

No se deben tener en cuenta restricciones en el uso de memoria principal por
parte de la aplicación tanto en el lado servidor como en el cliente.

Se deben tener en cuenta restricciones a la hora de crear las vistas de nuestra
aplicación web para que éstas sean lo mas livianas posibles para ahorrar el
mayor ancho de banda posible.

Para el desarrollo de todo el proyecto deberemos usar un \cursiva{IDE} que se
base en la filosofía de \cursiva{software libre}, en este caso
\cursiva{NetBeans}.

El producto habrá de funcionar en cualquier sistema operativo que tenga
disponibilidad para tener instalado un intérprete de \cursiva{Ruby} en el lado
servidor y en cualquier navegador del mercado compatible con los estándares web
para \cursiva{HTML 4.0}, \cursiva{CSS 2.0}, y la librería \cursiva{JQuery}. Esto
hará que nuestro software tenga un mayor público objetivo de desarrolladores.

Como sistema gestor de bases de datos se podrá utilizar cualquiera que tenga una
\cursiva{gema} que sea capaz de conectar \cursiva{RoR} con el \cursiva{SGBD}
escogido.

Es necesario tener en cuenta que nuestra aplicación puede correr en cualquier
\cursiva{pc} pero es conveniente que éste tenga la capacidad suficiente para
atender varias peticiones a la vez, además de tener la configuración necesaria
para que peticiones que lleguen desde \cursiva{internet} puedan ser atendidas.

Para toda la creación y generación de la documentación presente hemos utilizado
LaTeX.

Como controlador de la configuración hemos optado por usar \cursiva{GIT}, un
sistema de control de versiones distribuido que es el usado oficialmente por el
proyecto \cursiva{RoR} y que se integra de forma plena con \cursiva{NetBeans}.

\section{Requisitos no funcionales del sistema}
\subsection{Requisitos Generales}
La solución debe cumplir como mínimo con las siguientes características basadas
en las especificaciones funcionales y los requisitos no funcionales:

\begin{itemize}
\item Basada en la web.
\item La aplicación debe esta diseñada y desarrollada sobre la plataforma
  \cursiva{Ruby On Rails}.
\item La aplicación debe ser escalable bajo estrategia vertical (Añadir más
  recursos al servidor) y horizontal (Añadir más servidores), según las
  necesidades de procesamiento.
\item La aplicación debe tener bajo nivel de acoplamiento y la posibilidad de
  editar fácilmente los parámetros que se consideren dinámicos y requieran
  cambios frecuentes.
\item Orientada a objetos.
\item De fácil mantenimiento en cuanto a cumplimiento de estándares, uso de
  guías y patrones con especial énfasis en el patrón \cursiva{MVC},
  documentación y de fácil ubicación de componentes.
\item Que permita y utilice la reutilización de código.
\item Basada en la arquitectura cliente-servidor.
\item La aplicación debe permitir generar avisos a través de correo electrónico.
\end{itemize}

\subsection{Requisitos específicos}
Los requisitos no funcionales generales estarán enmarcados en los siguientes
aspectos:
\begin{itemize}
\item Escalabilidad:
  \begin{itemize}
  \item El diseño debe contemplar el uso óptimo de recursos como las conexiones
    con la base de datos.
  \item El diseño debe contemplar la clara división entre datos, recursos, y
    aplicaciones para optimizar la escalabilidad del sistema.
  \item Debe contemplar requisitos de crecimiento para usuarios internos al
    sistema.
  \end{itemize}
\item Disponibilidad:
  \begin{itemize}
  \item La disponibilidad del sistema ha de ser continua con un nivel de
    servicio para los usuarios de siete días y 24 horas.
  \item En caso de fallos de algún subsistema no debe haber pérdida de
    información
  \item Debe contemplar requisitos de consistencia transaccional.
  \end{itemize}
\item Seguridad:
  \begin{itemize}
  \item La aplicación debe reflejar patrones de seguridad teniendo para cumplir
    con la ley orgánica de protección de datos.
  \end{itemize}
\item Mantenibilidad:
  \begin{itemize}
  \item Se debe estructura el código de manera consistente y predecible.
  \item Aprovechar al máximo las facilidades que nos aporta el
    \cursiva{framework Ruby On Rails}, con especial énfasis en el patrón
    \cursiva{MVC}.
  \end{itemize}
\item Rendimiento: La aplicación debe ofrecer una buena respuesta ante alta
  demanda de usuarios.
\end{itemize}
\section{Requisitos para futuras versiones}
Para futuras versiones del producto, podrían introducirse funciones avanzadas
para los usuarios como selección de escudo del equipo, personalización de
dorsales, avisos por sms, mercados de agentes libres, cantera y ojeadores.

Podría introducirse nuevas plantillas de diseño para personalizar la vista de
diversas formas para que el usuario escoja la que más le guste.

También sería interesante establecer un apartado de estadísticas para poder
llevar un seguimiento del historial en tiempo de la evolución de los equipos
tanto en el plano técnico como en el económico, este apartado también sería
aplicable a los futbolistas que integran los clubes.

Podría añadirse un sistema de foros y chat para una interacción entre usuarios
que no se limite en exclusiva a las funcionalidades específicas de nuestro
sistema.

Como restricción a futuras versiones o proyectos que estén basados en nuestra
aplicación se aplicarán las referentes a las propias especificadas en el
apartado de licencia de este documento.


\chapter{Análisis}
\label{cap:analisis}
% -*-cap4.tex-*-
% Este fichero es parte de la plantilla LaTeX para
% la realización de Proyectos Final de Carrera, protejido
% bajo los términos de la licencia GFDL.
% Para más información, la licencia completa viene incluida en el
% fichero fdl-1.3.tex
% Copyright (C) 2009 Pablo Recio Quijano 
\section{Metodología de desarrollo}
Hemos decidido seguir una metodología de desarrollo que se adapte lo
suficientemente bien al framework en torno al cual desarrollaremos
nuestra aplicación.\\

Hemos decidido usar la metodología de desarrollo \cursiva{Xtreme
  Programming (XP)}

La metodología será seguida dentro de unos límites y adaptada a
nuestra necesidades, debido entre otras cosas a que somos nosotros
mismos los que hemos decidido qué funcionalidades y qué problemas
queremos resolver con nuestra aplicación; además, no hay que olvidar
que no tenemos ningún equipo que coordinar ya que sólo será una
persona quién se encargue de realizar todas y cada una de las
funcionalidades.

La programación extrema, o \cursiva{Xtreme Programming} es una de las
llamadas \cursiva{Metodologías Ágiles} de desarrollo de software más
populares hoy en día. Inicialmente fue descrita por \cursiva{Kent
  Beck} cuando trabajaba en la \cursiva{Chrysler Corporation}. En la
\cursiva{XP} se da por supuesto que no se debe prever todo antes de
empezar a codificar, que es imposible capturar todos los requisitos
del sistema, ni saber qué es todo lo que debe hacer y por tanto es
imposible hacer un diseño correcto desde el principio.

La idea principal de esta metodología consiste en trabajar
estrechamente con el cliente, diseñando pequeñas versiones
frecuentemente. El objetivo de estas versiones no es otro que
conseguir que la aplicación funcione de la forma más simple y
eficiente posible con el mínimo código. Cuando el cliente le transmite
al programador lo que necesita, éste puede hacer una estimación
aproximada del tiempo que le llevará codificarlo, aunque lógicamente
dicha planificación deberá revisarse y modificarse continuamente a lo
largo del desarrollo del proyecto, es decir, se va iterando. Cada vez
que el desarrollador consigue una versión funcional de lo que el
cliente ha solicitado, ésta se le muestra al cliente para que la
testee y haga peticiones de las modificaciones que crea
convenientes. De esta manera se evita perder tiempo desarrollando una
aplicación que no sea la que el cliente esperaba. Este ciclo se
repetirá tantas veces como el cliente necesite para sentirse
satisfecho con la aplicación.

La \cursiva{Xtreme Programming} agrupa trece prácticas básicas que se
deben cumplir para asegurar el éxito del proyecto, ellas son:

\begin{itemize}
  \item Equipo completo.
  \item Planificación continua.
  \item Test del cliente.
  \item Versiones pequeñas.
  \item Diseño simple.
  \item Programación por parejas.
  \item Desarrollo guiado por pruebas automáticas.
  \item Mejora continua del diseño.
  \item Integración continua.
  \item Código en multipropiedad.
  \item Normas de codificación comunes.
  \item Utilización de metáforas.
  \item Mantener un ritmo sostenible.
\end{itemize}

La \cursiva{Xtreme Programming} consta de unos valores, unos
principios fundamentales y unas prácticas. Los principios
fundamentales en los que se basa son:

\begin{itemize}
  \item Feedback: Retroalimentación veloz.
  \item Modificaciones incrementales.
  \item Refactorización
  \item Asunción de simplicidad.
  \item Respeto: Comunicación entre desarrolladores.
\end{itemize}



\section{Descripción General}
A continuación vamos a ver los factores que afectan al producto y sus
requerimentos

\section{Propósito}
El propósito del siguiente apartado es definir cuáles son los
requisitos que debe tener nuestro portal web que simula la gestión de
un club de fútbol en un entorno multijugador y simular partidos.\\

Esta especificación de requisitos está destinada a ser leída por los
usuarios o cualquier sujeto que tenga interés en saber cómo funciona
el producto.

\section{Ámbito}
El producto que vamos a describir puede clasificarse como un
videojuego multijugador online integrado en un servidor web en la cual
los usuarios del portal se dedicarán a gestionar un club de fútbol en
la parcela económica y deportiva, y dispondrá de un simulador de los
partidos entre clubes.\\

El producto ha de ofrecer una interfaz lo bastante sencilla para que
cualquier usuario pueda gestionar las distintas facetas en escasos
minutos. Dado que la aplicación es de tipo web, deberemos de tener
especial cuidado en respetar los estándares para que la aplicación sea
visible en cualquier navegador del mercado obteniendo así un
videojuego que sea totalmente multiplataforma. La tecnología que
usaremos en el lado del servidor web será el framework Ruby On Rails.

\section{Perspectiva del Producto}
El presente producto ha de correr en un servidor que sea compatible
con \programa{Ruby on Rails}. Antes que nada \programa{Ruby On Rails} es un framework que
hace mas fácil desarrollar, implementar y mantener aplicaciones
web. Esto es debido entre otras razones a que:

\begin{itemize}
\item Toda aplicación web implementada usa el esquema \sigla{MVC}
  (Modelo Vista Controlador).
  
\item Toda pieza de código tiene su lugar y todas ellas interactúan
  según un camino estándar, es lo que ha venido a llamarse
  \sigla{COC} (Convención sobre Configuración).
  
\item Las aplicaciones en Rails están escritas en Ruby, un lenguaje de
script moderno y orientado a objetos que ayuda a que se potencie otro
de los pilares de la filosofía de desarrollo que viene abreviado en la
siguiente nomenclatura: \sigla{DRY} (No te repitas).
\end{itemize}

En cuanto a la máquina necesaria para ejecutar el servidor, valdrá
cualquier sistema que sea compatible con \programa{Windows}, \programa{Linux} o \programa{MacOS X}
recomendándose que se use un hardware acorde con las peticiones que el
portal reciba. Los usuarios deberán de acceder al portal mediante
cualquier navegador web disponible en el mercado que sea compatible
con los estándares web \sigla{XHTML 1.1 Strict} y \sigla{CSS 2.0.}

\section{Características del usuario}
A continuación vamos a ver a qué tipo de usuarios está dirigido el
producto y cómo afectan éstos a las funciones que debe realizar
nuestro producto.\\

El producto está dirigido a todo usuario que sepa manejar un navegador
web y que esté interesado en la simulación de la gestión de un club de
fútbol, así como conocer las reglas y premisas básicas de negocio del
deporte.
\section{Obligaciones generales}
Uno de los aspectos que influirá determinantemente en el éxito del
producto será la eficiencia en las peticiones y el diseño de una
intertaz amigable y sencilla para poder gestionar los diferentes
parámetros del club de una forma rápida y precisa.\\

De cara al diseño de la interfaz web, uno de los mayores desafíos con
los que nos encontraremos será el conseguir que el diseño de la
interfaz sea operativa en cualquiera de los navegadores del mercado,
ya que los dos navegadores más populares no soportan actualmente con
suficiente solvencia los estándares \sigla{CSS 2.0} (\programa{Internet Explorer 8} y
\programa{Firefox 3.5}).
\section{Asunciones y Dependencias}
El presente producto depende enteramente de que el usuario disponga de
una conexión red al servidor a través de un navegador web
(probablemente \sigla{internet}), por tanto es importante respetar los
estándares pertinente que estipula la \sigla{w3c}.\\

\section{Especificación de los requisitos del sistema}
\subsection{Requisitos de interfaces externas}
\subsubsection{Requisitos de interfaces de usuario}
Podemos dividir los requisitos de interfaz de usuario según tres roles:

\begin{itemize}
  \item \negrita{Usuario Jugador y Usuario Administrador: }
    \begin{itemize}
      \item Interfaz atractiva para los usuarios: Es fundamental para
        la buena aceptación de nuestra aplicación el crear una
        interfaz lo suficientemente clara, dinámica y estética para
        que el usuario sienta una inmersión en la experiencia de
        juego. Una interfaz no atractiva sólo traerá problemas de
        rechazo y disconformidad por parte del usuario.\\\\
        Para que la interfaz sea sencilla de modificar y tratándose de
        una aplicación web usaremos \cursiva{hojas de estilo CSS}, así
        todo el apartado de diseño gráfico podrá ser modificado
        posteriormente sin que por ello afecte al contenido ni la
        estructura del documento \cursiva{HTML}.
      \item Una característica de nuestra aplicación es que el sistema
        de avisos está embebido dentro de la página web y es muy
        parecido al sistema de notificación de los sistemas operativos
        \cursiva{Mac OS X} y de la  \cursiva {distribución Linux Ubuntu}.
    \end{itemize}
  \item \negrita{Superusuario: }
    En este caso la interfaz corresponderá a la que tenga el propio
    sistema operativo dónde esté instalado la aplicación.
\end{itemize}

\subsection{Requisitos de rendimiento}
Este aspecto es fundamental en el análisis del sistema, se ha de tener
en cuenta que la aplicación es de tipo servidor y que a ella estarán
conectados concurrentemente un número de usuarios elevado.\\

Se ha considerado también el rendimiento que nos ofrece el diseño web
realizado. Para que nuestra página web sea compatible con la mayoría
de dispositivos y se ejecute lo suficientemente fluida hemos decidido
no usar la tecnología propietaria \cursiva{Flash}, usando para la
renderización de efectos la librería \cursiva{JQuery} en combinación
con hojas de estilo \cursiva{CSS}.\\

El rendimiento de la aplicación podrá ser menor en la primera
instanciación de la aplicación debido a que no se habrá cacheado
ningún contenido.

\subsection{Requisitos de la base de datos}
La \cursiva{base de datos} es uno de los componentes fundamentales de
la aplicación. En ella se van a almacenar todos los datos registrados
por la aplicación.\\

La aplicación sólo hará uso de una conexión al \cursiva{SGBD} escogido
por el \cursiva{superusuario} en la configuración de instalación de la
aplicación, cuyo nombre también podrá ser escogido por el citado
\cursiva{superusuario}.\\

Dicha \cursiva{BD} estará formada por las siguientes tablas:

\begin{itemize}
  \item [User] Tabla donde se almacenan los datos de cada uno de los
    usuarios que interactúan con la aplicación.
  \item [Club] Tabla donde se almacenan los datos de los clubes que
    participan en las ligas.
  \item [League] Tabla donde se almacenan los datos de las ligas que
    existen en el sistema.
  \item  [Season] Tabla donde se almacenan los datos de las temporadas
    correspondientes a cada liga del sistema.
  \item [Round] Tabla donde se almacenan los datos de las jornadas de
    cada temporada.
  \item [MatchGeneral] Tabla dónde se almacenan los partidos que se
    han de jugar cada jornada.
  \item [MatchDetail] Tabla dónde se almacenan las acciones que tienen
    lugar en cada partido.
  \item [LineUp] Tabla dónde se almacenan las alineaciones de cada
    equipo y partido.
  \item [ClubFinancesRound] Tabla donde se almacenan los datos de
    finanzas correspondientes a cada jornada y club.
  \item [Player] Tabla donde se almacenan los datos de los jugadores
    de fútbol de los clubes.
  \item [Offer] Tabla donde se almacenan los datos de las ofertas que
    se realizan los usuarios de la aplicación sobre los jugadores de
    fútbol.
  \item [Training] Tabla donde se almacenan los datos de los
    entrenamientos de los jugadores de fútbol.
  \item [AdminMessages] Tabla donde se almacenan los mensajes de
    administración hacia los usuarios.
\end{itemize}

Toda la información sobre la \cursiva{base de datos} será ampliada en
su sección correspondiente.\\

La tabla \cursiva{User} contiene los datos de los usuarios que
interactúan con la aplicación, en ella se almacenará la contraseña de
cada uno de ellos; es por ello que será necesario un
\cursiva{algoritmo criptográfico} denominado \cursiva{MD5}. De esta
forma conseguimos proteger la contraseña de los usuarios frente a un
posible ataque al servidor dónde esté la \cursiva{base de datos}.\\

Es necesario que el \cursiva{superusuario} conozca el \cursiva{SGBD}
que será utilizado con nuestra aplicación, además de disponer de un
usuario y contraseña para la correcta creación de las tablas.

\subsection{Restricciones de diseño}

No existe ninguna restricción considerable en el diseño del sistema.\\

Todo el diseño del sistema se debe de regir con la especificación
realizada.\\

Es importante estimar cuál va ser el tráfico que nuestra aplicación va
sufrir para disponer de un \cursiva{servidor} y una conexión lo
suficientemente potentes para dar servicio a todos los usuarios que se
conecten a la aplicación.

\subsection{Atributos del sistema software}
El sistema ha de cumplir las siguientes propiedades:

\begin{itemize}
  \item \negrita{Fiabilidad}: Este sistema, como todo software de
    calidad, debe ser fiable. Al estar tratando de una aplicación web
    que ha de tener un elevado acceso concurrente a los datos, hemos
    de cerciorarnos que las operaciones no dejen el sistema en un
    estado inconsistente por una posible caída del servidor.\\

    Por ello el sistema se encarga de generar un \cursiva{log} con
    todas las operaciones que ocurren en nuestra aplicación.

    Hemos de tener en cuenta que el usuario de nuestra aplicación no
    ha de conocer nada sobre la estructura interna de nuestra
    aplicación, por ello los mensajes de error serán lo más simples
    posibles para evitar posibles ataques al servidor dónde es
    alojada.

    \item \negrita{Disponibilidad}: La disponibilidad de este producto
      debe ser plena. La aplicación ha de estar disponible en
      cualquier momento y ser accesible desde cualquier lugar. Para
      llevar a cabo este precepto será necesario alojar la aplicación
      en un lugar web que tenga la suficiente capacidad para dar
      servicio a un elevado número de usuarios.

   \item \negrita{Seguridad}: La seguridad de este tipo de
     aplicaciones ha de ser bastante elevada para conseguir cumplir
     leyes de protección de datos y asegurar un correcto
     funcionamiento.

     Para la seguridad y confidencialidad de los datos se debe seguir
     un protocolo dirigido a asegurar la confidencialidad de los datos
     almacenados en la \cursiva{base de datos}.

     Será necesario que el superusuario de la aplicación disponga de
     un nombre de usuario y contraseña lo suficientemente seguros para
     trabajar con el \cursiva{SGBD} escogido.

     Los usuarios de la aplicación tendrán un nombre usuario y una
     contraseña asociada que encriptada con el algoritmo
     \cursiva{MD5}.

     Es posible que cualquier persona acceda a la página web para
     estar informado de los que ocurre en las diferentes ligas que
     componen el juego. Este tipo de usuarios no necesitan disponer de
     nombre de usuario y contraseña asociada.

     \item \negrita{Mantenibilidad}: La mantenibilidad de la
       aplicación es un aspecto clave del desarrollo de aplicaciones
       web debido al constante avance que sufre este tipo de
       tecnologías. Gracias a desarrollar la aplicación en base a un
       \cursiva{framework de desarrollo rápido de aplicaciones}
       conseguimos una alta mantenibilidad y legibilidad del
       \cursiva{código fuente} de la aplicación.

     \item \negrita{Portabilidad}: La aplicación web, al estar
       desarrollada bajo \cursiva{Ruby On Rails} tiene una amplia
       portabilidad en cuánto al lado servidor ya que éste es
       compatible con los principales sistemas operativos del mercado.

       En cuanto al lado cliente tenemos la portabilidad asegurada
       ya que sólo será necesario que el usuario disponga de una
       conexión a \cursiva{internet} y un navegador web compatible con
       los estándares.
\end{itemize}

\section{Descripción de Requisitos Funcionales}

\subsection{Configurar el sitio}
El administrador podrá navegar por la página de configuración web y la
usará para cambiar el comportamiento de la misma.\\

Esta función se realizará raras veces.
\subsection{Crear una cuenta de usuario}
Para que un usuario cualquiera sea capaz de poder usar nuestro
producto ha de tener una cuenta registrada en el servidor que lo
distinga unívocamente de cualquier otro usuario; para ello se le
pedirá introducir un nombre de usuario, y una dirección de e-mail. Una
vez registrado el sistema enviará un e-mail de confirmación de
registro dando la bienvenida a nuestro sistema.\\

Esta función es condición necesaria para que puedan realizarse todos y
cada uno de los requisitos que se expresan en los apartados siguientes
y se realizará una única vez por usuario.

\subsection{Acceder a la cuenta de usuario}
Para que cualquier usuario registrado sea capaz de usar nuestro
producto deberemos de proporcionarle un método de acceso e
identificación mediante el nombre de cuenta y contraseña que han sido
albergados en nuestro servidor.\\

Esta operación se realizará muy a menudo y tantas veces como el
usuario desee.
\subsection{Abandonar la cuenta de usuario}
Cualquier usuario que haya iniciado sesión en nuestra aplicación
deberá ser capaz de abandonar de forma segura la misma.\\

Esta operación se realizará muy a menudo y tantas veces como un
usuario que haya iniciado sesión desee.

\subsection{Abrir equipos}
El administrador será capaz de decidir cuándo los usuarios del sistema
podrán modificar sus alineaciones o tácticas, así como la posibilidad
de interactuar con los sistemas de finanzas y ofertas de jugadores.

Esta función será realizada muy a menudo.

\subsection{Cerrar equipos}
El administrador será capaz de decidir cuándo los usuarios del sistema
no podrán modificar sus alineaciones o tácticas, así como la
posibilidad de interactuar con los sistemas de finanzas y ofertas de jugadores.

Esta función será realizada muy a menudo.

\subsection{Listar ofertas}
El usuario podrá en cualquier momento visionar el listado de ofertas
pendientes, aceptadas o rechazadas que tenga en su historial.

Esta operación podrá ser realizada en cualquier momento.

\subsection{Emitir oferta por futbolista}
El usuario podrá en cualquier momento hacer una oferta por un
futbolista en el que esté interesado, interactuando de este modo con
otros usuarios a través de acuerdos económicos. Para evitar que dos
usuarios puedan realizar acuerdos económicos para hacer trampas y
beneficiarse deportiva y económicamente, éstos no podrán tasar a sus
propios jugadores haciendo de la oferta una especie de pujo, sino que
será el propio sistema el que estipulará el precio del futbolista en
base a la calidad que éste disponga.

Esta operación podrá ser realizada en cualquier momento en el que
los equipos estén abiertos.

\subsection{Aceptar oferta por futbolista}
El usuario podrá en cualquier momento aceptar una oferta de un
futbolista de su propiedad emitida por otro usuario.

Esta operación podrá ser realizada en cualquier momento en el que los
equipos estén abiertos.

\subsection{Rechazar oferta por futbolista}
El usuario podrá en cualquier momento rechazar una oferta de un
futbolista de su propiedad emitida por otro usuario.

Esta operación podrá ser realizada en cualquier momento en el que los
equipos estén abiertos.

\subsection{Cancelar oferta por futbolista}
Un usuario que esté arrepentido de una oferta realizada por un
futbolista podrá cancelarla en cualquier momento.

Esta operación podrá ser realizada en cualquier momento en el que los
equipos estén abiertos.

\subsection{Renovación de futbolistas}
El sistema actualizará los sueldos que reciben los futbolistas al
término de cada temporada, según la ponderación de los atributos
cualitativos que tengan.\\

Esta operación se realizará una vez por temporada y afectará a todos
los futbolistas del sistema.
\subsection{Venta de entradas}
El usuario podrá estipular el precio de venta de las entradas en los
partidos que se disputen en su estadio siendo ésta la forma principal
de obtención de ingresos para la gestión económica del club.\\

Esta función será realizada tantas veces como desee el usuario.

\subsection{Planificación de alineaciones}
El usuario podrá escoger cuál será la alineación que presentará para
el próximo partido que vaya a disputar su equipo, pasando ésta a ser
la alineación por defecto hasta que el usuario decida volver a
cambiarla.\\

Esta operación se realizará a menudo.
\subsection{Planificación de tácticas}
El usuario podrá escoger entre diversas tácticas de juego para
afrontar los distintos partidos. Esta acción repercutirá directamente
en el estilo de juego y por tanto en la simulación resultado del
partido que dispute.\\

Esta operación se realizará tantas veces como el usuario desee.
\subsection{Planificación de entrenamientos}
El usuario podrá personalizar los atributos a entrenar de los
futbolistas integrantes de su equipo. De esta forma el club adquirirá
valor deportivo y económico.\\

Esta función se realizará a menudo.

\subsection{Asignar club}
Una vez que un usuario haya sido registrado en el servidor mediante
una cuenta de usuario el sistema le asignará un nuevo club cuando
todas las ligas hayan concluido y el administrador así desee.\\

Esta operación se realizará una vez por temporada y por cada nuevo
usuario registrado en el sistema.

\subsection{Simular jornadas}
El sistema debe emitir comentarios durante el transcurso del partido para que
cualquier usuario sea capaz de conocer la evolución del mismo.\\

Esta función se realizará a menudo, una vez por jornada.

\subsection{Comenzar jornadas}
El sistema debe simular la disputa de un partido en base a la
ponderación de los atributos cualitativos de los futbolistas
alineados, tácticas empleadas y variables aleatorias confrontadas con las del equipo
rival.\\

Esta función se realizará a menudo, una vez por jornada.

\subsection{Pasar a siguiente jornada}
El sistema debe ser capaz de gestionar el calendario creado de cada
competición. Para ello se le da la oportunidad al administrador de
avanzar en la línea temporal del calendario para hacer discurrir las
jornadas del calendario.

Además, en esta acción se actualizarán los entrenamientos y traspasos
aceptados por futbolistas entre usuarios, así como restar los gastos
de la plantilla y mantenimiento del estadio.

Esta función se realizará a menudo a lo largo de la competición.

\subsection{Promoción y descenso de clubes}
El sistema debe ser capaz de gestionar varias ligas organizadas de
forma jerárquica según el número de usuarios que estén registrados en
nuestro servidor. De esta forma conseguiremos que el usuario desee
mejorar la posición de su club, lo que repercutirá directamente en la
capacidad deportiva y económica del mismo.\\

Esta función se realizará una vez por temporada y una vez por jornada.

\subsection{Composición de calendarios}
El sistema deberá de crear un calendario de partidos por cada liga que
está albergada en el servidor. Los calendarios se crearán una vez por
temporada y será accesible a todos los usuarios.\\

Esta función se realizará una vez por creación de liga.
\subsection{Gestiones de lesiones y sanciones}
El sistema debe ser capaz de llevar a cabo una política realista de
imposición de sanciones a los futbolistas, así como simular lesiones
más o menos duraderas.\\

Esta función se realizará a menudo.

\section{Especificación de los Casos de Uso}
\subsection{Caso de uso crear cuenta de usuario}
\negrita{Descripción: } Crea una cuenta de usuario para poder 
participar y acceder a las acciones provistas por nuestra aplicación.\\
\negrita{Actores: } Usuario no registrado, usuario registrado,
sistema. \\
\negrita{Precondición: } Ninguna. \\
\negrita{Postcondición: } Se da de alta un nuevo usuario en el
sistema. \\
\negrita{Escenario principal:}
\begin{enumerate}
  \item Un usuario no registrado desea registrarse.
  \item El usuario introduce el nombre de usuario, la contraseña
    asociada al mismo, una cuenta de correo electrónico además de repetir la contraseña para evitar un
    error humano al ingresarla y también el idioma en el que desea ver
    por defecto la web y recibir los correos electrónicos.
  \item El sistema registra al usuario.
  \item El sistema avisa al usuario registrado mediante un correo
    electrónico y la propia interfaz web que ha completado su registro con éxito.
\end{enumerate}
\negrita{Escenario alternativo:}
\begin{enumerate}
  \item Si el usuario introduce un nombre de cuenta de usuario o e-mail
    existente o dos contraseñas no coincidentes se le avisará de los errores cometidos
    para que pueda realizar un registro exitoso.
\end{enumerate}

\subsection{Caso de uso acceder a la cuenta de usuario}
\negrita{Descripción: } El usuario provee un sistema para poder
identificar a los usuarios registrados en el mismo y dejarlos acceder
a la funciones principales de nuestro sistema.\\
\negrita{Actores: } Usuario registrado, usuario logueado. \\
\negrita{Precondición: } El usuario ha de estar registrado en nuestro sistema. \\
\negrita{Postcondición: } Se identifica al usuario en el sistema y se
le permite acceder a las funciones principales del mismo. \\
\negrita{Escenario principal:}
\begin{enumerate}
  \item Un usuario registrado desea iniciar sesión.
  \item El usuario introduce el nombre de usuario y la contraseña
    asociada al mismo, además de repetir la contraseña para evitar un
    error humano al ingresarla.
  \item El sistema registra al usuario.
  \item El sistema avisa al usuario registrado mediante un correo
    electrónico y la propia interfaz web que ha completado su registro con éxito.
\end{enumerate}
\negrita{Escenario alternativo:}
\begin{enumerate}
  \item Si el usuario introduce un nombre de cuenta de usuario o dos
    contraseñas no coincidentes se le avisará de los errores cometidos
    para que pueda realizar un registro exitoso.
\end{enumerate}

\subsection{Caso de uso abandonar la cuenta de usuario}
\negrita{Descripción: } El sistema provee un sistema para que un
usuario logueado pueda desconectarse de forma segura de nuestro sistema.\\
\negrita{Actores: } Usuario logueado. \\
\negrita{Precondición: } El usuario ha de haber iniciado sesión en
nuestro sistema. \\
\negrita{Postcondición: } Se cierra la sesión del usuario con nuestro sistema. \\
\negrita{Escenario principal:}
\begin{enumerate}
  \item El usuario logueado desea abandonar su sesión.
  \item Se le pregunta al usuario si está seguro de realizar la
    acción.
  \item El sistema avisa al usuario que acaba de finalizar la sesión a
    través de la interfaz web de nuestra aplicación.
\end{enumerate}
\negrita{Escenario alternativo:}
\begin{enumerate}
  \item Si el usuario decide que no está seguro de abandonar la sesión
    no se hace nada.
\end{enumerate}

\subsection{Caso de uso abrir equipos}
\negrita{Descripción: } El sistema provee un sistema para que el
usuario administrador pueda permitir cambios de jugadores, tácticas o
cualquier tipo de acción que modifique el estado de los equipos.\\
\negrita{Actores: } Usuario administrador. \\
\negrita{Precondición: } Ninguna. \\
\negrita{Postcondición: } Los equipos quedarán abiertos. \\
\negrita{Escenario principal:}
\begin{enumerate}
  \item El administrador cierra los equipos.
  \item El sistema avisa al administrador que acaba de abrir los
    equipos a través de la interfaz web de nuestra aplicación.
\end{enumerate}

\subsection{Caso de uso cerrar equipos}
\negrita{Descripción: } El sistema provee un sistema para que el
usuario administrador pueda evitar cambios de jugadores, tácticas o
cualquier tipo de acción que modifique el estado de los equipos.\\
\negrita{Actores: } Usuario administrador. \\
\negrita{Precondición: } Ninguna. \\
\negrita{Postcondición: } Los equipos quedarán cerrados. \\
\negrita{Escenario principal:}
\begin{enumerate}
  \item El administrador cierra los equipos.
  \item El sistema avisa al administrador que acaba de cerrar los
    equipos a través de la interfaz web de nuestra aplicación.
\end{enumerate}

\subsection{Caso de uso de notificación de registro por e-mail}
\negrita{Descripción: } El sistema notificará por correo electrónico
que un registro asociado al e-mail proporcionado se ha efectuado correctamente.\\
\negrita{Actores: } Sistema y usuario registrado \\
\negrita{Precondición: } El usuario ha de haber realizado un registro exitoso. \\
\negrita{Postcondición: } El sistema remite un correo electrónico en
el idioma que seleccionó en el registro nuestro usuario. \\
\negrita{Escenario principal:}
\begin{enumerate}
  \item El sistema detecta que un nuevo usuario ha sido creado.
  \item Remite un correo electrónico de bienvenida a la aplicación a
    la cuenta de correo asociada al usuario registrado.
\end{enumerate}

\subsection{Caso de uso de listar ofertas emitidas}
\negrita{Descripción: } El sistema ha de proporcionará una interfaz
para poder listar el historial de ofertas emitidas del club asociado.
\negrita{Actores: } Usuario logueado. \\
\negrita{Precondición: } El usuario ha de estar logueado en
nuestra aplicación.
\negrita{Postcondición: } Se muestra un listado del historial de
ofertas emitidas del club.
\negrita{Escenario principal:}
\begin{enumerate}
  \item Se muestra un listado del historial de ofertas emitidas del club.
\end{enumerate}

\subsection{Caso de uso de listar ofertas recibidas}
\negrita{Descripción: } El sistema ha de proporcionará una interfaz
para poder listar el historial de ofertas recibidas del club asociado.
\negrita{Actores: } Usuario logueado. \\
\negrita{Precondición: } El usuario ha de estar logueado en
nuestra aplicación.
\negrita{Postcondición: } Se muestra un listado del historial de
ofertas recibidas del club.
\negrita{Escenario principal:}
\begin{enumerate}
  \item Se muestra un listado del historial de ofertas recibidas del club.
\end{enumerate}

\subsection{Caso de uso de emitir oferta por futbolista}
\negrita{Descripción: } El sistema ha de proporcionará una interfaz
para poder realizar una oferta por un futbolista ajeno al club del que
el usuario es propietario.
\negrita{Actores: } Usuario logueado. \\
\negrita{Precondición: } El usuario ha de estar logueado en
nuestra aplicación, el comprador ha de tener la suficiencia económica
para acometer el fichaje, además los equipos han de estar abiertos. \\
\negrita{Postcondición: } El usuario emite una oferta por un
futbolista ajeno a su club y el propietario del mismo recibe la misma.
\negrita{Escenario principal:}
\begin{enumerate}
  \item Un usuario emite una oferta por un futbolista que posee
    otro usuario.
  \item El usuario propietario del futbolista recibe la oferta.
\end{enumerate}
\negrita{Escenario alternativo:}
\begin{enumerate}
  \item El usuario comprador no puede efectuar la oferta.
  \item El sistema emitirá un aviso a través de la interfaz web para
    informar que es imposible emitir la oferta.
\end{enumerate}

\subsection{Caso de uso de cancelar oferta emitida por futbolista}
\negrita{Descripción: } El sistema ha de proporcionará una interfaz
para poder cancelar una oferta emitida por un futbolista.
\negrita{Actores: } Usuario logueado. \\
\negrita{Precondición: } El usuario ha de estar logueado en
nuestra aplicación, la oferta ha de existir y los equipos han de estar
abiertos. \\
\negrita{Postcondición: } El usuario cancela una oferta por un
futbolista de un equipo ajeno al suyo.
\negrita{Escenario principal:}
\begin{enumerate}
  \item El usuario cancela la oferta previamente realizada.
\end{enumerate}

\subsection{Caso de uso de rechazar oferta por futbolista}
\negrita{Descripción: } El sistema ha de proporcionará una interfaz
para poder rechazar una oferta realizada por un futbolista que
provenga de otro usuario.
\negrita{Actores: } Usuario logueado. \\
\negrita{Precondición: } El usuario ha de estar logueado en
nuestra aplicación, la oferta ha de existir y los equipos han de estar
abiertos. \\
\negrita{Postcondición: } El usuario rechaza una oferta por un
futbolista de un equipo ajeno al suyo.
\negrita{Escenario principal:}
\begin{enumerate}
  \item El usuario rechaza la oferta previamente realizada.
\end{enumerate}

\subsection{Caso de uso de aceptar oferta por futbolista}
\negrita{Descripción: } El sistema ha de proporcionará una interfaz
para poder aceptar una oferta realizada por un futbolista que
provenga de otro usuario.
\negrita{Actores: } Usuario logueado. \\
\negrita{Precondición: } El usuario ha de estar logueado en
nuestra aplicación, la oferta ha de existir, los equipos han de estar
abiertos, la venta del jugador no debe suponer rebasar el límite
mínimo de jugadores en plantilla. \\
\negrita{Postcondición: } El usuario acepta la oferta, el montante
económico de la clausula del jugador es transferido a las arcas del
club y el jugador ya no es perteneciente a la plantilla.
\negrita{Escenario principal:}
\begin{enumerate}
  \item El usuario acepta la oferta.
  \item El futbolista es transferido a la plantilla del usuario emisor
    de la oferta.
  \item Se incrementa la caja del club en el montante económico de la
    clausula del jugador transferido.
\end{enumerate}
\negrita{Primer Escenario alternativo: }
\begin{enumerate}
  \item El usuario emisor no tiene recursos económicos suficientes
    para acometer el fichaje.
  \item El sistema informará al usuario que no puede aceptar la oferta
    y ésta será automáticamente rechazada.
\end{enumerate}
\negrita{Segundo Escenario alternativo: }
\begin{enumerate}
  \item El usuario receptor de la oferta está en el número mínimo de
    jugadores en plantilla.
  \item El sistema informará al usuario que no puede aceptar la oferta
    y la misma será automáticamente rechazada.
\end{enumerate}
\subsection{Caso de uso de renovación de futbolistas}
\negrita{Descripción: } El sistema renovará cada final de temporada a
los jugadores de nuestro equipo actualizando el sueldo que reciben y
su clausula de rescisión.\\
\negrita{Actores: } Sistema.\\
\negrita{Precondición: } Las temporadas han de haber concluido y los
equipos bloqueados. \\
\negrita{Postcondición: } Los jugadores serán renovados en función de
la nueva ponderación de sus atributos así como su clausula de rescisión.\\
\negrita{Escenario principal:}
\begin{enumerate}
  \item El sistema realizará una ponderación de los atributos de los
    futbolistas.
  \item Se asigna una clausula de rescisión en función de la
    ponderación.
  \item Se asigna un sueldo anual en función de la ponderación.
\end{enumerate}

\subsection{Caso de uso de fijar precio de venta de entradas}
\negrita{Descripción: } El usuario podrá modificar el precio de las
entradas de su equipo de fútbol. Este hecho repercutirá directamente
en la afluencia al estadio por parte de nuestros aficionados y a las
arcas del club.\\
\negrita{Actores: } Usuario logueado.\\
\negrita{Precondición: } Los equipos deben estar desbloqueados\\
\negrita{Postcondición: } El precio del ticket de entrada es
modificado.\\
\negrita{Escenario principal:} 
\begin{enumerate}
  \item El usuario desea cambiar el precio del ticket de entrada.
  \item El ticket de entrada es cambiado al precio estipulado por el usuario.
\end{enumerate}
\negrita{Escenario alternativo:}
\begin{enumerate}
  \item El usuario introduce un precio menor de cero.
  \item El sistema informará al usuario que no puede asignar un precio
    negativo a través de la interfaz web.
\end{enumerate}

\subsection{Caso de uso de planificación de alineaciones}
\negrita{Descripción: } El usuario podrá modificar quiénes serán los
futbolistas que jueguen el siguiente partido.\\
\negrita{Actores: } Usuario logueado\\
\negrita{Precondición: } Los equipos deben estar desbloqueados.\\
\negrita{Postcondición: } El usuario dejará formada la alineación que
usará en el siguiente partido.\\
\negrita{Escenario principal:}
\begin{enumerate}
  \item El usuario desea cambiar la posición entre dos jugadores.
  \item El usuario seleccionará dos jugadores.
  \item Efectuará el cambio de posición.
  \item El usuario repetirá este proceso todas las veces que necesite hasta dejar
    la alineación como desee.
\end{enumerate}
\negrita{Escenario alternativo:}
\begin{enumerate}
  \item El usuario selecciona un sólo jugador.
  \item Efectúa el cambio de posición.
  \item El sistema emitirá un aviso para informar que la operación que
    intenta hacer es imposible.
\end{enumerate}

\subsection{Caso de uso de planificación de tácticas}
\negrita{Descripción: } El usuario podrá seleccionar una táctica
distinta para afrontar el próximo partido.\\
\negrita{Actores: } Usuario logueado.\\
\negrita{Precondición: } Los equipos han de estar desbloqueados.\\
\negrita{Postcondición: } Se cambia la táctica por la seleccionada por
el usuario.\\
\negrita{Escenario principal:}
\begin{enumerate}
  \item El usuario desea cambiar la táctica usada en el partido.
  \item El usuario selecciona la táctica que desea usar y la fija como
    táctica por defecto para los siguientes partidos.
\end{enumerate}

\subsection{Caso de uso de planificación de entrenamientos}
\negrita{Descripción: } El usuario podrá mejorar una de las
habilidades que integran un jugador cada vez que lo mande a entrenar.\\
\negrita{Actores: } Usuario logueado.\\
\negrita{Precondición: } Los equipos han de estar desbloqueados y el
jugador no puede estar entrenando otra habilidad previa.\\
\negrita{Postcondición: } El jugador entrenará la habilidad
seleccionada por el usuario las jornadas que el sistema estipule.\\
\negrita{Escenario principal:}
\begin{enumerate}
  \item El usuario desea que un jugador de su plantilla entrene cierta habilidad.
  \item El sistema informará de cuántas jornadas serán necesarias para
    terminar el entrenamiento.
\end{enumerate}
\negrita{Escenario alternativo: }
\begin{enumerate}
  \item Si el usuario intenta poner a entrenar la habilidad de un
    jugador que haya llegado al máximo posible para el atributo el
    sistema le informará a través de la interfaz web de que no es
    necesario realizar dicha acción.
\end{enumerate}

\subsection{Caso de uso asignar club}
\negrita{Descripción: } El sistema proporciona una forma de asignar
nuevos club de fútbol a los nuevos usuarios registrados cuando el administrador lo desee entre
temporada y temporada.\\
\negrita{Actores: } Sistema, usuario registrado y administrador \\
\negrita{Precondición: } Las temporadas han de haber finalizado, debe
haber un mínimo de nuevos usuarios registrados para poder formar la
nueva liga y los equipos han de estar bloqueados. \\
\negrita{Postcondición: } Se les asigna a cada usuario un nuevo club
con varios jugadores y un monto inicial de dinero para poder comenzar
con las transacciones. \\
\negrita{Escenario principal:}
\begin{enumerate}
  \item El administrador desea crear una nueva liga.
  \item El sistema verifica que haya los suficientes nuevos usuarios
    sin clubes asignados para crearla.
  \item El sistema genera una plantilla de futbolistas nuevos a cada
    nuevo club creado.
  \item Se asigna cada club a cada usuario nuevo que formará la nueva
    liga.
\end{enumerate}

\subsection{Caso de uso de simular jornadas}
\negrita{Descripción: } El sistema simulará cada partido de cada
jornada de cada temporada según las alineaciones y tácticas de los
equipos que se enfrenten.\\
\negrita{Actores: } Administrador y Sistema. \\
\negrita{Precondición: } Los equipos deben estar bloqueados.\\
\negrita{Postcondición: } Se generará un conjunto de comentarios
ordenados por cada acción que tenga lugar en el partido.\\
\negrita{Escenario principal: }
\begin{enumerate}
  \item El administrador decidirá simular los partidos de la jornada
    en la que se encuentre el sistema.
  \item El sistema generará una ristra de comentarios por cada acción
    generada en el encuentro usando la ponderación de los equipos
    alineados y las tácticas utilizadas.
\end{enumerate}

\subsection{Caso de uso de comenzar jornadas}
\negrita{Descripción: } El sistema emulará la retransmisión de los
partidos englobados en las jornadas en tiempo real.\\
\negrita{Actores: } Administrador y Sistema. \\
\negrita{Precondición: } Los equipos deben estar bloqueados.\\
\negrita{Postcondición: } Se emitirá a través de la interfaz web el
conjunto de comentarios generados previamente en la simulación de los partidos.\\
\negrita{Escenario principal: }
\begin{enumerate}
  \item El administrador decide que se comiencen a emitir los
    comentarios de los partidos.
  \item El sistema empieza a mostrar los comentarios de cada partido
    de forma ordenada y actualizándose cada minuto.
\end{enumerate}

\subsection{Caso de uso de pasar a siguiente jornada}
\negrita{Descripción: } El sistema proveerá una forma de avanzar a lo
largo del calendario de partidos y actualizar traspasos, finanzas y entrenamientos.\\
\negrita{Actores: } Administrador y Sistema.\\
\negrita{Precondición: } Los equipos deben estar bloqueados\\
\negrita{Postcondición: }Se pasará de jornada en el calendario de las
ligas, se actualizarán los entrenamientos, traspasos y finanzas de los
clubes registrados en el sistema.\\
\negrita{Escenario principal: }
\begin{enumerate}
  \item El administrador decide pasar a la siguiente jornada.
  \item El sistema actualiza los resultados y las clasificaciones de
    cada liga y hace de la jornada siguiente la actual.
  \item Se actualizan las finanzas de los clubes.
  \item Se actualizan los traspasos aceptados.
  \item Se actualizan los entrenamientos de los jugadores.
\end{enumerate}
\negrita{Escenario alternativo: }
\begin{enumerate}
  \item Si la jornada es la última se dará por concluidas las
    temporadas que tenían lugar en cada liga.
\end{enumerate}


\subsection{Caso de uso de promoción y descenso de clubes}
\negrita{Descripción: } El sistema proveerá una forma de organizar las
ligas de forma jerárquica.\\
\negrita{Actores: } Administrador y Sistema.\\
\negrita{Precondición: } Los equipos deben estar bloqueados\\
\negrita{Postcondición: } Se ascenderán y descenderán los clubes que
estén en posiciones de descenso y ascenso según la organización
jerárquica de las ligas.\\
\negrita{Escenario principal: }
\begin{enumerate}
  \item El sistema reorganizará las ligas para reflejar los cambios en
    las ligas debida a las posiciones que obtengan en la tabla de
    clasificación los equipos de nuestro sistema.
\end{enumerate}

\subsection{Caso de uso de composición de calendarios}
\negrita{Descripción: } El sistema proveerá una forma de componer los
calendarios de las ligas que se estén jugando en nuestra aplicación.\\
\negrita{Actores: } Administrador y Sistema.\\
\negrita{Precondición: } Los equipos deben estar bloqueados y las
temporadas finalizadas.\\
\negrita{Postcondición: } Se generará un calendario de jornadas
siguiendo un algoritmo round-robin de creación de calendarios por cada
liga albergada en nuestro sistema.\\
\negrita{Escenario principal: }
\begin{enumerate}
  \item El sistema creará un calendario por liga de las nuevas
    temporadas que van a jugarse en nuestro sistema.
\end{enumerate}

\section{Diagramas de Casos de Uso}

\figura{casos_uso/funciones_principales.png}{scale=0.35}{Diagrama de
  casos de uso de las funciones principales}{casos-uso-funciones-generales}{H}

\negrita{Descripción:} 
\begin{enumerate}
  \item{Gestión de usuarios}
    \begin{itemize}
      \item Configurar el sitio
      \item Crear una cuenta de usuario
      \item Acceder a la cuenta de usuario
      \item Abandonar la cuenta de usuario
      \item Notificación de registro por e-mail
      \figura{casos_uso/gestion_usuarios.png}{scale=0.35}{Diagrama de
        casos de uso para la gestión de usuarios}{casos-uso-gestion-usuarios}{H}
    \end{itemize}
  \item{Gestión económica}
    \begin{itemize}
      \item Listar ofertas emitidas
      \item Listar ofertas recibidas
      \item Emitir oferta por futbolista
      \item Aceptar oferta por futbolista
      \item Rechazar oferta por futbolista
      \item Cancelar oferta emitida por futbolista
      \item Renovación de futbolistas
      \item Fijar precio de venta de entradas
        \figura{casos_uso/gestion_economica.png}{scale=0.35}{Diagrama
          de casos de uso para la gestión económica}{casos-uso-gestion-economica}{H}
    \end{itemize}
  \item{Gestión deportiva}
    \begin{itemize}
      \item Planificación de alineaciones
      \item Planificación de tácticas
      \item Planificación de entrenamientos
      \figura{casos_uso/gestion_deportiva.png}{scale=0.35}{Diagrama de
      casos de uso para la gestión deportiva}{casos-uso-gestion-deportiva}{H}
    \end{itemize}
  \item{Gestión de campeonatos}
    \begin{itemize}
      \item Asignar club
      \item Simular jornadas
      \item Comenzar jornadas
      \item Pasar a siguiente jornada
      \item Promoción y descenso de clubes
      \item Composición de calendarios
      \figura{casos_uso/gestion_campeonatos.png}{scale=0.35}{Diagrama
        de casos de uso para la gestión de campeonatos}{casos-uso-gestion-campeonatos}{H}
     \end{itemize}
\end{enumerate}

\section{Diagrama de clases conceptuales}
En la figura \ref{diagrama-clases-uml} podemos observar el diagrama de clases de
nuestra aplicación según la recolección de requisitos que hemos
obtenido.

\figura{uml.png}{scale=0.30}{Diagrama de Clases
  UML}{diagrama-clases-uml}{H}

\section{Diagramas de secuencia de sistemas}

\subsection{Diagrama de secuencia de crear cuenta de usuario}

\figura{diagramas_secuencia/crear_cuenta_usuario.png}{scale=0.35}{Diagrama de
  secuencia de crear cuenta de usuario}{diagrama-secuencia-crear-cuenta-usuario}{H}

\subsection{Diagrama de secuencia de acceder a cuenta de usuario}

\figura{diagramas_secuencia/acceder_cuenta_usuario.png}{scale=0.35}{Diagrama de
  secuencia de acceder a cuenta de
  usuario}{diagrama-secuencia-acceder-cuenta-usuario}{H}

\subsection{Diagrama de secuencia de abandonar cuenta de usuario}

\figura{diagramas_secuencia/abandona_cuenta_usuario.png}{scale=0.35}{Diagrama de
  secuencia de abandonar cuenta de
  usuario}{diagrama-secuencia-abandonar-cuenta-usuario}{H}

\subsection{Diagrama de secuencia de abrir equipos}

\figura{diagramas_secuencia/abrir_equipos.png}{scale=0.35}{Diagrama de
  secuencia de abrir equipos}{diagrama-secuencia-abrir-equipos}{H}

\subsection{Diagrama de secuencia de cerrar equipos}

Este diagrama es igual al \ref{diagrama-secuencia-abrir-equipos} pero
con la ruta \cursiva{/leagues/close\_teams} y la llamada al modelo con
el método \cursiva{close}.


\subsection{Diagrama de secuencia de notificación de registro por
  e-mail}

\figura{diagramas_secuencia/crear_cuenta_usuario.png}{scale=0.35}{Diagrama de
  secuencia de notificación de registro por e-mail}{diagrama-secuencia-notificacion-registro}{H}


\subsection{Diagrama de secuencia de listar ofertas emitidas}

\figura{diagramas_secuencia/listar_ofertas_emitidas.png}{scale=0.35}{Diagrama de
  secuencia de listar ofertas emitidas}{diagrama-secuencia-listar-ofertas-emitidas}{H}

\subsection{Diagrama de secuencia de listar ofertas recibidas}

Este diagrama es igual al
\ref{diagrama-secuencia-listar-ofertas-emitidas} pero con la ruta
\cursiva{/clubs/*/received\_offers} y la llamada al modelo club con el
método \cursiva{offers\_as\_seller}.

\subsection{Diagrama de secuencia de emitir oferta por futbolista}

\figura{diagramas_secuencia/emitir_oferta_futbolista.png}{scale=0.35}{Diagrama de
  secuencia de emitir oferta por
  futbolista}{diagrama-secuencia-emitir-oferta-futbolista}{H}

\subsection{Diagrama de secuencia de cancelar oferta emitida por
  futbolista}

\figura{diagramas_secuencia/cancelar_oferta_emitida_futbolista.png}{scale=0.35}{Diagrama de
  secuencia de cancelar oferta emitida por
  futbolista}{diagrama-secuencia-cancelar-oferta-emitida-futbolista}{H}

\subsection{Diagrama de secuencia de rechazar oferta recibida por
  futbolista}

\figura{diagramas_secuencia/rechazar_oferta_recibida_futbolista.png}{scale=0.35}{Diagrama de
  secuencia de rechazar oferta recibida por
  futbolista}{diagrama-secuencia-rechazar-oferta-recibida-futbolista}{H}

\subsection{Diagrama de secuencia de aceptar oferta recibida por
  futbolista}

\figura{diagramas_secuencia/aceptar_oferta_recibida_futbolista.png}{scale=0.35}{Diagrama de
  secuencia de aceptar oferta recibida por
  futbolista}{diagrama-secuencia-aceptar-oferta-recibida-futbolista}{H}

\subsection{Diagrama de secuencia de renovación de futbolistas}

\figura{diagramas_secuencia/renovacion_futbolistas.png}{scale=0.35}{Diagrama de
  secuencia de renovación de
  futbolistas}{diagrama-secuencia-renovacion-futbolistas}{H}

El proceso de renovación está integrado en start.

\subsection{Diagrama de secuencia de fijar precio de venta de
  entradas}

\figura{diagramas_secuencia/fijar_precio_venta_entradas.png}{scale=0.35}{Diagrama de
  secuencia de fijar precio de venta de
  entradas}{diagrama-secuencia-fijar-precio-venta-entradas}{H}

\subsection{Diagrama de secuencia de planificación de alineaciones}

\figura{diagramas_secuencia/planificacion_alineaciones.png}{scale=0.35}{Diagrama de
  secuencia de emitir planificación de
  alineaciones}{diagrama-secuencia-planificacion-alineaciones}{H}

\subsection{Diagrama de secuencia de planificación de tácticas}

\figura{diagramas_secuencia/planificacion_tacticas.png}{scale=0.35}{Diagrama de
  secuencia de planificación de tácticas
}{diagrama-secuencia-planificacion-tacticas}{H}

\subsection{Diagrama de secuencia de planificación de entrenamientos}

\figura{diagramas_secuencia/planificacion_entrenamientos.png}{scale=0.35}{Diagrama de
  secuencia de planificación de entrenamientos
}{diagrama-secuencia-planificacion-entrenamientos}{H}

\subsection{Diagrama de secuencia de asignar club}

\figura{diagramas_secuencia/asignar_club.png}{scale=0.35}{Diagrama de
  secuencia de secuencia de asignar club
}{diagrama-secuencia-asignar-club}{H}

\clearpage

\subsection{Diagrama de secuencia de simular jornadas}

\figura{diagramas_secuencia/simular_jornadas.png}{scale=0.32, angle=270}{Diagrama de
  secuencia de simular
  jornadas}{diagrama-secuencia-simular-jornadas}{H}

\subsection{Diagrama de secuencia de comenzar jornadas}

\figura{diagramas_secuencia/comenzar_jornadas.png}{scale=0.43, angle=270}{Diagrama de
  secuencia de comenzar
  jornadas}{diagrama-secuencia-comenzar-jornadas}{H}


\subsection{Diagrama de secuencia de pasar a siguiente jornada}

\figura{diagramas_secuencia/pasar_siguiente_jornada.png}{scale=0.27, angle=270}{Diagrama
 de secuencia de pasar a siguiente
  jornada}{diagrama-secuencia-pasar-siguiente-jornada}{H}

\subsection{Diagrama de secuencia de promoción y descenso de clubes}

\figura{diagramas_secuencia/promocion_descenso_clubes.png}{scale=0.35}{Diagrama
 de secuencia de promocion y descenso de
 clubes}{diagrama-secuencia-promocion-descenso-clubes}{H}

\clearpage

\subsection{Diagrama de secuencia de composición de calendarios}

\figura{diagramas_secuencia/crear_calendario.png}{scale=0.32, angle=270}{Diagrama
 de secuencia de composición de calendarios}{diagrama-secuencia-composicion-calendarios}{H}

\chapter{Diseño}
\label{cap:diseño}
% -*-cap5.tex-*- Este fichero es parte de la plantilla LaTeX para la realización
% de Proyectos Final de Carrera, protejido bajo los términos de la licencia
% GFDL.  Para más información, la licencia completa viene incluida en el fichero
% fdl-1.3.tex Copyright (C) 2009 Pablo Recio Quijano

Para la implementación de toda la funcionalidad descrita en el capítulo anterior
seguiremos las convenciones que el framework \cursiva{Ruby on Rails} nos
proporciona para una mayor velocidad y corrección en el desarrollo.

\section{Arquitectura}

\label{pattern:mvc}
Debido a \cursiva{Ruby on Rails} es un framework que hace un uso intensivo del
patrón \cursiva{Modelo-Vista-Controlador} (MVC) \cite{agilerails}.

El patrón \cursiva{Modelo-Vista-Controlador} divide el software en tres capas
bien diferenciadas que se adaptan muy bien a la tecnología
Cliente-Servidor. Dichas capas son:

\begin{itemize}
\item \negrita{Modelo}: Esta capa es la relativa a la representación de la
  información con la que el proyecto interactuará. Está compuesto por las clases
  necesarias para el acceso, modificación, búsqueda y demás operaciones con los
  datos.
\item \negrita{Vista}: En esta capa agruparemos todas las clases necesarias para
  la representación de la interfaz con la que nuestros usuarios interactuarán.
\item \negrita{Controlador}: Será la capa que nos permita gestionar las
  peticiones que lleguen a nuestro servidor.
\end{itemize}

\subsection{Capa Modelo}

Para la capa de modelo y la consecuente persistencia de datos vamos a usar el
\cursiva{ORM} (mapeo objeto-relacional) que nos proporciona por defecto
\cursiva{Ruby On Rails}, la clase \cursiva{ActiveRecord} que implementa el
patrón \cursiva{Active Record} pudiéndonos abstraer del \cursiva{SGBD}
relacional que usemos y las especificaciones únicas del lenguaje \cursiva{SQL}
que implemente el mismo.

\subsection{Capa Vista}

Para esta capa vamos a utilizar la clase \cursiva{ActionView}. Dicha clase es
una forma poderosa, sencilla y elegante, de embeber código \cursiva{Ruby} dentro
de cualquier tipo de fichero de la capa de presentación de nuestra página web
(\cursiva{JavaScript}, \cursiva{CSS}).

Además, haremos uso de los frameworks \cursiva{CoffeScript}
\cite{lang:coffescript} y \cursiva{SASS} \cite{lang:sass} para el desarrollo de
los ficheros \cursiva{JavaScript} y \cursiva{CSS} respectivamente.

\subsection{Capa Controlador}
En esta capa haremos uso de la clase \cursiva{ApplicationController}. Esta clase
nos ayudará a modelar de forma sencilla y fácil el enrutador de nuestra
aplicación y las respuestas que se hagan a través del cliente en la petición
web.

Además, nos ayudará a establecer un flujo sencillo entre nuestras capas Vista y
Modelo consiguiendo un menor acoplamiento y una mayor cohesión de nuestro
sistema.

\subsection{Active Record}
Se basa en el patrón de mismo nombre el cual es un patrón de arquitectura que
podemos encontrar en software que almacene información en bases de datos
relacionales. Conforme a este patrón, un objeto que lo use habrá de integrar
funciones para insertar, actualizar, eliminar y otras que correspondan en mayor
o menor medida a las operaciones con columnas de la tabla de una base de datos.

Este patrón crea una interfaz que envuelve una tabla de una base de datos o una
vista en una clase. En base a ello, un objeto de la clase corresponde a una fila
en la tabla. La clase que envuelve a la tabla o vista implementa métodos o
propiedades para acceder a cada columna.

Este patrón es normalmente usado para dar persistencia a los datos en un
proyecto software, haciendo de \cursiva{ORM} para la aplicación. Normalmente,
las relaciones de clave foránea entre clases son implementadas como una
propiedad del objeto \cite{pattern:activerecord}.

Las funcionalidades que implementa la clase \cursiva{Active Record} en
\cursiva{Rails} son:

\begin{itemize}
\item \negrita{Creación}: Permite la creación de nuevas líneas en la base de
  datos.
\item \negrita{Condiciones}: Permite consultar datos en la base de datos usando
  para ello cadenas de texto o \cursiva{hashes} que representen la parte
  \cursiva{WHERE} en una sentencia SQL.
\item \negrita{Sobrescribir métodos de accesibilidad por defecto}: Se pueden
  sobrescribir los métodos de accesibilidad para modificar el comportamiento en
  las operaciones de escritura, creación, lectura, y demás operaciones de
  lectura/escritura.
\item \negrita{Métodos de consulta como atributos}: Se pueden crear nuevos
  métodos que se comporten como atributos para conocer si un atributo dentro de
  la tabla está inicializado.
\item \negrita{Posibilidad de acceder a los atributos antes de que hayan pasado
    el casting de tipo}: Es una funcionalidad orientada sobre todo en las
  situaciones de validación.
\item \negrita{Buscadores dinámicos en base al atributo}: Métodos estándar
  creados en base al nombre de los atributos correspondientes al nombre de la
  columna dentro de la tabla.
\item \negrita{Serializar vectores, hashes y otros objetos en columnas de
    texto}: Es posible guardar este tipo de objetos en formato texto usando el
  lenguaje \cursiva{YAML}.
\item \negrita{Herencia simple de tabla}: Es posible el uso de la herencia
  simple entre tablas.
\item \negrita{Conexión a múltiples bases de datos en diferentes modelos}: Es
  posible establecer diferentes conexiones a diferentes \cursiva{SGBD} para
  diferentes clases.
\end{itemize}

\subsection{Action View}
\cursiva{Action View} hace uso del sistema de plantillas \cursiva{eRuby} para
embeber código \cursiva{Ruby} dentro de un documento de texto. Es ampliamente
utilizado para dentro de documentos \cursiva{HTML} y \cursiva{JavaScript}.

\subsection{Action Controller}
Es el núcleo de una petición web en \cursiva{Ruby on Rails}. Está formado por
una o más acciones que son ejecutadas cuando se produzca una petición, justo
después se puede redireccionar a otra acción o renderizar una plantilla. Una
acción se define como un método público dentro del controlador, el cuál es
automáticamente accesible al servidor web a través del enrutador de
\cursiva{Rails}.

Por defecto, sólo la clase \cursiva{ApplicationController} dentro de una
aplicación \cursiva{Rails} hereda de la clase
\cursiva{ActionController:Base}. Los demás controladores heredarán de la clase
\cursiva{ApplicationController}. Este tipo de estructura nos proporciona una
clase para configurar varias cosas que sean comunes a cualquier petición web.

Esta clase nos proporciona además las siguientes funcionalidades:

\begin{itemize}
\item \negrita{Peticiones}: El objeto de petición es totalmente accesible y se
  usa principalmente para conocer la cabecera de la petición \cursiva{HTTP}.
\item \negrita{Parámetros}: Todos los parámetros que lleva la petición, ya sean
  desde \cursiva{GET} o \cursiva{POST} o de la \cursiva{URL}, son accesibles a
  través del \cursiva{método} \negrita{params} que devolverá un \cursiva{hash}
  uni o multidimensional con todos los parámetros asociados a la petición.
\item \negrita{Sesiones}: Las sesiones permiten almacenar objetos entre petición
  y petición. Son bastante útiles para objetos que no estén aún preparados para
  pasar a ser almacenados en la base de datos. Para acceder a las sesiones
  bastará con llamar al método \negrita{session} y este devolverá un
  \cursiva{hash} uni o multidimensional con las sesiones almacenadas.
\item \negrita{Respuestas}: Toda acción lleva asociada una respuesta, la cual
  alberga las cabeceras y el documento para ser mandado al navegador del
  usuario. El objeto respuesta es generado automáticamente a través del uso de
  \cursiva{renders} y redirecciones, con lo que no necesita la interacción del
  usuario.
\item \negrita{Renderizaciones}: La clase \cursiva{ActionController} manda
  contenido al usuario a través del uso de métodos de renderización. El más
  común es el uso de una plantilla para la renderización del contenido, a través
  de plantillas \cursiva{ERB}.
\item \negrita{Redirecciones}: Son usadas para moverse de una acción a otra.
\item \negrita{Redirecciones y renderizaciones múltiples}: Es posible renderizar
  y redireccionar varias veces en la misma petición.
\end{itemize}

\subsection{Diagrama de componentes y estructura}

Podemos ver el diagrama de componentes en la figura \ref{diag:componentes}.

\figura{mvc-rails.png}{scale=0.50}{Diagrama de componentes}{diag:componentes}{H}

El framework \cursiva{Ruby on Rails} hace especial hincapié en tres conceptos
muy básicos que caracterizan toda su estructura, que son:

\begin{itemize}
\item \negrita{Convention over Configuration (COC)}: Convención frente a
  configuración.
\item \negrita{Don't Repeat Yourself (DRY)}: No te repitas
\item \negrita{Keep It Simple (KIS)}: Mantenlo simple.
\end{itemize}

Gracias a ello seguiremos la estructura de directorios en árbol para nuestro
código fuente que genera \cursiva{Rails} para una aplicación vacía.

\begin{itemize}
\item \texttt{/pfc\_sfo/} raíz del proyecto \cursiva{Rails}.
\item \texttt{/pfc\_sfo/app/} ficheros fuente principales.
\item \texttt{/pfc\_sfo/app/controllers/} ficheros fuente para los
  controladores.
\item \texttt{/pfc\_sfo/app/helpers/} ficheros fuente para los métodos de ayuda
  para refactorización de códigos en las vistas.
\item \texttt{/pfc\_sfo/app/mailers/} ficheros fuente para el código fuente de
  los correos que se envían desde el servidor.
\item \texttt{/pfc\_sfo/app/models/} ficheros fuente correspondientes a los
  modelos de datos.
\item \texttt{/pfc\_sfo/app/views/} ficheros fuente para las vistas de la
  aplicación, incluyendo como parte de las mismas las de los correos.
\item \texttt{/pfc\_sfo/app/assets/} ficheros fuente que han de ejecutarse en el
  navegador, archivos \cursiva{CSS}, \cursiva{JavasScript}, imágenes, etc...
\item \texttt{/pfc\_sfo/config/}, ficheros fuente de configuración de la
  aplicación web.
\item \texttt{/pfc\_sfo/config/environments/} ficheros fuente de configuración
  de los distintos entornos, desarrollo, test y producción.
\item \texttt{/pfc\_sfo/config/initializers/} ficheros de inicialización del
  servidor.
\item \texttt{/pfc\_sfo/config/locales/} ficheros \cursiva{YAML} para los textos
  de traducción en diferentes idiomas.
\item \texttt{/pfc\_sfo/db/} ficheros autogenerados dónde se albergan la
  configuración de la Base de Datos.
\item \texttt{/pfc\_sfo/db/migrate/} ficheros para la manipulación de las tablas
  de la base de datos.
\item \texttt{/pfc\_sfo/lib/} ficheros para la extensión de la aplicación con
  otros subsistemas software.
\item \texttt{/pfc\_sfo/log/} ficheros log del servidor.
\item \texttt{/pfc\_sfo/public/} ficheros de acceso público a través del
  enrutador.
\item \texttt{/pfc\_sfo/script/} scripts para la generación automática de código
  y la automatización de procesos.
\item \texttt{/pfc\_sfo/test/} ficheros para almacenar test automáticos.
\end{itemize}

\subsection{Entorno de ejecución}
En la figura \ref{fig:entornoejecucion} podemos ver el diagrama de despliegue de
la aplicación puesta en marcha en cualquiera de los entornos, ya sea este de
desarrollo, test o producción.

\figura{diagrama_despliegue.png}{scale=0.53}{Diagrama de
  estructura}{fig:entornoejecucion}{H}

El sistema (servidor) interactuará con uno o más clientes web (navegadores). El
servidor, a su vez estará formado por un servidor de aplicaciones web, en
nuestro caso \cursiva{Webrick}, y el sistema de gestión de base de datos será
\cursiva{SQLite3}.

Gracias a esta sencilla configuración no tendremos que configurar ninguna cuenta
ni servicio en nuestro servidor web.

\clearpage
\section{Diagramas de clases conceptuales}
En la figura \ref{diag:clases_modelo} podemos observar el diagrama de clases de
la capa modelo de nuestra aplicación según la recolección de requisitos que
hemos obtenido en el Capítulo \ref{cap:analisis} de esta memoria. Dichos
diagramas serán realizados en notación \cursiva{UML}
\cite{uml:distilled_standard}.

\figura{models_complete.png}{scale=0.25}{Diagrama de clases UML de la capa
  modelo}{diag:clases_modelo}{H}

\clearpage

En la figura \ref{diag:clases_controlador} podemos observar el diagramas de
clases de la capa controlador de nuestra aplicación según la recolección de
requisitos que hemos obtenido en el Capítulo \ref{cap:analisis}.

\figura{controllers_complete.png}{scale=0.22,angle=270}{Diagrama de Clases UML
  de la capa controlador}{diag:clases_controlador}{H}

\section{Diagramas de secuencia de sistemas}

En las siguientes secciones podremos observar los diagramas de secuencias del
sistema software que hemos desarrollado.

\subsection*{Diagrama de secuencia de crear cuenta de usuario}

En la figura \ref{diagrama-secuencia-crear-cuenta-usuario} podemos observar como
se construye un nuevo usuario dentro del sistema con los datos proporcionados
por el usuario web.

\figura{diagramas_secuencia/crear_cuenta_usuario.png}{scale=0.42}{Diagrama de
  secuencia de crear cuenta de
  usuario}{diagrama-secuencia-crear-cuenta-usuario}{H} \newpage

\subsection*{Diagrama de secuencia de acceder a cuenta de usuario}

En la figura \ref{diagrama-secuencia-acceder-cuenta-usuario} podemos observar
como se consigue acreditar las credenciales (usuario y contraseña) para iniciar
sesión.

\figura{diagramas_secuencia/acceder_cuenta_usuario.png}{scale=0.49}{Diagrama de
  secuencia de acceder a cuenta de
  usuario}{diagrama-secuencia-acceder-cuenta-usuario}{H} \newpage

\subsection*{Diagrama de secuencia de abandonar cuenta de usuario}

En la figura \ref{diagrama-secuencia-abandonar-cuenta-usuario} podemos observar
cómo se destruye la sesión iniciada por un usuario.

\figura{diagramas_secuencia/abandona_cuenta_usuario.png}{scale=0.59}{Diagrama de
  secuencia de abandonar cuenta de
  usuario}{diagrama-secuencia-abandonar-cuenta-usuario}{H} \newpage

\subsection*{Diagrama de secuencia de abrir equipos}

En la figura \ref{diagrama-secuencia-abrir-equipos} cómo se establece el estado
de las temporadas para permitir operaciones por parte de los usuarios en los
equipos que poseen.

\figura{diagramas_secuencia/abrir_equipos.png}{scale=0.78}{Diagrama de secuencia
  de abrir equipos}{diagrama-secuencia-abrir-equipos}{H}

\subsection*{Diagrama de secuencia de cerrar equipos}
En la figura \ref{diagrama-secuencia-abrir-equipos} podemos observar cómo se
establece el estado de las temporadas para impedir que los usuarios puedan
realizar operaciones con sus equipos.

Este diagrama es igual al \ref{diagrama-secuencia-abrir-equipos} pero con la
ruta \cursiva{/leagues/close\_teams} y la llamada al modelo con el método
\cursiva{close}.


\subsection*{Diagrama de secuencia de notificación de registro por e-mail}
En la figura \ref{diagrama-secuencia-notificacion-registro} se observa cómo se
envía un correo al usuario una vez se ha registrado en la aplicación web.

\figura{diagramas_secuencia/crear_cuenta_usuario.png}{scale=0.43}{Diagrama de
  secuencia de notificación de registro por
  e-mail}{diagrama-secuencia-notificacion-registro}{H}


\subsection*{Diagrama de secuencia de listar ofertas emitidas}
En la figura \ref{diagrama-secuencia-listar-ofertas-emitidas} observamos cómo se
establece el filtro de búsqueda para mostrar el listado de ofertas emitidas.

\figura{diagramas_secuencia/listar_ofertas_emitidas.png}{scale=0.52}{Diagrama de
  secuencia de listar ofertas
  emitidas}{diagrama-secuencia-listar-ofertas-emitidas}{H}

\subsection*{Diagrama de secuencia de listar ofertas recibidas}
En la figura \ref{diagrama-secuencia-listar-ofertas-emitidas} observamos cómo se
establece el filtro de búsqueda para mostrar el listado de ofertas recibidas.

Este diagrama es igual al \ref{diagrama-secuencia-listar-ofertas-emitidas} pero
con la ruta \cursiva{/clubs/*/received\_offers} y la llamada al modelo club con
el método \cursiva{offers\_as\_seller}.

\subsection*{Diagrama de secuencia de emitir oferta por futbolista}
En la figura \ref{diagrama-secuencia-emitir-oferta-futbolista} se observa cómo
se crea una nueva instancia de oferta y se relaciona con los usuarios y el
futbolista involucrado en la misma.

\figura{diagramas_secuencia/emitir_oferta_futbolista.png}{scale=0.45}{Diagrama
  de secuencia de emitir oferta por
  futbolista}{diagrama-secuencia-emitir-oferta-futbolista}{H}

\subsection*{Diagrama de secuencia de cancelar oferta emitida por futbolista}
En la figura \ref{diagrama-secuencia-cancelar-oferta-emitida-futbolista} se
observa cómo se recupera la instancia de la oferta emitida y se modifica su
estado para cancelarla.

\figura{diagramas_secuencia/cancelar_oferta_emitida_futbolista.png}{scale=0.68}{Diagrama
  de secuencia de cancelar oferta emitida por
  futbolista}{diagrama-secuencia-cancelar-oferta-emitida-futbolista}{H}

\subsection*{Diagrama de secuencia de rechazar oferta recibida por futbolista}
En la figura \ref{diagrama-secuencia-rechazar-oferta-recibida-futbolista} se
observa cómo se recupera la instancia de la oferta recibida y se modifica su
estado para rechazar la misma.

\figura{diagramas_secuencia/rechazar_oferta_recibida_futbolista.png}{scale=0.68}{Diagrama
  de secuencia de rechazar oferta recibida por
  futbolista}{diagrama-secuencia-rechazar-oferta-recibida-futbolista}{H}

\newpage

\subsection*{Diagrama de secuencia de aceptar oferta recibida por futbolista}
En la figura \ref{diagrama-secuencia-aceptar-oferta-recibida-futbolista} se
observa cómo se recupera la instancia de la oferta recibida y se actualizan los
modelos de datos involucrados (futbolista, club y usuarios) para llevar a cabo
la transferencia del futbolista entre los clubes.

\figura{diagramas_secuencia/aceptar_oferta_recibida_futbolista.png}{scale=0.65}{Diagrama
  de secuencia de aceptar oferta recibida por
  futbolista}{diagrama-secuencia-aceptar-oferta-recibida-futbolista}{H} \newpage

\subsection*{Diagrama de secuencia de renovación de futbolistas}
En la figura \ref{diagrama-secuencia-renovacion-futbolistas} se describe cómo se
renuevan automáticamente los futbolistas para las siguientes temporadas
actualizando sus atributos.

\figura{diagramas_secuencia/renovacion_futbolistas.png}{scale=0.63}{Diagrama de
  secuencia de renovación de
  futbolistas}{diagrama-secuencia-renovacion-futbolistas}{H}

El proceso de renovación está integrado en start.  \newpage

\subsection*{Diagrama de secuencia de fijar precio de venta de entradas}
En la figura \ref{diagrama-secuencia-fijar-precio-venta-entradas} se describe
cómo se modifica el atributo del precio de venta de entradas.

\figura{diagramas_secuencia/fijar_precio_venta_entradas.png}{scale=0.54}{Diagrama
  de secuencia de fijar precio de venta de
  entradas}{diagrama-secuencia-fijar-precio-venta-entradas}{H}

\subsection*{Diagrama de secuencia de planificación de alineaciones}
En la figura \ref{diagrama-secuencia-planificacion-alineaciones} se describe
cómo se cambian las alineaciones para el próximo partido a través de la
modificación del atributo de posición.

\figura{diagramas_secuencia/planificacion_alineaciones.png}{scale=0.52}{Diagrama
  de secuencia de emitir planificación de
  alineaciones}{diagrama-secuencia-planificacion-alineaciones}{H}

\subsection*{Diagrama de secuencia de planificación de tácticas}
En la figura \ref{diagrama-secuencia-planificacion-tacticas} se muestra cómo se
cambia la táctica usada para el próximo partido a través de la modificación del
atributo de táctica del club.

\figura{diagramas_secuencia/planificacion_tacticas.png}{scale=0.70}{Diagrama de
  secuencia de planificación de tácticas
}{diagrama-secuencia-planificacion-tacticas}{H}

\subsection*{Diagrama de secuencia de planificación de entrenamientos}
En la figura \ref{diagrama-secuencia-planificacion-entrenamientos} se muestra
cómo se crean nuevas instancias para la planificación de los entrenamientos de
los futbolistas asociados a un club.

\figura{diagramas_secuencia/planificacion_entrenamientos.png}{scale=0.45}{Diagrama
  de secuencia de planificación de entrenamientos
}{diagrama-secuencia-planificacion-entrenamientos}{H}

\subsection*{Diagrama de secuencia de asignar club}
En la figura \ref{diagrama-secuencia-asignar-club} se observa cómo se crea una
nueva instancia de un club para todo usuario que no tenga club asignado.

\figura{diagramas_secuencia/asignar_club.png}{scale=0.45}{Diagrama de secuencia
  de secuencia de asignar club }{diagrama-secuencia-asignar-club}{H}

\subsection*{Diagrama de secuencia de simular jornadas}
En la figura \ref{diagrama-secuencia-simular-jornadas} se observa cómo se crean
nuevas instancias de detalles de partidos por cada uno de los partidos no
jugados y pertenecientes a la jornada actual del sistema.

\figura{diagramas_secuencia/simular_jornadas.png}{scale=0.35,
  angle=270}{Diagrama de secuencia de simular
  jornadas}{diagrama-secuencia-simular-jornadas}{H}

\subsection*{Diagrama de secuencia de comenzar jornadas}
En la figura \ref{diagrama-secuencia-comenzar-jornadas} se observa cómo se
actualiza el atributo correspondiente de todos los partidos actuales para
estipular que el partido ha comenzado a la fecha actual del sistema.  \newpage

\figura{diagramas_secuencia/comenzar_jornadas.png}{scale=0.49,
  angle=270}{Diagrama de secuencia de comenzar
  jornadas}{diagrama-secuencia-comenzar-jornadas}{H}


\subsection*{Diagrama de secuencia de pasar a siguiente jornada}
En la figura \ref{diagrama-secuencia-pasar-siguiente-jornada} se detalla cómo se
actualizan los modelos de datos de temporada, jornada, jugador, entrenamiento,
finanzas y club para avanzar en la competición.  \newpage
 
\figura{diagramas_secuencia/pasar_siguiente_jornada.png}{scale=0.29,
  angle=270}{Diagrama de secuencia de pasar a siguiente
  jornada}{diagrama-secuencia-pasar-siguiente-jornada}{H}

\subsection*{Diagrama de secuencia de promoción y descenso de clubes}
En la figura \ref{diagrama-secuencia-promocion-descenso-clubes} se observa cómo
se crean nuevas instancias de nuevas temporadas y se modifican los estados de
las temporadas actuales para simular el avance en la competición y el progreso y
descenso jerárquico en las divisiones del juego.

\figura{diagramas_secuencia/promocion_descenso_clubes.png}{scale=0.38}{Diagrama
  de secuencia de promocion y descenso de
  clubes}{diagrama-secuencia-promocion-descenso-clubes}{H}


\subsection*{Diagrama de secuencia de composición de calendarios}
En la figura \ref{diagrama-secuencia-composicion-calendarios} se observa cómo se
crean nuevas instancias de las jornadas que estarán relacionadas con los clubes
en los que se coincida en las mismas divisiones.  \newpage

\figura{diagramas_secuencia/crear_calendario.png}{scale=0.36,
  angle=270}{Diagrama de secuencia de composición de
  calendarios}{diagrama-secuencia-composicion-calendarios}{H}

\section{Sistema de simulación}
Este apartado de la memoria requiere una especial mención; ya que es el sistema
que hará que el que los cambios y modificaciones a la táctica, alineación,
entrenamientos y precio de entradas que realice en su equipo, tengan un reflejo
directo en el desarrollo de los partidos que conforman la temporada que disputa.

Para perfilar el sistema de simulación de nuestro proyecto de fin de carrera
tuvimos varias reuniones con nuestros tutores y acotamos la forma de simular un
partido en base a la táctica y alineación que presentan los clubes que
disputarán el partido.

A continuación mostraremos la forma en que nuestro algoritmo arroja un resultado
para el partido.

\subsection{Número de jugadas}
El sistema de simulación hará varios cálculos para conformar cuantas serán las
jugadas de ataque de las que dispondrá cada uno de los clubes en base a la
calidad ponderada de los atributos que los jugadores poseen en la táctica
seleccionada por el usuario.
\subsubsection{Cálculo general}
Dicha forma de realizar la ponderación se estipulará en un fichero yml como el
siguiente, perteneciente en este caso a la táctica de fútbol 4-4-2 (es decir, el
portero, 4 defensas, 4 centrocampistas, y 2 delanteros):

\lstinputlisting[style=yml]{codigo/4-4-2.yml}

En dicho fichero podemos ver la ponderación de cada atributo que damos a un
jugador según la posición que éste ocupe en el sistema táctico, la cual hemos
denominamos en el sistema como \cursiva{calidad táctica}.

Una vez tengamos calculada la \cursiva{calidad táctica} de todos los integrantes
del once inicial en nuestro equipo nos dispondremos a sacar la media de dichas
calidades y asignaremos un número de jugadas a cada equipo con un máximo de 88
jugadas por partido con una variación aleatoria de un $\pm$20\%.

Esto quiere decir que si se enfrentan dos equipos cuyos onces iniciales tienen
la misma media, por ejemplo 50, cada equipo dispondrá de 44$\pm$9 jugadas de
ataque, las cuales podrán ser contrarrestadas si con la táctica seleccionada se
consigue robar jugadas de ataque o neutralizar las jugadas de gol.

\subsubsection{Jugada de ataque}
Una vez que hemos calculado el número de jugadas de ataque del que dispone cada
equipo, hemos de simular qué tipo de jugada va tener lugar en el desarrollo del
partido. Para ello se hará uso de la posibilidad de emular la tirada de un dado
a través de la generación de un número aleatorio para reflejar la aleatoriedad y
la suerte que existe en la vida real a la hora de jugar un partido de fútbol.

En nuestro sistema hemos decidido que existan 15 tipos de jugadas de ataque,
entre ellas, la que marca la diferencia a la hora de simular un partido, el gol.

En concreto todas las jugadas contempladas por el simulador, serán las siguientes:
\begin{enumerate}
\item El jugador que posee el balón marca gol.
\item El jugador que posee el balón consigue dar un buen pase a otro compañero
  de su equipo.
\item El jugador que posee comete un error y pierde la pelota que va parar al equipo
  contrario.
\item El jugador que posee el balón lo despeja por sentirse presionado.
\item El jugador que posee el balón lo manda fuera provocando lo silbidos de una
  buena parte de su afición.
\item El jugador del equipo que no posee el balón comete una falta.
\item El jugador que posee el balón hace un regate que deja sentado al
  adversario.
\item El jugador que posee el balón hace un espectacular pase sin mirar.
\item El jugador que posee el balón hace un regate que deja sentados a varios
  adversarios.
\item El jugador que posee el balón tira a puerta pero el portero del equipo
  contrario lo para.
\item El jugador que posee el balón tira un disparo al poste.
\item El jugador que posee el balón realizar una jugada muy peligrosa, pero el
  balón acaba saliendo por la línea de fondo.
\item El jugador que posee el balón da un mal pase que va parar a un jugador del
  otro equipo.
\item El jugador que posee el balón da un pase a un compañero que se encuentra
  en fuera de juego.
\item El jugador del equipo que no posee el balón consigue robar el balón de
  forma limpia
\end{enumerate}

Por tanto, podemos imaginar nuestro sistema de simulación como un dado de 15
caras en el cual existe una de ellas correspondiente a marcar gol, de esta forma
tenemos un quinceavo de posibilidades por jugada de marcar.

\subsubsection{Neutralizar jugadas de gol}
Una vez calculadas el número de jugadas, será posible resistir una posible
jugada de gol dependiendo de la suma de \cursiva{calidades tácticas} de los
jugadores situados en defensa para el equipo que sufre la jugada de gol,
respecto a la suma de calidades \cursiva{tácticas} de los jugadores atacantes
del equipo que está haciendo uso de la jugada de gol.

De esta forma se consigue emular la dificultad que supondría el marcar un gol
para un equipo que utiliza una táctica estándar 4-4-2 contra un equipo que
utiliza una táctica 5-4-1, respetando aún así la calidad de dichos jugadores, ya
que será necesario que los jugadores dispuestos en la táctica tengan una buena
calidad para conseguir una buena suma y así disponer de más facilidad para
neutralizar jugadas de gol.

\subsubsection{Robar jugadas de ataque}
De la misma forma que es posible neutralizar jugadas de gol, para contemplar el
beneficio de tener un centro del campo poblado, el sistema sumará las calidades
de los jugadores dispuestos en el medio campo de ambos equipo y los ponderará de
tal forma que se le roben algunas jugadas de ataque al equipo que presente una
suma menor en las \cursiva{calidades tácticas} de los jugadores situados en el
centro del campo.

\subsection{Número de espectadores}
El sistema simulará el número de espectadores que acudieron al estadio en base a
la posición que mantengan los equipos en la tabla clasificatoria, así como la
calidad que posean los equipos que disputan el partido, todo ello ponderado en
base al precio de venta de las entradas para emular la afluencia al estadio de
forma realista.



\backmatter % Apéndices, bibliografia ...


\clearpage
\addcontentsline{toc}{chapter}{Bibliografia y referencias}
\bibliographystyle{plain}
\bibliography{bibliografia}


\addcontentsline{toc}{chapter}{Software usado}
\chapter*{Software utilizado}
\input{programas.tex}

\addcontentsline{toc}{chapter}{Instalación de \LaTeX}
\chapter*{Instalación de \LaTeX}
\input{instalacion.tex}

\input{fdl-1.3.tex}

\end{document}
