% -*-cap2.tex-*-
% Este fichero es parte de la plantilla LaTeX para
% la realización de Proyectos Final de Carrera, protejido
% bajo los términos de la licencia GFDL.
% Para más información, la licencia completa viene incluida en el
% fichero fdl-1.3.tex

% Copyright (C) 2009 Pablo Recio Quijano 

\section{Planificación}
\subsection{Introducción}
La distribución temporal seguida en líneas generales es la descrita a
continuación, de la cual se harán revisiones constantes al tratarse de
una aplicación en la que se desea aprender a desarrollar bajo una
metodología ágil.
\begin{itemize}
  \item Análisis: En esta fase llevaremos a cabo la recogida de todos
    los requisitos funcionales de nuestra aplicación. Al utilizar una
    metodología ágil la recogida de requisitos podrá ser más laxa que
    si hubiéramos seguido un modelo lineal.
  \item Diseño: En esta fase recrearemos diseños conceptuales de cómo
    serán nuestras vistas, y a partir de ellas podremos obtener
    nuestros modelos de datos.
  \item Programación: En esta fase nos dedicaremos a codificar de
    forma sinérgica las diferentes secciones en las que se compone un
    desarrollo basado en el patrón Modelo Vista Controlador
  \item Pruebas: Las pruebas que realizaremos serán:
    \begin{enumerate}
      \item Test de usabilidad
      \item Pruebas de caja negra.
      \item Estudio y resolución de los resultados de las pruebas.
    \end{enumerate}
  \item Documentación: Para la documentación del código usaremos la
    herramienta propia del lenguaje Ruby, Ruby-doc, gracias a esta
    aplicación podremos generar un documento web fácilmente extensible
    y manipulable. En esta fase también desarrollaremos la memoria de
    la aplicación, el manual de usuario y la presentación multimedia
    para difundir nuestra aplicación.
\end{itemize}