% -*-cap2.tex-*- Este fichero es parte de la plantilla LaTeX para la realización
% de Proyectos Final de Carrera, protejido bajo los términos de la licencia
% GFDL.  Para más información, la licencia completa viene incluida en el fichero
% fdl-1.3.tex

% Copyright (C) 2009 Pablo Recio Quijano

\section{Metodología de desarrollo}
La distribución temporal seguida en líneas generales es la descrita a
continuación, de la cual se harán revisiones constantes al tratarse de una
aplicación en la que se desea aprender a desarrollar bajo una metodología ágil.
\begin{itemize}
\item Análisis: En esta fase llevaremos a cabo la recogida de todos los
  requisitos funcionales de nuestra aplicación. Al utilizar una metodología ágil
  la recogida de requisitos podrá ser más laxa que si hubiéramos seguido un
  modelo lineal.
\item Diseño: En esta fase recrearemos diseños conceptuales de cómo serán
  nuestras vistas, y a partir de ellas podremos obtener nuestros modelos de
  datos.
\item Programación: En esta fase nos dedicaremos a codificar de forma sinérgica
  las diferentes secciones en las que se compone un desarrollo basado en el
  patrón Modelo Vista Controlador
\item Pruebas: Las pruebas que realizaremos serán:
  \begin{enumerate}
  \item Test de usabilidad
  \item Pruebas de caja negra.
  \item Estudio y resolución de los resultados de las pruebas.
  \end{enumerate}
\item Documentación: Para la documentación del código usaremos la herramienta
  propia del lenguaje Ruby, Ruby-doc, gracias a esta aplicación podremos generar
  un documento web fácilmente extensible y manipulable. En esta fase también
  desarrollaremos la memoria de la aplicación, el manual de usuario y la
  presentación multimedia para difundir nuestra aplicación.
\end{itemize}

\section{Calendario}

En la figura .... se puede ver un diagrama de Gantt con la división y el tiempo
tomado para realizar cada una de las tareas en las que hemos dividido el
proyecto. La unidad de tiempo básica está expresada en días laborables de 8
horas.

Aunque la etapa de Memoria se ha realizado de forma solidaria con las demás,
para simplificar se ha extraído en un sólo bloque.

\begin{itemize}
\item Entrevistas: Nos reunimos con nuestros tutores de proyecto para recopilar
  los requisitos que deberá tener nuestro proyecto de fin de carrera para que
  sea lo suficientemente completo para pasar la aprobación del tribunal y
  aprender de forma exhausta el framework \cursiva{Ruby On Rails} (5 días).
\item Análisis: Una vez recopilada toda la información en las entrevistas con
  nuestros tutores, extraemos la información importante y la esquematizamos (7
  días).
\item Especificación de Requisitos: Definimos, organizamos, cada uno de los
  requisitos que hayamos extraído y los desarrollamos de forma exhausta. En esta
  fase también describiremos los casos de uso, evitando en la medida de lo
  posible la redundancia de los mismos y maximizando la posibilidad de
  reutilizarlos (10 días).
\item Diseño: En esta fase estudiaremos la mejor forma para incorporar la
  funcionalidad del sistema teniendo en cuenta que usaremos el framework de
  desarrollo rápido de aplicaciones \cursiva{Ruby On Rails}, con especial
  atención a que haremos uso del patrón Modelo-Vista-Controlador (25 días).
\item Implementación: En esta fase codificaremos la aplicación para que sea
  funcional según hemos especificado en la fase de diseño; además, será
  necesario aprender la forma de utilizar de forma correcta el framework
  \cursiva{Ruby On Rails)}. En esta fase va incluida la creación de la
  estructuras necesarias para la interacción con el Sistema de Gestión de Bases
  de Datos (\cursiva{SGBD}) a través de un sistema \cursiva{ORM} (90 días).
\item Pruebas: En este paso comprobaremos que la corrección del sistema que
  hemos desarrollado. En caso de encontrar algún fallo es el momento de
  solucionarlo y repetir las pruebas (5 días).
\item Memoria: Toda la documentación que conforma esta memoria es el resultado
  de detallar cada uno de los documentos que hemos ido generando a través de la
  realización de la misma (40 días).
\item Presentación: Creación y preparación de la presentación que mostraremos en
  la etapa de defensa ante el tribunal (10 días).
\end{itemize}

\section{Presupuesto}

Para realizar una estimación del montante económico del proyecto realizado será
necesario calcular y dirimir todos los recursos que han sido necesarios para
llevarlo a cabo. En dicho presupuesto no se incluirán los materiales de oficina
como serían escritorio, folios, ordenador, impresora, y demás recursos
ofimáticos. Sabiendo que excluiremos todos los gastos que podríamos considerar
como indirectos, sólo tendremos que fijarnos en el gasto que supondría el coste
de mantener a un \cursiva{Analista-programador y Diseñador de página Web} en la
empresa. Para ello consultaremos las tablas salariales de dicha categoría
profesional en el \cursiva{XVI CONVENIO COLECTIVO ESTATAL DE EMPRESAS
  CONSULTORAS, DE PLANIFICACIÓN, ORGANIZACIÓN DE EMPRESA Y CONTABLE, EMPRESAS DE
  SERVICIOS DE INFORMÁTICA Y DE ESTUDIOS DE MERCADO Y DE LA OPINIÓN PÚBLICA} tal
y como vemos en la tabla \ref{tab:salario}. Para el cálculo del pago de la
Seguridad Social aplicaremos un 33\% del coste del salario bruto del trabajador.

\begin{table}[H]
  \begin{center}
  \begin{tabular}{| c | c | c | c |}
    \hline
    Nombre & Coste mensual & Coste anual & Coste jornada (25 días por mes)\\ \hline
    Analista-programador y Diseñador Web & 1.468,54 \euro & 20.559,56
    & 58,75 \euro\\ \hline
    Seguridad Social & 484,82 \euro & 6.784,65 \euro & No aplica \\
    \hline
  \end{tabular}
\end{center}
\caption{Salario según último convenio de las TIC}
\label{tab:salario}
\end{table}


Teniendo en cuenta que el proyecto lo hemos realizado en solitario y nos ha
llevado un total de 192 jornadas laborales, lo que equivaldría a 7 meses y 3
semanas naturales, tendremos que el coste total del proyecto es el referido en
la tabla \ref{tab:coste}

\begin{table}[H]
  \begin{center}
  \begin{tabular}{| c | c | c | c |}
    \hline
    Cantidad & Descripción & Coste unitario & Coste Total\\ \hline
    192 & Jornada Analista-Programador y Diseñador Web & 58,75 \euro &
    11.280 \euro\\ \hline
    8 & Seguridad Social Analista-Programador y Diseñador Web & 484,82
    \euro & 3.878,56 \euro \\ \hline
    \multicolumn{3}{|r|}{Total} &  15.158,56 \euro\\
    \hline
  \end{tabular}
\end{center}
\caption{Coste total del proyecto realizado}
\label{tab:coste}
\end{table}