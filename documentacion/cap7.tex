% -*-cap7.tex-*- Este fichero es parte de la plantilla LaTeX para la realización
% de Proyectos Final de Carrera, protejido bajo los términos de la licencia
% GFDL.  Para más información, la licencia completa viene incluida en el fichero
% fdl-1.3.tex

% Copyright (C) 2009 Pablo Recio Quijano

La fase de pruebas es una de las partes más importantes del desarrollo software
\cite{test:myers}. El objetivo de las pruebas software es la verificación de que
el proyecto cumple con los requisitos especificados inicialmente cuando se
comenzó el desarrollo. Según la metodología clásica de desarrollo software
existen diferentes enfoques a la hora de realizar las pruebas de software,
siendo todos ellos complementarios entre sí.

En este capítulo vamos a describir las pruebas que hemos que realizado en
nuestro software para asegurar la corrección del mismo.

\section{Tipos de Pruebas}
Este proyecto es una mezcla entre un videojuego online y una aplicación de
gestión. Por ello deberemos de hacer varios enfoques a la hora de hacer las
pruebas de nuestro sistema software, a saber:

\begin{itemize}
\item \negrita{Pruebas clásicas:} Aquí deberíamos englobar las pruebas que se
  aplican a cualquier tipo de desarrollo software:
  \begin{itemize}
  \item Pruebas unitarias: Son aquellas que nos permiten probar el correcto
    funcionamiento de un módulo de código. De esta forma conseguimos saber que
    cada uno de los módulos que integrar el sistema software funcionan
    correctamente por separado. \cite{test:unitaria}
  \item Pruebas de integración: Son aquellas que se realizan en el ámbito del
    desarrollo software una vez que han sido aprobadas las pruebas unitarias. Se
    refieren a la prueba de todos los elementos unitarios que componen un
    proceso realizadas en conjunto. De esta forma conseguimos verificar que las
    partes de un sistema software funcionan de forma
    conjunta. \cite{test:integracion}
  \item Pruebas funcionales: Son pruebas basadas en la ejecución, revisión y
    retroalimentación de las funcionalidades previamente diseñadas para el
    software \cite{test:funcionales}.
  \end{itemize}
\item \negrita{Pruebas de usabilidad}: Son aquellas pruebas que se encargan de
  medir cómo de bien puede una persona usar un sistema hecho por el hombre, como
  en nuestro caso es la página web. Consisten en seleccionar un grupo de
  usuarios y solicitarles que lleven a cabo las tareas para las cuales fue
  diseñada la aplicación, mientras que el equipo de desarrollo toma nota para
  evaluar la respuesta del usuario \cite{test:usability}.
\end{itemize}

El primer grupo de pruebas pueden realizarse de forma automatizada, sin embargo,
para la realización de las pruebas del segundo grupo será necesario la
intervención de un grupo de usuarios que interactúen con la aplicación y que
contesten un cuestionario para recopilar la información necesaria para la mejora
de la aplicación.

\section{Pruebas unitarias}
Durante la fase de codificación perteneciente al capítulo
\ref{cap:implementacion} se fueron realizando al unísono las pruebas unitarias
de cada componente software de forma no automatizada, evitando de esta forma en
la medida de lo posible que en la fase de integración aparecieran errores
referentes a la codificación.

\cursiva{Ruby On Rails} proporciona un completo \cursiva{framework} para el
desarrollo de pruebas unitarias, de hecho, por cada modelo de nuestro sistema
software, dentro del directorio \texttt{test/unit} se crea un fichero homónimo
para la realización de pruebas unitarias. En nuestro caso, y debido a la
necesidad de cumplir con el calendario de entrega del proyecto, no hemos hecho
uso del mismo.

Una herramienta muy útil a la hora de desarrollar nuestro software y evitar
errores en el código ha sido el uso del programa \cursiva{SublimeText2} con el
plugin \cursiva{IntelLinter}. Este plugin es una herramienta que analiza las
librerías y archivos del proyecto software y detecta un conjunto de errores
sintácticos a través del análisis del código fuente.

\section{Pruebas de integración}
Según se iban desarrollando las diferentes funcionalidades del proyecto y estos
iban cumpliendo los requisitos software recopilados en el capítulo
\ref{cap:analisis} a través de las pruebas unitarias de cada uno de de los
módulos que íbamos completando comprobamos que la integración de los mismos no
repercutían en el software haciendo que se incumpliera alguno de los requisitos
recolectados en la fase de análisis (Capítulo \ref{cap:analisis}).


\section{Pruebas funcionales}
Una vez que habíamos comprobado que la integración de los módulos no remitía
ningún tipo de error ni comportaba ningún tipo de comportamiento extraño en los
módulos ya testeados con sus pruebas unitarias nos dispusimos a comprobar que el
software desarrollado cumplía con los requisitos funcionales recopilados en la
fase de análisis en el capítulo \ref{cap:analisis}.

\section{Pruebas de usabilidad}
Para este proyecto se ha puesto especial énfasis en que la usabilidad de la web
sea sencilla para cualquier tipo de usuario y para que además se visualice de
forma correcta en la mayoría de navegadores web del mercado. Para ello se ha
hecho uso del \cursiva{framework CSS} \cursiva{Twitter Bootstrap}
\cite{prog:twitter_bootstrap}.

Las características de este \cursiva{framework CSS} son varias, entre ellas:

\begin{itemize}
\item \negrita{Multiplataforma}: Compatible con los navegadores web más
  actuales, ya sean móviles, tablets o de escritorio, inclusive para
  \cursiva{Internet Explorer 7}.
\item \negrita{Tabla de 12 columnas}: Trae predefinida una plantilla de 12
  columnas que puede ser modificada.
\item \negrita{Diseño Responsivo}: Los componentes de la página web se escalan
  según el tamaño del navegador y la resolución a la que son visualizados los
  mismos.
\item \negrita{Guía de estilos documentada}: Es de los poco \cursiva{frameworks
    CSS} que poseen una guía de estilos bien documentada por sus creadores para
  la modificación de la plantilla.
\item \negrita{Plugins jQuery personalizados}: De esta forma cubrimos las
  necesidades más habituales en el desarrollo de una página web. Alertas,
  ventanas modales, popups, pestañas, etc...
\item \negrita{Desarollado con LESS/SASS:} Aunque \cursiva{Twitter Bottstrap}
  fue desarrollado para su perfecta integración con \cursiva{LESS}, dado que
  \cursiva{Rails} es compatible de forma nativa con \cursiva{SASS} las
  características de la definición de reglas se han migrado a este último. De
  esta forma podemos aprovechar las cualidades de anidado, \cursiva{mixins},
  declaración de variables, etc...
\end{itemize}

En las figuras \ref{fig:bootstrap1} y \ref{fig:bootstrap2} podemos ver la
característica de diseño responsivo con la plantilla original de
\cursiva{Twitter Bootstrap}

\newpage \figura{full_bootstrap.png}{scale=0.35,angle=270}{Twitter bootstrap
  maximizado en Firefox}{fig:bootstrap1}{H}


\figura{compact_bootstrap.png}{scale=0.5}{Twitter bootstrap minimizado en
  Firefox}{fig:bootstrap2}{H}

\subsection{Casos de pruebas con usuarios reales}
Dado que la aplicación es web, se prevé que a ella accedan un gran número de
usuarios. Dado que el deporte al que está orientado el software es el fútbol, se
prevé que la edad de los posibles usuarios sea de cualquier rango de edades,
desde infantes hasta personas adultas.

Por ello hemos seleccionado un conjunto de sujetos de prueba que abarquen
diversas edades y sexo para conseguir una información lo suficientemente
representativa y variada para perfeccionar nuestro sistema.

Gracias a estas pruebas hemos conseguido valiosa información para ir haciendo
modificaciones al producto para hacerlo más completo y entretenido para el
usuario.

Los usuarios que han realizado los casos de prueba son los siguientes:

\begin{description}
\item[Sujeto 1] Hombre, 15 años, bachillerato de la rama de ciencias
  sociales. Nivel de informática usuario alto. Conocimientos en ofimática y
  navegación web, mucha experiencia en videojuegos y redes sociales, alto nivel
  del panorama futbolístico.
\item[Sujeto 2] Mujer, 20 años, estudios universitarios de la rama de
  letras. Nivel de informática usuario, alto. Conocimientos en ofimática,
  navegación web, redes sociales y dispositivos móviles, bajo nivel del panorama
  futbolístico.
\item[Sujeto 3] Hombre, 25 años, estudios no universitarios. Nivel de
  informática usuario, alto. Conocimientos en ofimática, navegación web, redes
  sociales, desarrollo web. Conocimiento del panorama futbolístico, medio.
\item[Sujeto 4] Hombre, 35 años, estudios universitarios de la rama de
  ciencias. Alto conocimiento en ofimática, desarrollo software, navegación
  web. Conocimiento alto del panorama futbolístico.
\item[Sujeto 5] Hombre, 55 años, estudios universitarios de la rama de
  ciencias. Conocimiento medio de ofimática, bajo nivel de conocimiento en
  navegación web. Alto conocimiento del panorama futbolístico.
\end{description}

Los objetivos que han de pasar los usuarios son:

\begin{description}
\item[Objetivo 1] Ser capaz de registrarse, iniciar sesión y salir de la cuenta
  de usuario.
\item[Objetivo 2] Ser capaz de crear su propia alineación para el siguiente
  partido y la táctica que desea utilizar.
\item[Objetivo 3] Ser capaz de estipular el precio de las entradas para el
  siguiente partido disputado en su estadio.
\item[Objetivo 4] Ser capaz de poner a entrenar a cualesquiera de los jugadores
  de su equipo desee y el entrenamiento que así fije.
\item[Objetivo 5] Ser capaz de emitir, cancelar, recibir, aceptar y rechazar una
  oferta por uno de sus jugadores.
\item[Objetivo 6] Ser capaz de visualizar las jornadas siguientes de la
  temporada actual en curso.
\end{description}

Para realizar estos casos de prueba por parte de los sujetos del estudio se
contemplaron las siguientes variables:

\begin{itemize}
\item Todas las pruebas se realizaron bajo el navegador web Firefox 14 instalado
  bajo Windows 7.
\item Se les proporciona un enlace web a la aplicación web alojada en un
  servidor remoto \cite{prog:pfc_sfo}
\item No se les dio ninguna indicación de cómo navegar a través de la página
  web.
\end{itemize}

Los resultados de las pruebas realizadas según los objetivos propuestos se
pueden observar en la tabla \ref{tab:pruebas}

\begin{table}[H]
  \begin{center}
    \begin{tabular}{ | c | c | c | c | c | c | c |}
      \hline
      & Objetivo 1 & Objetivo 2 & Objetivo 3 & Objetivo 4 &  Objetivo 5 & Objetivo
      6\\ \hline
      Sujeto 1 & \checkmark & \checkmark & \checkmark & \checkmark & \checkmark &
      \checkmark \\ \hline
      Sujeto 2 & \checkmark & \checkmark & \checkmark & \checkmark & \checkmark &
      \checkmark \\ \hline
      Sujeto 3 & \checkmark & \checkmark & \checkmark & \checkmark & \checkmark &
      \checkmark \\ \hline
      Sujeto 4 & \checkmark & \checkmark & \checkmark & \checkmark & \checkmark &
      \checkmark \\ \hline
      Sujeto 5 & \checkmark & \checkmark &  X & X  & X & X \\ \hline
    \end{tabular}
  \end{center}
  \caption{Pruebas pasadas por los usuarios}
  \label{tab:pruebas}
\end{table}

Como podemos observar el único usuario con problemas para terminar de pasar las
pruebas fue el Sujeto 5, no es de extrañar puesto que sus conocimientos
informáticos son básicos y tiene poca intuición al haber usado muy poco
aplicaciones web interactivas.
