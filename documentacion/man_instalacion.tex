% -*-man_instalacion.tex-*- Este fichero es parte de la plantilla LaTeX para la
% realización de Proyectos Final de Carrera, protejido bajo los términos de la
% licencia GFDL.  Para más información, la licencia completa viene incluida en
% el fichero fdl-1.3.tex

% Copyright (C) 2009 Pablo Recio Quijano

\section{Suposiciones previas}
Supondremos que el sistema dónde vamos a instalar la aplicación es una
distribución \cursiva{Ubuntu 12.04} y que el \cursiva{SGBD} que utilizaremos
será \cursiva{SQLite3}.

\section{Instalación}
\subsection{Preparación del sistema}
Para configurar el sistema vamos a instalar un gestor de versiones de
\cursiva{Ruby} llamado \cursiva{RVM} \cite{prog:rvm}, gracias a este gestor, el
sistema podrá tener diferentes versiones de interpretes \cursiva{Ruby}.

Para ello primero instalemos los siguientes paquetes:

\begin{lstlisting}[style=consola]
  sudo aptitude install build-essential git-core curl
\end{lstlisting}

Posteriormente nos descargaremos la versión más reciente de \cursiva{RVM}

\begin{lstlisting}[style=consola]
  curl -L https://get.rvm.io | bash -s stable
\end{lstlisting}

Esto descargará un \cursiva{script} en \cursiva{bash} que será ejecutado
automáticamente, clonando el repositorio alojado en \cursiva{GitHub}.

Para que \cursiva{RVM} sea reconocido sin problemas por nuestra terminal
ejecutaremos el siguiente comando:

\begin{lstlisting}[style=consola]
  echo '[[ -s "$HOME/.rvm/scripts/rvm" ]] && . "$HOME/.rvm/scripts/rvm" # Load
  RVM function' && ~/.bash_profile
\end{lstlisting}

Por último habremos de instalar unos últimos paquetes paras asegurarnos cumplir
con las dependencias necesarias para instalar \cursiva{Ruby 1.9.2}. Para ello
ejecutaremos el siguiente comando:

\begin{lstlisting}[style=consola]
  sudo aptitude install build-essential bison openssl libreadline6
  libreadline6-dev curl git-core zlib1g zlib1g-dev libssl-dev libyaml-dev
  libsqlite3-0 libsqlite3-dev sqlite3 libxml2-dev libxslt-dev autoconf libc6-dev
  ncurses-dev
\end{lstlisting}

Ya tenemos todas las dependencias necesarias instaladas, ahora sólo
necesitaremos instalar la versión de \cursiva{Rails} en el sistema, para ello
ejecutaremos el siguiente comando:

\begin{lstlisting}[style=consola]
  rvm install 1.9.2
\end{lstlisting}

Una vez se termine de instalar modificaremos nuestro fichero personal de
\texttt{.bashrc} añadiendo al final del fichero la siguiente línea:

\begin{lstlisting}[style=consola]
  source \$HOME/.rvm/scripts/rvm
\end{lstlisting}

Ahora ejecutaremos la siguiente línea para actualizar el sistema de gemas de
\cursiva{Ruby}

\begin{lstlisting}[style=consola]
  gem update --system
\end{lstlisting}

Ya tenemos el sistema preparado para instalar en nuestro sistema
\cursiva{Simulator Football Online}.

\section{Instalación de Simulator Football Online}

Lo primero que tendremos que hacer será bajarnos la última versión del
repositorio de nuestra aplicación, para ello ejecutaremos el siguiente comando:

\begin{lstlisting}[style=consola]
  git clone git://github.com/paliyoes/pfc_sfo_rails3.git pfc_sfo
\end{lstlisting}

Una vez descargado deberemos ejecutar el siguiente comando dentro del directorio
\texttt{pfc\_sfo} de nuestro sistema:

\begin{lstlisting}[style=consola]
  mkdir doc log tmp
\end{lstlisting}

A continuación tendremos que instalar todas las dependencias de nuestro
proyecto, para ello ejecutaremos el siguiente comando:

\begin{lstlisting}[style=consola]
  bundle install
\end{lstlisting}

En caso de no estar instalada la gema bundle, aprovecharemos para instalarla y
volver a ejecutar el comando anterior.

\begin{lstlisting}[style=consola]
  gem install bundle
\end{lstlisting}

Ahora deberemos preparar la base de datos para que pueda correr en el servidor
nuestra aplicación, para ello ejecutaremos los siguientes comandos:

\begin{lstlisting}[style=consola]
  rake db:create 
  rake db:migrate
\end{lstlisting}

Una vez terminado de ejecutar ambos comandos, podremos ejecutar un servidor con
nuestra aplicación a través del comando:

\begin{lstlisting}[style=consola]
  rails s
\end{lstlisting}

Para acceder a la aplicación sólo tendremos que ir a nuestro navegador favorito
y acceder a la ruta \url{http://localhost:3000}
