% -*-cap7.tex-*- Este fichero es parte de la plantilla LaTeX para la realización
% de Proyectos Final de Carrera, protejido bajo los términos de la licencia
% GFDL.  Para más información, la licencia completa viene incluida en el fichero
% fdl-1.3.tex

% Copyright (C) 2009 Pablo Recio Quijano

En este proyecto desarrollaremos un sistema web de gestión de clubes de fútbol orientado al disfrute
del usuario como de si un videojuego se tratara. Este sistema nos ha permitido
conocer los entresijos del desarrollo software web mediante el uso del
\cursiva{framework} de desarollo rápido de aplicaciones web \cursiva{Ruby On
  Rails}.

\section{Introducción y objetivos del proyecto}
Este proyecto se inició para completar la formación académica de la titulación
de Ingeniería Técnica en Informática de Gestión de la Escuela Superior de
Ingeniería de Cádiz de la Universidad de Cádiz. Gracias a ella, pudimos aprender
cómo desarrollar un producto totalmente funcional usando para ello el uso de un
framework basado en el patrón \cursiva{Modelo-Vista-Controlador} \cursiva{Ruby
  On Rails}. 

Además, comprendimos la dificultad que conlleva el desarrollo de un sistema
software, desde la misma etapa de conceptualización del sistema.

El objetivo del proyecto es proporcionar un videojuego web basado en la gestión
de clubes de fútbol que se adapte a las condiciones de licencia del software
libre que sea capaz de competir en un futuro con las alternativas actualmente
existentes en el mercado actual.

Al concluir el desarrollo, la aplicación ha de ser capaz de desempeñar todas las
funcionalidades descritas en el capítulo \ref{cap:analisis}.

\section{Análisis del sistema}
Para conocer los requisitos funcionales del sistema a desarrollar nos reunimos
varias veces con nuestro tutor de proyecto en las primeras semanas de desarrollo
del sistema.

De dichas reuniones, fuimos capaces de llegar a numerosas conclusiones sobre
cuáles serían los límites y funcionalidades que habríamos de desarrollar en el
proyecto, para poder crear la Especificación de Requerimentos (SRS)
\cite{doc:requisitos}

\subsection{Características de los usuarios}
El producto está dirigido a todo usuario que sepa manejar un navegador web y que
esté interesado en la simulación de la gestión de un club de fútbol, así como
conocer las reglas y premisas básicas de negocio del deporte.

Identificamos dos roles bien diferenciados en nuestro sistema:

\begin{enumerate}
\item \negrita{Usuario Jugador}: Se refiere a una persona que se registra en el
  portal web y que participará en la gestión de un club al que se le
  asignará. Este usuario podrá conectarse al sistema e interactuar con el mismo
  a través de la interfaz web.
\item \negrita{Usuario Administrador}: Es un tipo de usuario que será capaz de
  modificar el comportamiento del sistema a través del menú de administración de
  nuestra aplicación para llevar a cabo las acciones de gestión de la
  competición de nuestro sistema.
\end{enumerate}

\subsection{Requisitos de rendimiento}
Este aspecto es fundamental en el análisis del sistema, se ha de tener en cuenta
que la aplicación es de tipo servidor y que a ella estarán conectados
concurrentemente un número de usuarios elevado.

Se ha considerado también el rendimiento que nos ofrece el diseño web
realizado. Para que nuestra página web sea compatible con la mayoría de
dispositivos y se ejecute lo suficientemente fluida hemos decidido no usar la
tecnología propietaria \cursiva{Flash}, usando para la renderización de efectos
la librería \cursiva{JQuery} en combinación con hojas de estilo \cursiva{CSS}.

El rendimiento de la aplicación podrá ser menor en la primera instanciación de
la aplicación debido a que no se habrá cacheado ningún contenido.

\subsection{Requisitos de interfaces externas}
\subsubsection{Interfaces de usuario}
Podemos dividir los requisitos de interfaz de usuario según tres roles:

\begin{itemize}
\item \negrita{Usuario Jugador y Usuario Administrador: }
  \begin{itemize}
  \item Interfaz atractiva para los usuarios: Es fundamental para la buena
    aceptación de nuestra aplicación el crear una interfaz lo suficientemente
    clara, dinámica y estética para que el usuario sienta una inmersión en la
    experiencia de juego. Una interfaz no atractiva sólo traerá problemas de
    rechazo y disconformidad por parte del usuario.\\\\
    Para que la interfaz sea sencilla de modificar y tratándose de una
    aplicación web usaremos \cursiva{hojas de estilo CSS}, así todo el apartado
    de diseño gráfico podrá ser modificado posteriormente sin que por ello
    afecte al contenido ni la estructura del documento \cursiva{HTML}.
  \item Una característica de nuestra aplicación es que el sistema de avisos
    está embebido dentro de la página web y es muy parecido al sistema de
    notificación de los sistemas operativos \cursiva{Mac OS X} y de la \cursiva
    {distribución Linux Ubuntu}.
  \end{itemize}
\item \negrita{Superusuario: } En este caso la interfaz corresponderá a la que
  tenga el propio sistema operativo dónde esté instalado la aplicación.
\end{itemize}
\subsubsection{Interfaz con el hardware}
Las copias de seguridad del sistema se realizarán de forma periódica cada día, semana y mes de for-
ma incremental y utilizando la herramienta rsync. Dichas copias de seguridad se almacenarán en otro
servidor dedicado exclusivamente al almacenaje de las copias.

\subsubsection{Interfaz con el software}
La aplicación web correrá en el lado del servidor en un sistema \cursiva{Linux}
con \cursiva{Ubuntu Server 12.04}. Mientras que en el lado cliente podrá
ejecutarse en cualquier sistema que disponga de un navegador web que cumpla los
estándares 2.0 WCC.

\subsection{Requisitos de rendimiento}
Este aspecto es fundamental en el análisis del sistema, se ha de tener en cuenta que la aplicación es de
tipo servidor y que a ella estarán conectados concurrentemente un número de
usuarios elevado.

Se ha considerado también el rendimiento que nos ofrece el diseño web realizado. Para que nuestra
página web sea compatible con la mayoría de dispositivos y se ejecute lo suficientemente fluida hemos
decidido no usar la tecnología propietaria \cursiva{Flash}, usando para la renderización de efectos la librería
\cursiva{JQuery} en combinación con hojas de estilo \cursiva{CSS}.

El rendimiento de la aplicación podrá ser menor en la primera instanciación de la aplicación debido a
que no se habrá cacheado ningún contenido.

\subsection{Requisitos de Información}
El sistema almacenará en forma de tablas la siguiente información relativa a las
siguientes entidades:

\begin{description}
\item[User] Se almacenan los datos de cada uno de los usuarios que
  interactúan con la aplicación.
\item[Club] Se almacenan los datos de los clubes que participan en
  las ligas.
\item[League] Se almacenan los datos de las ligas que existen en el
  sistema.
\item[Season] Se almacenan los datos de las temporadas
  correspondientes a cada liga del sistema.
\item[Round] Se almacenan los datos de las jornadas de cada
  temporada.
\item[MatchGeneral] Se almacenan los partidos que se han de jugar
  cada jornada.
\item[MatchDetail] Se almacenan las acciones que tienen lugar en cada
  partido.
\item[LineUp] Se almacenan las alineaciones de cada equipo y partido.
\item[ClubFinancesRound] Se almacenan los datos de finanzas
  correspondientes a cada jornada y club.
\item[Player] Se almacenan los datos de los jugadores de fútbol de
  los clubes.
\item[Offer] Se almacenan los datos de las ofertas que se realizan
  los usuarios de la aplicación sobre los jugadores de fútbol.
\item[Training] Se almacenan los datos de los entrenamientos de los
  jugadores de fútbol.
\item[AdminMessages] Se almacenan los mensajes de administración
  hacia los usuarios.
\end{description}

\section{Diseño del sistema}
Para el diseño del sistema tuvimos que tomar algunas decisiones básicas, sobre
todo orientadas a qué tecnologías emplear.

Entre ellas destacamos el uso de la tecnología de Software Libre \cursiva{Ruby
  On Rails}, basado en el lenguaje de \cursiva{script} \cursiva{Ruby}. Dicho
\cursiva{framework} nos proporciona todas las herramientas necesarias para el
desarrollo del sistema incluyendo el de la interfaz cliente.

\subsection{Arquitectura}
La arquitectura que emplearemos es la que mejor se ajusta a las aplicaciones
cliente/servidor, aplicando el patrón \cursiva{Modelo-Vista-Controlador} que nos
proporciona de base \cursiva{Rails} \cite{agilerails}.

El patrón \cursiva{Modelo-Vista-Controlador} divide el software en tres capas
bien diferenciadas que se adaptan muy bien a la tecnología
Cliente-Servidor. Dichas capas son:

\begin{itemize}
\item \negrita{Modelo}: Esta capa es la relativa a la representación de la
  información con la que el proyecto interactuará. Está compuesto por las clases
  necesarias para el acceso, modificación, búsqueda y demás operaciones con los
  datos.
\item \negrita{Vista}: En esta capa agruparemos todas las clases necesarias para
  la representación de la interfaz con la que nuestros usuarios interactuarán.
\item \negrita{Controlador}: Será la capa que nos permita gestionar las
  peticiones que lleguen a nuestro servidor.
\end{itemize}

Los modelos usan el motor de persistencia \cursiva{ActiveRecord} integrado en
\cursiva{Rails} y pasado en el patrón de mismo nombre \cursiva{Active Record}
\cite{pattern:activerecord}. Esta clase es la que en última instancia ejecuta
las órdenes de escritura y lectura en lenguaje \cursiva{SQL} en el
\cursiva{SGBD} que el administrador haya decidido usar, en nuestro caso
\cursiva{SQLite3}.

La vista son los ficheros con extensión \texttt{.erb} basados en el sistema de
plantillas \cursiva{eRuby}.

\subsection{Diagrama de componentes}
Podemos ver el diagrama de componentes en la figura \ref{diag:componentes}.

\figura{mvc-rails.png}{scale=0.50}{Diagrama de componentes}{diag:componentes}{H}

El framework \cursiva{Ruby on Rails} hace especial hincapié en tres conceptos
muy básicos que caracterizan toda su estructura, que son:

\begin{itemize}
\item \negrita{Convention over Configuration (COC)}: Convención frente a
  configuración.
\item \negrita{Don't Repeat Yourself (DRY)}: No te repitas
\item \negrita{Keep It Simple (KIS)}: Mantenlo simple.
\end{itemize}

Gracias a ello seguiremos la estructura de directorios en árbol para nuestro
código fuente que genera \cursiva{Rails} para una aplicación vacía.
\subsection{Entorno de ejecución}
En la figura \ref{fig:entornoejecucion} podemos ver el diagrama de despliegue de
la aplicación puesta en marcha en cualquiera de los entornos, ya sea este de
desarrollo, test o producción.

El sistema (servidor) interactuará con uno o más clientes web (navegadores). El
servidor, a su vez estará formado por un servidor de aplicaciones web, en
nuestro caso \cursiva{Webrick}, y el sistema de gestión de base de datos será
\cursiva{SQLite3}.

Gracias a esta sencilla configuración no tendremos que configurar ninguna cuenta
ni servicio en nuestro servidor web.
\subsection{Diagramas de clases conceptuales}
A continuación mostramos algunos de los modelos de clases conceptuales en
notación \cursiva{UML} \cite{uml:distilled_standard}

\clearpage

\figura{models_complete.png}{scale=0.25}{Diagrama de Clases UML de la capa
  modelo}{diag:clases_modelo}{H}

\newpage

\figura{controllers_complete.png}{scale=0.22,angle=270}{Diagrama de Clases UML
  de la capa controlador}{diag:clases_controlador}{H}
\clearpage
\section{Implementación}
Para la implementación del software hemos utilizado los siguientes lenguajes:

\begin{itemize}
\item Ruby.
\item HTML.
\item CoffeScript \cite{lang:coffescript}.
\item SASS \cite{lang:sass}.
\end{itemize}

Y las aplicaciones utilizadas para codificar a de una forma sencilla y
productiva:

\begin{itemize}
\item El editor de texto extensible \cursiva{SublimeText2} \cite{prog:sublimetext2}.
\item El depurador de \cursiva{Firefox} \cite{prog:firefox} \cursiva{Firebug}.
  \cite{prog:firebug}.
\item El controlador de versiones \cursiva{GIT} \cite{wiki:GIT}.
\item El \cursiva{framework CSS} \cursiva{TwitterBootstrap}.
  \cite{prog:twitter_bootstrap}.
\end{itemize}

\section{Pruebas y validación}
En la fase de análisis comprobamos la validez de nuestros requisitos
identificando de forma correcta cuáles serían las funcionalidades, restricciones
y el ámbito de desarrollo y ejecución de nuestra aplicación; además de comprobar
que dichos requisitos no se contradicen ni acoplan.

Posteriormente validamos que la especificación de requisitos era correcta con
nuestro tutor de proyecto de fin de carrera.

Las pruebas efectuadas sobre el entorno de desarrollo han sido de tipo manual en
todo momento debido a la necesidad de cumplir con el calendario de entrega de
las aplicación.

\section{Conclusiones y trabajo futuro}
Este proyecto ha servido para poder completar la formación académica recibida en
la Escuela Superior de Ingeniería orientada a la finalización de la titulación
de Ingeniería Técnica en Informática de Gestión de la Universidad de Cádiz.

El desarrollo de este sistema software nos ha permitido conocer en profundidad
cómo se realiza el desarrollo de una aplicación a través de la utilización de un
\cursiva{framework} y las dificultades y retos que esto conlleva.

Este sistema es altamente ampliable debido a la naturaleza de videojuego del
mismo. Una de las características a mejorar y ampliar sería la de la creación de
un sistema de simulación de partidos basadas en tecnologías avanzadas de
inteligencia artificial, lo cual llevaría asociada una plena experiencia de usuario.