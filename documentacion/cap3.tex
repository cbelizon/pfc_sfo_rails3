\section{Perspectiva del Producto}
En este capítulo explicaremos qué pretendemos conseguir con la realización de
este proyecto y qué características generales ha de cumplir.

\subsection{Dependencias del producto}
El proyecto es independiente, no es un subsistema de un proyecto de mayor
envergadura, tampoco es continuación de ningún otro proyecto, pero al estar
liberado con una licencia de \cursiva{software libre} podría ser continuado,
mejorado y actualizado por su propio creador o terceras personas ajenas a éste.

\subsection{Interfaces de usuario}
La interfaz de usuario se basa en una página web dinámica que lleva asociada
varias opciones para poder usar las herramientas
proporcionadas.\\

Para el uso de la aplicación, la página principal dispone de varios menús, entre
ellos la de selección de idiomas y un \cursiva{hipervínculo} al
subsistema de registro e inicio de sesión.\\

Todo el diseño de la aplicación respeta los estándares \cursiva{HTML 4.0} y
\cursiva{CSS 2.0 } así como el uso de la librería de \cursiva{JavaScript JQuery}
lo que nos proporciona una gran
compatibilidad con cualquier navegador web del mercado.\\

Al usar \cursiva{hojas de estilo CSS} la apariencia de nuestra aplicación puede
ser fácilmente modificada para adaptarla a nuevas
tendencias de diseño de forma sencilla y eficiente.\\

Siempre que se termine cualquier operación, ya sea exitosa o no, el sistema nos
informará a través de un sistema de notificación embebido dentro del navegador
muy parecido al sistema usado en \cursiva{Mac OS
  X } y últimas versiones de la \cursiva{distribución Linux Ubuntu}.\\

Como mejora para un uso mucho más cómodo y rápido por parte de cualquier
usuario, la aplicación dispone de un menú estático para acceder a cualquier tipo
de herramienta o funcionalidad según el usuario sea un jugador de la aplicación
o sea el administrador.

\subsection{Interfaces software}
El producto interactúa con el sistema operativo en el lado del servidor en dónde
está instalado y con el navegador usado en el lado
del cliente.\\

De hecho, el producto es totalmente multiplataforma tanto en el lado servidor
como en el lado cliente, gracias a que el \cursiva{framework RoR} está diseñado
para ser instalado en entornos \cursiva{Windows, Linux y Mac OS X}; mientras que
en el lado cliente disponemos de varios navegadores web cualquiera que sea el
sistema operativo en el que se
ejecute.\\

Debemos destacar que el mencionado \cursiva{framework} tiene una gran comunidad
que lo sustenta, que da como resultado la disponibilidad de una gran cantidad de
documentación y un buen soporte de la plataforma,
repercutiendo directamente en la calidad del software que desarrollemos.\\

Hemos de indicar que \cursiva{RoR} es agnóstico en cuanto al \cursiva{SGBD} que
utilicemos, con lo que el producto podrá ser desplegado con cualquiera de los
\cursiva{SGBD} más comunes como son \cursiva{Oracle, MySQL, SQLite2, SQLite3 y
  Postgre}, a través de una \cursiva{gema} específica y una simple modificación
en uno de los archivos de configuración.

\subsection{Operaciones}
Existen dos tipos de usuarios dentro de la aplicación:

\begin{itemize}
\item \negrita{Superusuario: } Es el responsable de instalar, mantener,
  interconectar con la \cursiva{base de datos} y optimizar el servidor dónde se
  encuentra la aplicación.
\item \negrita{Administrador: } Es el responsable de toda la gestión de
  campeonatos, así como de dar de alta a los nuevos usuarios para que puedan
  empezar a interactuar con las funcionalidades de la aplicación.
\item \negrita{Usuario: } Conforma el resto de usuarios del sistema que
  participan en la simulación del entorno de gestión de equipos de fútbol, a
  excepción de las funcionalidades exclusivas del administrador.
\end{itemize}

\subsection{Requisitos de adaptación a la ubicación}
Para poder adaptar la aplicación al entorno dónde se instalará será necesario
disponer de un \cursiva{SGBD} compatible con \cursiva{RoR} además de modificar
el fichero de configuración para conectarlo con nuestro sistema.

\subsection{Funciones del producto}
La función principal del producto que estamos presentando es la de desarrollar
un videojuego multijugador dónde diversos usuarios puedan interactuar en un
entorno que simule ser real a través de \cursiva{Internet}.  Para ello ha sido
necesario crear una aplicación informática encargada de poder registrar a
cualquier usuario que acceda a nuestra web y proporcionarles las herramientas
necesarias para gestionar su propio club; además de simular los partidos que en
una competición de fútbol
se desarrollan.\\

Las características más importantes de esta aplicación se describen a
continuación:

\begin{itemize}
\item Posibilidad de registro y control de acceso a una cuenta de usuario
  personal e intransferible que distinga al usuario unívocamente de los demás a
  través de una simple interfaz web.
\item Capacidad para la gestión de todos los aspectos económicos de los clubes
  de los usuarios registrados.
\item Capacidad para la visualización del calendario y clasificación de todos
  los clubes y usuarios registrados en el sistema.
\item Capacidad para modificar la estructura deportiva y técnica de los clubes
  de los usuarios registrados.
\item Capacidad para simular un partido de forma realista y mostrarlo a los
  usuarios como si se estuviera desarrollando en directo.
\item Capacidad para gestionar el estado de la competición.
\item Capacidad para escoger el idioma en el que se mostrará la información de
  nuestra aplicación.
\end{itemize}

Aquí sólo se han descrito las características generales que realiza la
aplicación, se podrá comprobar toda la funcionalidad en los puntos siguientes de
esta memoria, en especial en el \cursiva{Manual de usuario}.

\section{Características de los usuarios}

Para el correcto manejo de la aplicación \cursiva{Simulador de fútbol
  online} existen dos perfiles distintos de manejo.\\

Podríamos dividir los perfiles en:

\begin{itemize}
\item \negrita{Usuario (Jugador):} Necesitan unos conocimientos básicos de
  informática para su manejo, sobre todo orientado al manejo de aplicaciones
  web. También será necesario un conocimiento meramente básico de cuáles son las
  reglas de negocio y deportivas de un entorno de gestión de clubes deportivos
  de fútbol 11.
\item \negrita{Usuario (Administrador): } Necesitará unos conocimientos básicos
  en informático para su manejo, sobre todo orientado al manejo de aplicaciones
  web. También será necesario conocer cómo es el discurrir de las competiciones
  en el diseño de nuestra aplicación.
\item \negrita{Superusuario (Administrador): } Serán necesarios unos
  conocimientos avanzados de informática, incidiendo en conocimientos de
  despliegue de aplicaciones web.
\end{itemize}

\section{Restricciones generales}
El proyecto se ha desarrollado bajo la metodología \cursiva{Rational Unified
  Process (RUP)}, proceso de desarrollo orientado a objetos, al estar basado
todo el proyecto en un entorno de desarrollo bajo el paradigma
de programación orientada a objetos.\\

El lenguaje de programación que usaremos será \cursiva{Ruby}, que es libre así
como el \cursiva{framework} sobre el que apoyaremos nuestra
aplicación web \cursiva{RoR}.\\

No se deben tener en cuenta restricciones en el uso de memoria principal por
parte de la aplicación tanto en el lado servidor como en
el cliente.\\

Se deben tener en cuenta restricciones a la hora de crear las vistas de nuestra
aplicación web para que éstas sean lo mas livianas posibles
para ahorrar el mayor ancho de banda posible.\\

Para el desarrollo de todo el proyecto deberemos usar un \cursiva{IDE} que se
base en la filosofía de \cursiva{software libre}, en este caso
\cursiva{NetBeans}.\\

El producto habrá de funcionar en cualquier sistema operativo que tenga
disponibilidad para tener instalado un intérprete de \cursiva{Ruby} en el lado
servidor y en cualquier navegador del mercado compatible con los estándares web
para \cursiva{HTML 4.0}, \cursiva{CSS 2.0}, y la librería \cursiva{JQuery}. Esto
hará que
nuestro software tenga un mayor público objetivo de desarrolladores.\\

Como sistema gestor de bases de datos se podrá utilizar cualquiera que tenga una
\cursiva{gema} que sea capaz de conectar \cursiva{RoR} con
el \cursiva{SGBD} escogido.\\

Es necesario tener en cuenta que nuestra aplicación puede correr en cualquier
\cursiva{pc} pero es conveniente que éste tenga la capacidad suficiente para
atender varias peticiones a la vez, además de tener la configuración necesaria
para que peticiones que lleguen desde
\cursiva{internet} puedan ser atendidas.\\

Para toda la creación y generación de la documentación presente hemos
utilizado LaTeX\\

Como controlador de la configuración hemos optado por usar \cursiva{GIT}, un
sistema de control de versiones distribuido que es el usado oficialmente por el
proyecto \cursiva{RoR} y que se integra de forma plena con \cursiva{NetBeans}.

\section{Requisitos no funcionales del sistema}
\subsection{Requisitos Generales}
La solución debe cumplir como mínimo con las siguientes características basadas
en las especificaciones funcionales y los requisitos no funcionales:

\begin{itemize}
\item Basada en la web.
\item La aplicación debe esta diseñada y desarrollada sobre la plataforma
  \cursiva{Ruby On Rails}.
\item La aplicación debe ser escalable bajo estrategia vertical (Añadir más
  recursos al servidor) y horizontal (Añadir más servidores), según las
  necesidades de procesamiento.
\item La aplicación debe tener bajo nivel de acoplamiento y la posibilidad de
  editar fácilmente los parámetros que se consideren dinámicos y requieran
  cambios frecuentes.
\item Orientada a objetos.
\item De fácil mantenimiento en cuanto a cumplimiento de estándares, uso de
  guías y patrones con especial énfasis en el patrón \cursiva{MVC},
  documentación y de fácil ubicación de componentes.
\item Que permita y utilice la reutilización de código.
\item Basada en la arquitectura cliente-servidor.
\item La aplicación debe permitir generar avisos a través de correo electrónico.
\end{itemize}

\subsection{Requisitos específicos}
Los requisitos no funcionales generales estarán enmarcados en los siguientes
aspectos:
\begin{itemize}
\item Escalabilidad:
  \begin{itemize}
  \item El diseño debe contemplar el uso óptimo de recursos como las conexiones
    con la base de datos.
  \item El diseño debe contemplar la clara división entre datos, recursos, y
    aplicaciones para optimizar la escalabilidad del sistema.
  \item Debe contemplar requisitos de crecimiento para usuarios internos al
    sistema.
  \end{itemize}
\item Disponibilidad:
  \begin{itemize}
  \item La disponibilidad del sistema ha de ser continua con un nivel de
    servicio para los usuarios de siete días y 24 horas.
  \item En caso de fallos de algún subsistema no debe haber pérdida de
    información
  \item Debe contemplar requisitos de consistencia transaccional.
  \end{itemize}
\item Seguridad:
  \begin{itemize}
  \item La aplicación debe reflejar patrones de seguridad teniendo para cumplir
    con la ley orgánica de protección de datos.
  \end{itemize}
\item Mantenibilidad:
  \begin{itemize}
  \item Se debe estructura el código de manera consistente y predecible.
  \item Aprovechar al máximo las facilidades que nos aporta el
    \cursiva{framework Ruby On Rails}, con especial énfasis en el patrón
    \cursiva{MVC}.
  \end{itemize}
\item Rendimiento: La aplicación debe ofrecer una buena respuesta ante alta
  demanda de usuarios.
\end{itemize}
\section{Requisitos para futuras versiones}
Para futuras versiones del producto, podrían introducirse funciones avanzadas
para los usuarios como selección de escudo del equipo, personalización de
dorsales, avisos por sms, mercados de agentes
libres, cantera y ojeadores.\\

Podría introducirse nuevas plantillas de diseño para personalizar la vista de
diversas formas para que el usuario escoja la que más le
guste.\\

También sería interesante establecer un apartado de estadísticas para poder
llevar un seguimiento del historial en tiempo de la evolución de los equipos
tanto en el plano técnico como en el económico, este apartado también sería
aplicable a los futbolistas que integran los
clubes.\\

Podría añadirse un sistema de foros y chat para una interacción entre usuarios
que no se limite en exclusiva a las funcionalidades
específicas de nuestro sistema.\\

Como restricción a futuras versiones o proyectos que estén basados en nuestra
aplicación se aplicarán las referentes a las propias especificadas en el
apartado de licencia de este documento.
