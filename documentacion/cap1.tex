% -*-cap1.tex-*- Este fichero es parte de la plantilla LaTeX para la realización
% de Proyectos Final de Carrera, protejido bajo los términos de la licencia
% GFDL.  Para más información, la licencia completa viene incluida en el fichero
% fdl-1.3.tex

% Copyright (C) 2009 Pablo Recio Quijano

\section{Introducción}

Este documento describe el proceso que hemos seguido para la realización del
proyecto.  Comenzamos con la recopilación de requisitos obteniendo al tinal un
documento de Especifacación de Requisitos del Software, realizada a través de
las sucesivas entrevistas con los tutores del proyecto de fin de carrera.

Posteriormente seleccionamos la tecnología más adecuada para llevar a cabo el
proyecto. A continuación pasamos a desarrollar los componentes del software que
añadiremos al sistema para que cumpla con los requisitos que hemos recopilado
previamente.

Para finalizar, seguiremos un desarollo software ágil, por el cual vamos
desarollando partes del sistema de forma que sean totalmente funcionales una vez
terminadas hasta conformar el proyecto según las especificaciones que se han
recogido.

\section{Objetivos}

Los objetivos de este proyecto son por una parte poner en práctica los
conocimientos adquiridos a lo largo de la formación académica recibida en la
titulación de \cursiva{Ingeniería Técnica en Informática de Gestión} y a la vez
aprender a desarrollar bajo una novedosa tecnología web basada en un
\programa{framework} basado en el patrón Modelo-Vista-Controlador (\sigla{MVC})
denominado \programa{Ruby On Rails} (\sigla{MVC}).

Este \programa{framework}, a su vez, basa su simplicidad en el uso de un
lenguaje de \programa{script} llamado \programa{Ruby} que tiene características
propias de lenguajes funcionales; además, es uno de los exponentes de una
perspectiva de desarrollo en auge que se engloba dentro de las denominadas
metodologías ágiles.

El resultado de este proyecto es el desarrollo de una aplicación informática
\programa{web} de código abierto para realizar la simulación virtual de la
\cursiva{gestión computerizada} de un club de fútbol, la simulación de los
partidos que se jueguen, además de la interacción con otros usuarios. En
definitiva: un videojuego multijugador con interfaz web.

Pasamos a enumerar los objetivos de forma esquemática:

\begin{itemize}
\item \negrita{Gestión de usuarios:} todas las herramientas necesarias para
  identificar a cada usuario de nuestra aplicación web de forma unívoca.
\item \negrita{Gestión de clubes: } todas las herramientas necesarias para
  gestionar precios de entrada, venta, compra, entrenamiento de futbolistas y
  tipos de entrenamientos.
\item \negrita{Gestión de campeonatos: } todas las herramientas necesarias para
  que el administrador pueda controlar el calendario de competición, y la
  creación es de nuevas ligas en función de los usuarios que se vayan
  registrando en la web.
\item \negrita{Simulación de partidos: } desarrollaremos un simulador que
  mediante diferentes ponderaciones sea capaz de simular la realidad de un
  partido de fútbol 11.
\end{itemize}

\section{Alcance}
Se desea informatizar toda la gestión y simulación de un entorno de competición
de fútbol 11 de forma que acabemos obteniendo un videojuego multijugador
interactivo con interfaz web.

\subsection{Identificación del producto mediante un nombre}
El producto realizado se llama ``Football Simulator''.

A lo largo de toda la memoria nos referiremos a él como
\negrita{\cursiva{Simulador de fútbol online}}.

\subsection{Funcionalidades del producto}
El producto que deseamos obtener tal y como hemos mencionado anteriormente es
capaz de simular la gestión, interacción y desarrollo de una competición de liga
de fútbol 11, proporcionando funcionalidades de interacción entre usuarios y del
administrador para controlar el comportamiento de la competición.

Por otra parte, nuestro producto también es capaz de simular mediante diversas
ponderaciones los resultados de un partido de fútbol.

Para disfrutar de las funcionalidades de nuestro producto es necesario el
registro y participación de varios usuarios en él, además de la interacción de
un administrador que llevará a cabo acciones de control, supervisión y
seguimiento del desarrollo de la competición.

\subsection{Aplicaciones del software: beneficios, objetivos y metas}

Con este producto podemos obtener diversas ventajas ya descritas anteriormente,
entre ellas cabe destacar la gestión de los equipos de fútbol y la interacción
de usuarios.

El principal objetivo de la utilización del producto es divertir y entretener al
usuario, mediante la inmersión en un entorno virtual de gestión de clubes con
diversos usuarios que compiten entre sí para alcanzar la posición más alta en
clasificación y división que puedan.

Al fin y al cabo lo que se desea es formar la comunidad más grande posible y
para ello es necesario hacer que el usuario desee acceder asiduamente a nuestra
aplicación.

La mayor ventaja es que al ser una aplicación web, cualquier usuario puede
registrarse sin ningún tipo de restricción.

\section{Definiciones, acrónimos y abreviaturas}

\begin{itemize}
\item \negrita{Framework: } Estructura conceptual y tecnológica de soporte
  definida, normalmente con artefactos o módulos de sofware concretos, con base
  en la cual otro proyecto software puede ser organizado y desarrollado.
\item \negrita{Patrón Modelo Vista Controlador (MVC): } Es un estilo de
  arquitectura software que separa los datos de una aplicación, la interfaz de
  usuario y la lógica de control en tres componentes distintos.
\item \negrita{COC (Convenction over Configuration): } En español significa
  \cursiva{``Convención sobre configuración''}, quiere decir que toda pieza de
  código tiene su lugar y todas ellas interactúan según un camino estándar.
\item \negrita{DRY (Don't Repeat Yourself): } En español significa \cursiva{``No
    te repitas''}. Se refiera a que es innecesario repetir código si éste está
  bien diseñado y refactorizado.
\item \negrita{Programación Orientada a Objetos (POO): } . Es un paradigma de la
  programación que se basa en la utilización de clases de objetos.
\item \negrita{Simulador de Fútbol Online:} es el nombre utilizado comúnmente
  durante todo el documento para hacer referencia al nombre del proyecto.
\item \negrita{Ruby: } Lenguaje de programación orientado a objetos de script
  con características propias de lenguajes funcionales.
\item \negrita{Gema: }
\item \negrita{Ruby On Rails (RoR): } Framework de desarrollo rápido de
  aplicaciones web basado en el lenguaje de programación \programa{Ruby}.
\item \negrita{GIT: } Sistema de control de la configuración distribuido.
\item \negrita{Integrated Development Environment (IDE): } En español significa
  \cursiva{``Entorno integrado de desarrollo''}. Es un programa informático
  compuesto por un conjunto de herramientas y módulos de programación que ayudan
  al desarrollo de productos software.
\item \negrita{NetBeans: } IDE con funcionalidades que ayudan al desarrollo de
  aplicaciones \programa{Ruby on Rails}
\item \negrita{Sistema de Gestión de Base de Datos (SGBD)}:
\item \negrita{Oracle: } Es un sistema de Gestión de Base de Datos privativo.
\item \negrita{MySQL: } Sistema de gestión de base de datos de software libre.
\item \negrita{SQLite2: } Sistema de gestión de base de datos de software libre.
\item \negrita{SQLite3: } Sistema de gestión de base de datos de software libre.
\item \negrita{Postgre: } Sistema de gestión de base de datos de software libre.
\item \negrita{Cascade Style Sheet (CSS):} En español significa \cursiva{``Hojas
    de estilo en cascada''}.
\item \negrita{Hypertext Markup Language (HTML))} En español significa
  \cursiva{``Lenguaje de marcado de hipertexto''}.
\item \negrita{JavaScript: } Lenguaje de script que se ejecuta en el lado
  cliente.
\item \negrita{JQuery: } Librería de desarrollo rápido JavaScript.
\item \negrita{Algoritmo criptográfico: } En computación y criptografía un
  algoritmo criptográfico es un algoritmo que modifica los datos de un documento
  con el objeto de alcanzar algunas características de seguridad como
  autenticación, integridad y confidencialidad.
\item \negrita{Algoritmo MD5: } En criptografía, MD5 (abreviatura de
  Message-Digest Algorithm 5, Algoritmo de Resumen del Mensaje 5) es un
  \cursiva{algoritmo criptográfico} de 128 bits ampliamente usado.
\item \negrita{Log:} Archivo de texto plano que nos indica las todas las
  acciones que se realizan en nuestra aplicación con la finalidad de conseguir
  auditar el uso de la misma.


\end{itemize}

\section{Visión General}
Esta memoria sigue las pautas descritas en el documento de referencia
\cursiva{``Recomendaciones para la documentación del Proyecto de fin de
  Carrera''} realizada por varios profesores del departamento de lenguajes y
sistemas informáticos.

Las partes que componen el documento son:

\begin{enumerate}
\item \negrita{Introducción: } Visión general. Objetivos, alcance y estructura.
\item \negrita{Planificación: } Planificación temporal del desarrollo del
  proyecto y porcentaje de esfuerzo dedicado a las diversas tareas y fases que
  lo forman.
\item \negrita{Descripción general del proyecto: } Ampliación de la visión
  global del proyecto. Perspectiva, funciones y restricciones.
\item \negrita{Análisis: } Identificación de las metas globales, perspectivas
  del cliente y recogida de información necesaria para llevar a cabo el
  proyecto.
\item \negrita{Diseño: } Aplicamos diferentes técnicas y procedimientos para
  obtener un producto con el suficiente detalle que permita su realización
  física.
\item \negrita{Implementación: } Aspectos más relevantes de la implementación de
  la aplicación software.
\item \negrita{Pruebas: } Todo lo concerniente a las pruebas para que la
  aplicación logre su cometido sin fallos.
\item \negrita{Resumen: } Breve resumen de lo más destacable del proyecto
  realizado.
\item \negrita{Conclusiones y trabajo futuro: } Valoración global, posibles mejoras y
  ampliaciones del proyecto.
\item \negrita{Apéndices: }
  \begin{itemize}
  \item \negrita{Manual de usuario: } Manual del funcionamiento de la aplicación
    destinado a los usuarios de la misma.
  \item \negrita{Manual de instalación y puesta en funcionamiento: }
    Instrucciones para realizar la instalación de la aplicación, así como su
    puesta en funcionamiento.
  \item \negrita{Difusión: } Lugares dónde se ha publicado el proyecto.
  \item \negrita{Licencia: } Texto de la licencia adoptada para la divulgación
    del proyecto.
  \end{itemize}
\item \negrita{Bibliografía: } Libros y referencias físicas y electrónicas
  consultadas para la elaboración del proyecto.
\end{enumerate}
