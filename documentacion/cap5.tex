% -*-cap5.tex-*- Este fichero es parte de la plantilla LaTeX para la realización
% de Proyectos Final de Carrera, protejido bajo los términos de la licencia
% GFDL.  Para más información, la licencia completa viene incluida en el fichero
% fdl-1.3.tex Copyright (C) 2009 Pablo Recio Quijano

Para la implementación de toda la funcionalidad descrita en el capítulo anterior
seguiremos las convenciones que el framework \cursiva{Ruby on Rails} nos
proporciona para una mayor velocidad y corrección en el desarrollo.

\section{Arquitectura}

\label{pattern:mvc}
Debido a \cursiva{Ruby on Rails} es un framework que hace un uso intensivo del
patrón \cursiva{Modelo-Vista-Controlador} (MVC) \cite{agilerails}.

El patrón \cursiva{Modelo-Vista-Controlador} divide el software en tres capas
bien diferenciadas que se adaptan muy bien a la tecnología
Cliente-Servidor. Dichas capas son:

\begin{itemize}
\item \negrita{Modelo}: Esta capa es la relativa a la representación de la
  información con la que el proyecto interactuará. Está compuesto por las clases
  necesarias para el acceso, modificación, búsqueda y demás operaciones con los
  datos.
\item \negrita{Vista}: En esta capa agruparemos todas las clases necesarias para
  la representación de la interfaz con la que nuestros usuarios interactuarán.
\item \negrita{Controlador}: Será la capa que nos permita gestionar las
  peticiones que lleguen a nuestro servidor.
\end{itemize}

\subsection{Capa Modelo}

Para la capa de modelo y la consecuente persistencia de datos vamos a usar el
\cursiva{ORM} (mapeo objeto-relacional) que nos proporciona por defecto
\cursiva{Ruby On Rails}, la clase \cursiva{ActiveRecord} que implementa el
patrón \cursiva{Active Record} pudiéndonos abstraer del \cursiva{SGBD}
relacional que usemos y las especificaciones únicas del lenguaje \cursiva{SQL}
que implemente el mismo.

\subsection{Capa Vista}

Para esta capa vamos a utilizar la clase \cursiva{ActionView}. Dicha clase es
una forma poderosa, sencilla y elegante, de embeber código \cursiva{Ruby} dentro
de cualquier tipo de fichero de la capa de presentación de nuestra página web
(\cursiva{JavaScript}, \cursiva{CSS}).

Además, haremos uso de los frameworks \cursiva{CoffeScript}
\cite{lang:coffescript} y \cursiva{SASS} \cite{lang:sass} para el desarrollo de
los ficheros \cursiva{JavaScript} y \cursiva{CSS} respectivamente.

\subsection{Capa Controlador}
En esta capa haremos uso de la clase \cursiva{ApplicationController}. Esta clase
nos ayudará a modelar de forma sencilla y fácil el enrutador de nuestra
aplicación y las respuestas que se hagan a través del cliente en la petición
web.

Además, nos ayudará a establecer un flujo sencillo entre nuestras capas Vista y
Modelo consiguiendo un menor acoplamiento y una mayor cohesión de nuestro
sistema.

\subsection{Active Record}
Se basa en el patrón de mismo nombre el cual es un patrón de arquitectura que
podemos encontrar en software que almacene información en bases de datos
relacionales. Conforme a este patrón, un objeto que lo use habrá de integrar
funciones para insertar, actualizar, eliminar y otras que correspondan en mayor
o menor medida a las operaciones con columnas de la tabla de una base de datos.

Este patrón crea una interfaz que envuelve una tabla de una base de datos o una
vista en una clase. En base a ello, un objeto de la clase corresponde a una fila
en la tabla. La clase que envuelve a la tabla o vista implementa métodos o
propiedades para acceder a cada columna.

Este patrón es normalmente usado para dar persistencia a los datos en un
proyecto software, haciendo de \cursiva{ORM} para la aplicación. Normalmente,
las relaciones de clave foránea entre clases son implementadas como una
propiedad del objeto \cite{pattern:activerecord}.

Las funcionalidades que implementa la clase \cursiva{Active Record} en
\cursiva{Rails} son:

\begin{itemize}
\item \negrita{Creación}: Permite la creación de nuevas líneas en la base de
  datos.
\item \negrita{Condiciones}: Permite consultar datos en la base de datos usando
  para ello cadenas de texto o \cursiva{hashes} que representen la parte
  \cursiva{WHERE} en una sentencia SQL.
\item \negrita{Sobrescribir métodos de accesibilidad por defecto}: Se pueden
  sobrescribir los métodos de accesibilidad para modificar el comportamiento en
  las operaciones de escritura, creación, lectura, y demás operaciones de
  lectura/escritura.
\item \negrita{Métodos de consulta como atributos}: Se pueden crear nuevos
  métodos que se comporten como atributos para conocer si un atributo dentro de
  la tabla está inicializado.
\item \negrita{Posibilidad de acceder a los atributos antes de que hayan pasado
    el casting de tipo}: Es una funcionalidad orientada sobre todo en las
  situaciones de validación.
\item \negrita{Buscadores dinámicos en base al atributo}: Métodos estándar
  creados en base al nombre de los atributos correspondientes al nombre de la
  columna dentro de la tabla.
\item \negrita{Serializar vectores, hashes y otros objetos en columnas de
    texto}: Es posible guardar este tipo de objetos en formato texto usando el
  lenguaje \cursiva{YAML}.
\item \negrita{Herencia simple de tabla}: Es posible el uso de la herencia
  simple entre tablas.
\item \negrita{Conexión a múltiples bases de datos en diferentes modelos}: Es
  posible establecer diferentes conexiones a diferentes \cursiva{SGBD} para
  diferentes clases.
\end{itemize}

\subsection{Action View}
\cursiva{Action View} hace uso del sistema de plantillas \cursiva{eRuby} para
embeber código \cursiva{Ruby} dentro de un documento de texto. Es ampliamente
utilizado para dentro de documentos \cursiva{HTML} y \cursiva{JavaScript}.

\subsection{Action Controller}
Es el núcleo de una petición web en \cursiva{Ruby on Rails}. Está formado por
una o más acciones que son ejecutadas cuando se produzca una petición, justo
después se puede redireccionar a otra acción o renderizar una plantilla. Una
acción se define como un método público dentro del controlador, el cuál es
automáticamente accesible al servidor web a través del enrutador de
\cursiva{Rails}.

Por defecto, sólo la clase \cursiva{ApplicationController} dentro de una
aplicación \cursiva{Rails} hereda de la clase
\cursiva{ActionController:Base}. Los demás controladores heredarán de la clase
\cursiva{ApplicationController}. Este tipo de estructura nos proporciona una
clase para configurar varias cosas que sean comunes a cualquier petición web.

Esta clase nos proporciona además las siguientes funcionalidades:

\begin{itemize}
\item \negrita{Peticiones}: El objeto de petición es totalmente accesible y se
  usa principalmente para conocer la cabecera de la petición \cursiva{HTTP}.
\item \negrita{Parámetros}: Todos los parámetros que lleva la petición, ya sean
  desde \cursiva{GET} o \cursiva{POST} o de la \cursiva{URL}, son accesibles a
  través del \cursiva{método} \negrita{params} que devolverá un \cursiva{hash}
  uni o multidimensional con todos los parámetros asociados a la petición.
\item \negrita{Sesiones}: Las sesiones permiten almacenar objetos entre petición
  y petición. Son bastante útiles para objetos que no estén aún preparados para
  pasar a ser almacenados en la base de datos. Para acceder a las sesiones
  bastará con llamar al método \negrita{session} y este devolverá un
  \cursiva{hash} uni o multidimensional con las sesiones almacenadas.
\item \negrita{Respuestas}: Toda acción lleva asociada una respuesta, la cual
  alberga las cabeceras y el documento para ser mandado al navegador del
  usuario. El objeto respuesta es generado automáticamente a través del uso de
  \cursiva{renders} y redirecciones, con lo que no necesita la interacción del
  usuario.
\item \negrita{Renderizaciones}: La clase \cursiva{ActionController} manda
  contenido al usuario a través del uso de métodos de renderización. El más
  común es el uso de una plantilla para la renderización del contenido, a través
  de plantillas \cursiva{ERB}.
\item \negrita{Redirecciones}: Son usadas para moverse de una acción a otra.
\item \negrita{Redirecciones y renderizaciones múltiples}: Es posible renderizar
  y redireccionar varias veces en la misma petición.
\end{itemize}

\subsection{Diagrama de componentes y estructura}

Podemos ver el diagrama de componentes en la figura \ref{diag:componentes}.

\figura{mvc-rails.png}{scale=0.50}{Diagrama de componentes}{diag:componentes}{H}

El framework \cursiva{Ruby on Rails} hace especial hincapié en tres conceptos
muy básicos que caracterizan toda su estructura, que son:

\begin{itemize}
\item \negrita{Convention over Configuration (COC)}: Convención frente a
  configuración.
\item \negrita{Don't Repeat Yourself (DRY)}: No te repitas
\item \negrita{Keep It Simple (KIS)}: Mantenlo simple.
\end{itemize}

Gracias a ello seguiremos la estructura de directorios en árbol para nuestro
código fuente que genera \cursiva{Rails} para una aplicación vacía.

\begin{itemize}
\item \texttt{/pfc\_sfo/} raíz del proyecto \cursiva{Rails}.
\item \texttt{/pfc\_sfo/app/} ficheros fuente principales.
\item \texttt{/pfc\_sfo/app/controllers/} ficheros fuente para los
  controladores.
\item \texttt{/pfc\_sfo/app/helpers/} ficheros fuente para los métodos de ayuda
  para refactorización de códigos en las vistas.
\item \texttt{/pfc\_sfo/app/mailers/} ficheros fuente para el código fuente de
  los correos que se envían desde el servidor.
\item \texttt{/pfc\_sfo/app/models/} ficheros fuente correspondientes a los
  modelos de datos.
\item \texttt{/pfc\_sfo/app/views/} ficheros fuente para las vistas de la
  aplicación, incluyendo como parte de las mismas las de los correos.
\item \texttt{/pfc\_sfo/app/assets/} ficheros fuente que han de ejecutarse en el
  navegador, archivos \cursiva{CSS}, \cursiva{JavasScript}, imágenes, etc...
\item \texttt{/pfc\_sfo/config/}, ficheros fuente de configuración de la
  aplicación web.
\item \texttt{/pfc\_sfo/config/environments/} ficheros fuente de configuración
  de los distintos entornos, desarrollo, test y producción.
\item \texttt{/pfc\_sfo/config/initializers/} ficheros de inicialización del
  servidor.
\item \texttt{/pfc\_sfo/config/locales/} ficheros \cursiva{YAML} para los textos
  de traducción en diferentes idiomas.
\item \texttt{/pfc\_sfo/db/} ficheros autogenerados dónde se albergan la
  configuración de la Base de Datos.
\item \texttt{/pfc\_sfo/db/migrate/} ficheros para la manipulación de las tablas
  de la base de datos.
\item \texttt{/pfc\_sfo/lib/} ficheros para la extensión de la aplicación con
  otros subsistemas software.
\item \texttt{/pfc\_sfo/log/} ficheros log del servidor.
\item \texttt{/pfc\_sfo/public/} ficheros de acceso público a través del
  enrutador.
\item \texttt{/pfc\_sfo/script/} scripts para la generación automática de código
  y la automatización de procesos.
\item \texttt{/pfc\_sfo/test/} ficheros para almacenar test automáticos.
\end{itemize}

\subsection{Entorno de ejecución}
En la figura \ref{fig:entornoejecucion} podemos ver el diagrama de despliegue de
la aplicación puesta en marcha en cualquiera de los entornos, ya sea este de
desarrollo, test o producción.

\figura{diagrama_despliegue.png}{scale=0.53}{Diagrama de
  estructura}{fig:entornoejecucion}{H}

El sistema (servidor) interactuará con uno o más clientes web (navegadores). El
servidor, a su vez estará formado por un servidor de aplicaciones web, en
nuestro caso \cursiva{Webrick}, y el sistema de gestión de base de datos será
\cursiva{SQLite3}.

Gracias a esta sencilla configuración no tendremos que configurar ninguna cuenta
ni servicio en nuestro servidor web.

\clearpage
\section{Diagramas de clases conceptuales}
En la figura \ref{diag:clases_modelo} podemos observar el diagrama de clases de
la capa modelo de nuestra aplicación según la recolección de requisitos que
hemos obtenido en el Capítulo \ref{cap:analisis} de esta memoria. Dichos
diagramas serán realizados en notación \cursiva{UML}
\cite{uml:distilled_standard}.

\figura{models_complete.png}{scale=0.25}{Diagrama de clases UML de la capa
  modelo}{diag:clases_modelo}{H}

\clearpage

En la figura \ref{diag:clases_controlador} podemos observar el diagramas de
clases de la capa controlador de nuestra aplicación según la recolección de
requisitos que hemos obtenido en el Capítulo \ref{cap:analisis}.

\figura{controllers_complete.png}{scale=0.22,angle=270}{Diagrama de Clases UML
  de la capa controlador}{diag:clases_controlador}{H}

\section{Diagramas de secuencia de sistemas}

En las siguientes secciones podremos observar los diagramas de secuencias del
sistema software que hemos desarrollado.

\subsection*{Diagrama de secuencia de crear cuenta de usuario}

En la figura \ref{diagrama-secuencia-crear-cuenta-usuario} podemos observar como
se construye un nuevo usuario dentro del sistema con los datos proporcionados
por el usuario web.

\figura{diagramas_secuencia/crear_cuenta_usuario.png}{scale=0.42}{Diagrama de
  secuencia de crear cuenta de
  usuario}{diagrama-secuencia-crear-cuenta-usuario}{H} \newpage

\subsection*{Diagrama de secuencia de acceder a cuenta de usuario}

En la figura \ref{diagrama-secuencia-acceder-cuenta-usuario} podemos observar
como se consigue acreditar las credenciales (usuario y contraseña) para iniciar
sesión.

\figura{diagramas_secuencia/acceder_cuenta_usuario.png}{scale=0.49}{Diagrama de
  secuencia de acceder a cuenta de
  usuario}{diagrama-secuencia-acceder-cuenta-usuario}{H} \newpage

\subsection*{Diagrama de secuencia de abandonar cuenta de usuario}

En la figura \ref{diagrama-secuencia-abandonar-cuenta-usuario} podemos observar
cómo se destruye la sesión iniciada por un usuario.

\figura{diagramas_secuencia/abandona_cuenta_usuario.png}{scale=0.59}{Diagrama de
  secuencia de abandonar cuenta de
  usuario}{diagrama-secuencia-abandonar-cuenta-usuario}{H} \newpage

\subsection*{Diagrama de secuencia de abrir equipos}

En la figura \ref{diagrama-secuencia-abrir-equipos} cómo se establece el estado
de las temporadas para permitir operaciones por parte de los usuarios en los
equipos que poseen.

\figura{diagramas_secuencia/abrir_equipos.png}{scale=0.78}{Diagrama de secuencia
  de abrir equipos}{diagrama-secuencia-abrir-equipos}{H}

\subsection*{Diagrama de secuencia de cerrar equipos}
En la figura \ref{diagrama-secuencia-abrir-equipos} podemos observar cómo se
establece el estado de las temporadas para impedir que los usuarios puedan
realizar operaciones con sus equipos.

Este diagrama es igual al \ref{diagrama-secuencia-abrir-equipos} pero con la
ruta \cursiva{/leagues/close\_teams} y la llamada al modelo con el método
\cursiva{close}.


\subsection*{Diagrama de secuencia de notificación de registro por e-mail}
En la figura \ref{diagrama-secuencia-notificacion-registro} se observa cómo se
envía un correo al usuario una vez se ha registrado en la aplicación web.

\figura{diagramas_secuencia/crear_cuenta_usuario.png}{scale=0.43}{Diagrama de
  secuencia de notificación de registro por
  e-mail}{diagrama-secuencia-notificacion-registro}{H}


\subsection*{Diagrama de secuencia de listar ofertas emitidas}
En la figura \ref{diagrama-secuencia-listar-ofertas-emitidas} observamos cómo se
establece el filtro de búsqueda para mostrar el listado de ofertas emitidas.

\figura{diagramas_secuencia/listar_ofertas_emitidas.png}{scale=0.52}{Diagrama de
  secuencia de listar ofertas
  emitidas}{diagrama-secuencia-listar-ofertas-emitidas}{H}

\subsection*{Diagrama de secuencia de listar ofertas recibidas}
En la figura \ref{diagrama-secuencia-listar-ofertas-emitidas} observamos cómo se
establece el filtro de búsqueda para mostrar el listado de ofertas recibidas.

Este diagrama es igual al \ref{diagrama-secuencia-listar-ofertas-emitidas} pero
con la ruta \cursiva{/clubs/*/received\_offers} y la llamada al modelo club con
el método \cursiva{offers\_as\_seller}.

\subsection*{Diagrama de secuencia de emitir oferta por futbolista}
En la figura \ref{diagrama-secuencia-emitir-oferta-futbolista} se observa cómo
se crea una nueva instancia de oferta y se relaciona con los usuarios y el
futbolista involucrado en la misma.

\figura{diagramas_secuencia/emitir_oferta_futbolista.png}{scale=0.45}{Diagrama
  de secuencia de emitir oferta por
  futbolista}{diagrama-secuencia-emitir-oferta-futbolista}{H}

\subsection*{Diagrama de secuencia de cancelar oferta emitida por futbolista}
En la figura \ref{diagrama-secuencia-cancelar-oferta-emitida-futbolista} se
observa cómo se recupera la instancia de la oferta emitida y se modifica su
estado para cancelarla.

\figura{diagramas_secuencia/cancelar_oferta_emitida_futbolista.png}{scale=0.68}{Diagrama
  de secuencia de cancelar oferta emitida por
  futbolista}{diagrama-secuencia-cancelar-oferta-emitida-futbolista}{H}

\subsection*{Diagrama de secuencia de rechazar oferta recibida por futbolista}
En la figura \ref{diagrama-secuencia-rechazar-oferta-recibida-futbolista} se
observa cómo se recupera la instancia de la oferta recibida y se modifica su
estado para rechazar la misma.

\figura{diagramas_secuencia/rechazar_oferta_recibida_futbolista.png}{scale=0.68}{Diagrama
  de secuencia de rechazar oferta recibida por
  futbolista}{diagrama-secuencia-rechazar-oferta-recibida-futbolista}{H}

\newpage

\subsection*{Diagrama de secuencia de aceptar oferta recibida por futbolista}
En la figura \ref{diagrama-secuencia-aceptar-oferta-recibida-futbolista} se
observa cómo se recupera la instancia de la oferta recibida y se actualizan los
modelos de datos involucrados (futbolista, club y usuarios) para llevar a cabo
la transferencia del futbolista entre los clubes.

\figura{diagramas_secuencia/aceptar_oferta_recibida_futbolista.png}{scale=0.65}{Diagrama
  de secuencia de aceptar oferta recibida por
  futbolista}{diagrama-secuencia-aceptar-oferta-recibida-futbolista}{H} \newpage

\subsection*{Diagrama de secuencia de renovación de futbolistas}
En la figura \ref{diagrama-secuencia-renovacion-futbolistas} se describe cómo se
renuevan automáticamente los futbolistas para las siguientes temporadas
actualizando sus atributos.

\figura{diagramas_secuencia/renovacion_futbolistas.png}{scale=0.63}{Diagrama de
  secuencia de renovación de
  futbolistas}{diagrama-secuencia-renovacion-futbolistas}{H}

El proceso de renovación está integrado en start.  \newpage

\subsection*{Diagrama de secuencia de fijar precio de venta de entradas}
En la figura \ref{diagrama-secuencia-fijar-precio-venta-entradas} se describe
cómo se modifica el atributo del precio de venta de entradas.

\figura{diagramas_secuencia/fijar_precio_venta_entradas.png}{scale=0.54}{Diagrama
  de secuencia de fijar precio de venta de
  entradas}{diagrama-secuencia-fijar-precio-venta-entradas}{H}

\subsection*{Diagrama de secuencia de planificación de alineaciones}
En la figura \ref{diagrama-secuencia-planificacion-alineaciones} se describe
cómo se cambian las alineaciones para el próximo partido a través de la
modificación del atributo de posición.

\figura{diagramas_secuencia/planificacion_alineaciones.png}{scale=0.52}{Diagrama
  de secuencia de emitir planificación de
  alineaciones}{diagrama-secuencia-planificacion-alineaciones}{H}

\subsection*{Diagrama de secuencia de planificación de tácticas}
En la figura \ref{diagrama-secuencia-planificacion-tacticas} se muestra cómo se
cambia la táctica usada para el próximo partido a través de la modificación del
atributo de táctica del club.

\figura{diagramas_secuencia/planificacion_tacticas.png}{scale=0.70}{Diagrama de
  secuencia de planificación de tácticas
}{diagrama-secuencia-planificacion-tacticas}{H}

\subsection*{Diagrama de secuencia de planificación de entrenamientos}
En la figura \ref{diagrama-secuencia-planificacion-entrenamientos} se muestra
cómo se crean nuevas instancias para la planificación de los entrenamientos de
los futbolistas asociados a un club.

\figura{diagramas_secuencia/planificacion_entrenamientos.png}{scale=0.45}{Diagrama
  de secuencia de planificación de entrenamientos
}{diagrama-secuencia-planificacion-entrenamientos}{H}

\subsection*{Diagrama de secuencia de asignar club}
En la figura \ref{diagrama-secuencia-asignar-club} se observa cómo se crea una
nueva instancia de un club para todo usuario que no tenga club asignado.

\figura{diagramas_secuencia/asignar_club.png}{scale=0.45}{Diagrama de secuencia
  de secuencia de asignar club }{diagrama-secuencia-asignar-club}{H}

\subsection*{Diagrama de secuencia de simular jornadas}
En la figura \ref{diagrama-secuencia-simular-jornadas} se observa cómo se crean
nuevas instancias de detalles de partidos por cada uno de los partidos no
jugados y pertenecientes a la jornada actual del sistema.

\figura{diagramas_secuencia/simular_jornadas.png}{scale=0.35,
  angle=270}{Diagrama de secuencia de simular
  jornadas}{diagrama-secuencia-simular-jornadas}{H}

\subsection*{Diagrama de secuencia de comenzar jornadas}
En la figura \ref{diagrama-secuencia-comenzar-jornadas} se observa cómo se
actualiza el atributo correspondiente de todos los partidos actuales para
estipular que el partido ha comenzado a la fecha actual del sistema.  \newpage

\figura{diagramas_secuencia/comenzar_jornadas.png}{scale=0.49,
  angle=270}{Diagrama de secuencia de comenzar
  jornadas}{diagrama-secuencia-comenzar-jornadas}{H}


\subsection*{Diagrama de secuencia de pasar a siguiente jornada}
En la figura \ref{diagrama-secuencia-pasar-siguiente-jornada} se detalla cómo se
actualizan los modelos de datos de temporada, jornada, jugador, entrenamiento,
finanzas y club para avanzar en la competición.  \newpage
 
\figura{diagramas_secuencia/pasar_siguiente_jornada.png}{scale=0.29,
  angle=270}{Diagrama de secuencia de pasar a siguiente
  jornada}{diagrama-secuencia-pasar-siguiente-jornada}{H}

\subsection*{Diagrama de secuencia de promoción y descenso de clubes}
En la figura \ref{diagrama-secuencia-promocion-descenso-clubes} se observa cómo
se crean nuevas instancias de nuevas temporadas y se modifican los estados de
las temporadas actuales para simular el avance en la competición y el progreso y
descenso jerárquico en las divisiones del juego.

\figura{diagramas_secuencia/promocion_descenso_clubes.png}{scale=0.38}{Diagrama
  de secuencia de promocion y descenso de
  clubes}{diagrama-secuencia-promocion-descenso-clubes}{H}


\subsection*{Diagrama de secuencia de composición de calendarios}
En la figura \ref{diagrama-secuencia-composicion-calendarios} se observa cómo se
crean nuevas instancias de las jornadas que estarán relacionadas con los clubes
en los que se coincida en las mismas divisiones.  \newpage

\figura{diagramas_secuencia/crear_calendario.png}{scale=0.36,
  angle=270}{Diagrama de secuencia de composición de
  calendarios}{diagrama-secuencia-composicion-calendarios}{H}

\section{Sistema de simulación}
Este apartado de la memoria requiere una especial mención; ya que es el sistema
que hará que el que los cambios y modificaciones a la táctica, alineación,
entrenamientos y precio de entradas que realice en su equipo, tengan un reflejo
directo en el desarrollo de los partidos que conforman la temporada que disputa.

Para perfilar el sistema de simulación de nuestro proyecto de fin de carrera
tuvimos varias reuniones con nuestros tutores y acotamos la forma de simular un
partido en base a la táctica y alineación que presentan los clubes que
disputarán el partido.

A continuación mostraremos la forma en que nuestro algoritmo arroja un resultado
para el partido.

\subsection{Número de jugadas}
El sistema de simulación hará varios cálculos para conformar cuantas serán las
jugadas de ataque de las que dispondrá cada uno de los clubes en base a la
calidad ponderada de los atributos que los jugadores poseen en la táctica
seleccionada por el usuario.
\subsubsection{Cálculo general}
Dicha forma de realizar la ponderación se estipulará en un fichero yml como el
siguiente, perteneciente en este caso a la táctica de fútbol 4-4-2 (es decir, el
portero, 4 defensas, 4 centrocampistas, y 2 delanteros):

\lstinputlisting[style=yml]{codigo/4-4-2.yml}

En dicho fichero podemos ver la ponderación de cada atributo que damos a un
jugador según la posición que éste ocupe en el sistema táctico, la cual hemos
denominamos en el sistema como \cursiva{calidad táctica}.

Una vez tengamos calculada la \cursiva{calidad táctica} de todos los integrantes
del once inicial en nuestro equipo nos dispondremos a sacar la media de dichas
calidades y asignaremos un número de jugadas a cada equipo con un máximo de 88
jugadas por partido con una variación aleatoria de un $\pm$20\%.

Esto quiere decir que si se enfrentan dos equipos cuyos onces iniciales tienen
la misma media, por ejemplo 50, cada equipo dispondrá de 44$\pm$9 jugadas de
ataque, las cuales podrán ser contrarrestadas si con la táctica seleccionada se
consigue robar jugadas de ataque o neutralizar las jugadas de gol.

\subsubsection{Jugada de ataque}
Una vez que hemos calculado el número de jugadas de ataque del que dispone cada
equipo, hemos de simular qué tipo de jugada va tener lugar en el desarrollo del
partido. Para ello se hará uso de la posibilidad de emular la tirada de un dado
a través de la generación de un número aleatorio para reflejar la aleatoriedad y
la suerte que existe en la vida real a la hora de jugar un partido de fútbol.

En nuestro sistema hemos decidido que existan 15 tipos de jugadas de ataque,
entre ellas, la que marca la diferencia a la hora de simular un partido, el gol.

En concreto todas las jugadas contempladas por el simulador, serán las siguientes:
\begin{enumerate}
\item El jugador que posee el balón marca gol.
\item El jugador que posee el balón consigue dar un buen pase a otro compañero
  de su equipo.
\item El jugador que posee comete un error y pierde la pelota que va parar al equipo
  contrario.
\item El jugador que posee el balón lo despeja por sentirse presionado.
\item El jugador que posee el balón lo manda fuera provocando lo silbidos de una
  buena parte de su afición.
\item El jugador del equipo que no posee el balón comete una falta.
\item El jugador que posee el balón hace un regate que deja sentado al
  adversario.
\item El jugador que posee el balón hace un espectacular pase sin mirar.
\item El jugador que posee el balón hace un regate que deja sentados a varios
  adversarios.
\item El jugador que posee el balón tira a puerta pero el portero del equipo
  contrario lo para.
\item El jugador que posee el balón tira un disparo al poste.
\item El jugador que posee el balón realizar una jugada muy peligrosa, pero el
  balón acaba saliendo por la línea de fondo.
\item El jugador que posee el balón da un mal pase que va parar a un jugador del
  otro equipo.
\item El jugador que posee el balón da un pase a un compañero que se encuentra
  en fuera de juego.
\item El jugador del equipo que no posee el balón consigue robar el balón de
  forma limpia
\end{enumerate}

Por tanto, podemos imaginar nuestro sistema de simulación como un dado de 15
caras en el cual existe una de ellas correspondiente a marcar gol, de esta forma
tenemos un quinceavo de posibilidades por jugada de marcar.

\subsubsection{Neutralizar jugadas de gol}
Una vez calculadas el número de jugadas, será posible resistir una posible
jugada de gol dependiendo de la suma de \cursiva{calidades tácticas} de los
jugadores situados en defensa para el equipo que sufre la jugada de gol,
respecto a la suma de calidades \cursiva{tácticas} de los jugadores atacantes
del equipo que está haciendo uso de la jugada de gol.

De esta forma se consigue emular la dificultad que supondría el marcar un gol
para un equipo que utiliza una táctica estándar 4-4-2 contra un equipo que
utiliza una táctica 5-4-1, respetando aún así la calidad de dichos jugadores, ya
que será necesario que los jugadores dispuestos en la táctica tengan una buena
calidad para conseguir una buena suma y así disponer de más facilidad para
neutralizar jugadas de gol.

\subsubsection{Robar jugadas de ataque}
De la misma forma que es posible neutralizar jugadas de gol, para contemplar el
beneficio de tener un centro del campo poblado, el sistema sumará las calidades
de los jugadores dispuestos en el medio campo de ambos equipo y los ponderará de
tal forma que se le roben algunas jugadas de ataque al equipo que presente una
suma menor en las \cursiva{calidades tácticas} de los jugadores situados en el
centro del campo.

\subsection{Número de espectadores}
El sistema simulará el número de espectadores que acudieron al estadio en base a
la posición que mantengan los equipos en la tabla clasificatoria, así como la
calidad que posean los equipos que disputan el partido, todo ello ponderado en
base al precio de venta de las entradas para emular la afluencia al estadio de
forma realista.

