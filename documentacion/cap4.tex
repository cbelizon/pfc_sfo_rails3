% -*-cap4.tex-*-
% Este fichero es parte de la plantilla LaTeX para
% la realización de Proyectos Final de Carrera, protejido
% bajo los términos de la licencia GFDL.
% Para más información, la licencia completa viene incluida en el
% fichero fdl-1.3.tex
% Copyright (C) 2009 Pablo Recio Quijano 
\section{Metodología de desarrollo}
Hemos decidido seguir una metodología de desarrollo que se adapte lo
suficientemente bien al framework en torno al cual desarrollaremos
nuestra aplicación.\\

Hemos decidido usar la metodología de desarrollo \cursiva{Xtreme
  Programming (XP)}

La metodología será seguida dentro de unos límites y adaptada a
nuestra necesidades, debido entre otras cosas a que somos nosotros
mismos los que hemos decidido qué funcionalidades y qué problemas
queremos resolver con nuestra aplicación; además, no hay que olvidar
que no tenemos ningún equipo que coordinar ya que sólo será una
persona quién se encargue de realizar todas y cada una de las
funcionalidades.

La programación extrema, o \cursiva{Xtreme Programming} es una de las
llamadas \cursiva{Metodologías Ágiles} de desarrollo de software más
populares hoy en día. Inicialmente fue descrita por \cursiva{Kent
  Beck} cuando trabajaba en la \cursiva{Chrysler Corporation}. En la
\cursiva{XP} se da por supuesto que no se debe prever todo antes de
empezar a codificar, que es imposible capturar todos los requisitos
del sistema, ni saber qué es todo lo que debe hacer y por tanto es
imposible hacer un diseño correcto desde el principio.

La idea principal de esta metodología consiste en trabajar
estrechamente con el cliente, diseñando pequeñas versiones
frecuentemente. El objetivo de estas versiones no es otro que
conseguir que la aplicación funcione de la forma más simple y
eficiente posible con el mínimo código. Cuando el cliente le transmite
al programador lo que necesita, éste puede hacer una estimación
aproximada del tiempo que le llevará codificarlo, aunque lógicamente
dicha planificación deberá revisarse y modificarse continuamente a lo
largo del desarrollo del proyecto, es decir, se va iterando. Cada vez
que el desarrollador consigue una versión funcional de lo que el
cliente ha solicitado, ésta se le muestra al cliente para que la
testee y haga peticiones de las modificaciones que crea
convenientes. De esta manera se evita perder tiempo desarrollando una
aplicación que no sea la que el cliente esperaba. Este ciclo se
repetirá tantas veces como el cliente necesite para sentirse
satisfecho con la aplicación.

La \cursiva{Xtreme Programming} agrupa trece prácticas básicas que se
deben cumplir para asegurar el éxito del proyecto, ellas son:

\begin{itemize}
  \item Equipo completo.
  \item Planificación continua.
  \item Test del cliente.
  \item Versiones pequeñas.
  \item Diseño simple.
  \item Programación por parejas.
  \item Desarrollo guiado por pruebas automáticas.
  \item Mejora continua del diseño.
  \item Integración continua.
  \item Código en multipropiedad.
  \item Normas de codificación comunes.
  \item Utilización de metáforas.
  \item Mantener un ritmo sostenible.
\end{itemize}

La \cursiva{Xtreme Programming} consta de unos valores, unos
principios fundamentales y unas prácticas. Los principios
fundamentales en los que se basa son:

\begin{itemize}
  \item Feedback: Retroalimentación veloz.
  \item Modificaciones incrementales.
  \item Refactorización
  \item Asunción de simplicidad.
  \item Respeto: Comunicación entre desarrolladores.
\end{itemize}



\section{Descripción General}
A continuación vamos a ver los factores que afectan al producto y sus
requerimentos

\section{Propósito}
El propósito del siguiente apartado es definir cuáles son los
requisitos que debe tener nuestro portal web que simula la gestión de
un club de fútbol en un entorno multijugador y simular partidos.\\

Esta especificación de requisitos está destinada a ser leída por los
usuarios o cualquier sujeto que tenga interés en saber cómo funciona
el producto.

\section{Ámbito}
El producto que vamos a describir puede clasificarse como un
videojuego multijugador online integrado en un servidor web en la cual
los usuarios del portal se dedicarán a gestionar un club de fútbol en
la parcela económica y deportiva, y dispondrá de un simulador de los
partidos entre clubes.\\

El producto ha de ofrecer una interfaz lo bastante sencilla para que
cualquier usuario pueda gestionar las distintas facetas en escasos
minutos. Dado que la aplicación es de tipo web, deberemos de tener
especial cuidado en respetar los estándares para que la aplicación sea
visible en cualquier navegador del mercado obteniendo así un
videojuego que sea totalmente multiplataforma. La tecnología que
usaremos en el lado del servidor web será el framework Ruby On Rails.

\section{Perspectiva del Producto}
El presente producto ha de correr en un servidor que sea compatible
con \programa{Ruby on Rails}. Antes que nada \programa{Ruby On Rails} es un framework que
hace mas fácil desarrollar, implementar y mantener aplicaciones
web. Esto es debido entre otras razones a que:

\begin{itemize}
\item Toda aplicación web implementada usa el esquema \sigla{MVC}
  (Modelo Vista Controlador).
  
\item Toda pieza de código tiene su lugar y todas ellas interactúan
  según un camino estándar, es lo que ha venido a llamarse
  \sigla{COC} (Convención sobre Configuración).
  
\item Las aplicaciones en Rails están escritas en Ruby, un lenguaje de
script moderno y orientado a objetos que ayuda a que se potencie otro
de los pilares de la filosofía de desarrollo que viene abreviado en la
siguiente nomenclatura: \sigla{DRY} (No te repitas).
\end{itemize}

En cuanto a la máquina necesaria para ejecutar el servidor, valdrá
cualquier sistema que sea compatible con \programa{Windows}, \programa{Linux} o \programa{MacOS X}
recomendándose que se use un hardware acorde con las peticiones que el
portal reciba. Los usuarios deberán de acceder al portal mediante
cualquier navegador web disponible en el mercado que sea compatible
con los estándares web \sigla{XHTML 1.1 Strict} y \sigla{CSS 2.0.}

\section{Características del usuario}
A continuación vamos a ver a qué tipo de usuarios está dirigido el
producto y cómo afectan éstos a las funciones que debe realizar
nuestro producto.\\

El producto está dirigido a todo usuario que sepa manejar un navegador
web y que esté interesado en la simulación de la gestión de un club de
fútbol, así como conocer las reglas y premisas básicas de negocio del
deporte.
\section{Obligaciones generales}
Uno de los aspectos que influirá determinantemente en el éxito del
producto será la eficiencia en las peticiones y el diseño de una
intertaz amigable y sencilla para poder gestionar los diferentes
parámetros del club de una forma rápida y precisa.\\

De cara al diseño de la interfaz web, uno de los mayores desafíos con
los que nos encontraremos será el conseguir que el diseño de la
interfaz sea operativa en cualquiera de los navegadores del mercado,
ya que los dos navegadores más populares no soportan actualmente con
suficiente solvencia los estándares \sigla{CSS 2.0} (\programa{Internet Explorer 8} y
\programa{Firefox 3.5}).
\section{Asunciones y Dependencias}
El presente producto depende enteramente de que el usuario disponga de
una conexión red al servidor a través de un navegador web
(probablemente \sigla{internet}), por tanto es importante respetar los
estándares pertinente que estipula la \sigla{w3c}.\\

\section{Especificación de los requisitos del sistema}
\subsection{Requisitos de interfaces externas}
\subsubsection{Requisitos de interfaces de usuario}
Podemos dividir los requisitos de interfaz de usuario según tres roles:

\begin{itemize}
  \item \negrita{Usuario Jugador y Usuario Administrador: }
    \begin{itemize}
      \item Interfaz atractiva para los usuarios: Es fundamental para
        la buena aceptación de nuestra aplicación el crear una
        interfaz lo suficientemente clara, dinámica y estética para
        que el usuario sienta una inmersión en la experiencia de
        juego. Una interfaz no atractiva sólo traerá problemas de
        rechazo y disconformidad por parte del usuario.\\\\
        Para que la interfaz sea sencilla de modificar y tratándose de
        una aplicación web usaremos \cursiva{hojas de estilo CSS}, así
        todo el apartado de diseño gráfico podrá ser modificado
        posteriormente sin que por ello afecte al contenido ni la
        estructura del documento \cursiva{HTML}.
      \item Una característica de nuestra aplicación es que el sistema
        de avisos está embebido dentro de la página web y es muy
        parecido al sistema de notificación de los sistemas operativos
        \cursiva{Mac OS X} y de la  \cursiva {distribución Linux Ubuntu}.
    \end{itemize}
  \item \negrita{Superusuario: }
    En este caso la interfaz corresponderá a la que tenga el propio
    sistema operativo dónde esté instalado la aplicación.
\end{itemize}

\subsection{Requisitos de rendimiento}
Este aspecto es fundamental en el análisis del sistema, se ha de tener
en cuenta que la aplicación es de tipo servidor y que a ella estarán
conectados concurrentemente un número de usuarios elevado.\\

Se ha considerado también el rendimiento que nos ofrece el diseño web
realizado. Para que nuestra página web sea compatible con la mayoría
de dispositivos y se ejecute lo suficientemente fluida hemos decidido
no usar la tecnología propietaria \cursiva{Flash}, usando para la
renderización de efectos la librería \cursiva{JQuery} en combinación
con hojas de estilo \cursiva{CSS}.\\

El rendimiento de la aplicación podrá ser menor en la primera
instanciación de la aplicación debido a que no se habrá cacheado
ningún contenido.

\subsection{Requisitos de la base de datos}
La \cursiva{base de datos} es uno de los componentes fundamentales de
la aplicación. En ella se van a almacenar todos los datos registrados
por la aplicación.\\

La aplicación sólo hará uso de una conexión al \cursiva{SGBD} escogido
por el \cursiva{superusuario} en la configuración de instalación de la
aplicación, cuyo nombre también podrá ser escogido por el citado
\cursiva{superusuario}.\\

Dicha \cursiva{BD} estará formada por las siguientes tablas:

\begin{itemize}
  \item [User] Tabla donde se almacenan los datos de cada uno de los
    usuarios que interactúan con la aplicación.
  \item [Club] Tabla donde se almacenan los datos de los clubes que
    participan en las ligas.
  \item [League] Tabla donde se almacenan los datos de las ligas que
    existen en el sistema.
  \item  [Season] Tabla donde se almacenan los datos de las temporadas
    correspondientes a cada liga del sistema.
  \item [Round] Tabla donde se almacenan los datos de las jornadas de
    cada temporada.
  \item [MatchGeneral] Tabla dónde se almacenan los partidos que se
    han de jugar cada jornada.
  \item [MatchDetail] Tabla dónde se almacenan las acciones que tienen
    lugar en cada partido.
  \item [LineUp] Tabla dónde se almacenan las alineaciones de cada
    equipo y partido.
  \item [ClubFinancesRound] Tabla donde se almacenan los datos de
    finanzas correspondientes a cada jornada y club.
  \item [Player] Tabla donde se almacenan los datos de los jugadores
    de fútbol de los clubes.
  \item [Offer] Tabla donde se almacenan los datos de las ofertas que
    se realizan los usuarios de la aplicación sobre los jugadores de
    fútbol.
  \item [Training] Tabla donde se almacenan los datos de los
    entrenamientos de los jugadores de fútbol.
  \item [AdminMessages] Tabla donde se almacenan los mensajes de
    administración hacia los usuarios.
\end{itemize}

Toda la información sobre la \cursiva{base de datos} será ampliada en
su sección correspondiente.\\

La tabla \cursiva{User} contiene los datos de los usuarios que
interactúan con la aplicación, en ella se almacenará la contraseña de
cada uno de ellos; es por ello que será necesario un
\cursiva{algoritmo criptográfico} denominado \cursiva{MD5}. De esta
forma conseguimos proteger la contraseña de los usuarios frente a un
posible ataque al servidor dónde esté la \cursiva{base de datos}.\\

Es necesario que el \cursiva{superusuario} conozca el \cursiva{SGBD}
que será utilizado con nuestra aplicación, además de disponer de un
usuario y contraseña para la correcta creación de las tablas.

\subsection{Restricciones de diseño}

No existe ninguna restricción considerable en el diseño del sistema.\\

Todo el diseño del sistema se debe de regir con la especificación
realizada.\\

Es importante estimar cuál va ser el tráfico que nuestra aplicación va
sufrir para disponer de un \cursiva{servidor} y una conexión lo
suficientemente potentes para dar servicio a todos los usuarios que se
conecten a la aplicación.

\subsection{Atributos del sistema software}
El sistema ha de cumplir las siguientes propiedades:

\begin{itemize}
  \item \negrita{Fiabilidad}: Este sistema, como todo software de
    calidad, debe ser fiable. Al estar tratando de una aplicación web
    que ha de tener un elevado acceso concurrente a los datos, hemos
    de cerciorarnos que las operaciones no dejen el sistema en un
    estado inconsistente por una posible caída del servidor.\\

    Por ello el sistema se encarga de generar un \cursiva{log} con
    todas las operaciones que ocurren en nuestra aplicación.

    Hemos de tener en cuenta que el usuario de nuestra aplicación no
    ha de conocer nada sobre la estructura interna de nuestra
    aplicación, por ello los mensajes de error serán lo más simples
    posibles para evitar posibles ataques al servidor dónde es
    alojada.

    \item \negrita{Disponibilidad}: La disponibilidad de este producto
      debe ser plena. La aplicación ha de estar disponible en
      cualquier momento y ser accesible desde cualquier lugar. Para
      llevar a cabo este precepto será necesario alojar la aplicación
      en un lugar web que tenga la suficiente capacidad para dar
      servicio a un elevado número de usuarios.

   \item \negrita{Seguridad}: La seguridad de este tipo de
     aplicaciones ha de ser bastante elevada para conseguir cumplir
     leyes de protección de datos y asegurar un correcto
     funcionamiento.

     Para la seguridad y confidencialidad de los datos se debe seguir
     un protocolo dirigido a asegurar la confidencialidad de los datos
     almacenados en la \cursiva{base de datos}.

     Será necesario que el superusuario de la aplicación disponga de
     un nombre de usuario y contraseña lo suficientemente seguros para
     trabajar con el \cursiva{SGBD} escogido.

     Los usuarios de la aplicación tendrán un nombre usuario y una
     contraseña asociada que encriptada con el algoritmo
     \cursiva{MD5}.

     Es posible que cualquier persona acceda a la página web para
     estar informado de los que ocurre en las diferentes ligas que
     componen el juego. Este tipo de usuarios no necesitan disponer de
     nombre de usuario y contraseña asociada.

     \item \negrita{Mantenibilidad}: La mantenibilidad de la
       aplicación es un aspecto clave del desarrollo de aplicaciones
       web debido al constante avance que sufre este tipo de
       tecnologías. Gracias a desarrollar la aplicación en base a un
       \cursiva{framework de desarrollo rápido de aplicaciones}
       conseguimos una alta mantenibilidad y legibilidad del
       \cursiva{código fuente} de la aplicación.

     \item \negrita{Portabilidad}: La aplicación web, al estar
       desarrollada bajo \cursiva{Ruby On Rails} tiene una amplia
       portabilidad en cuánto al lado servidor ya que éste es
       compatible con los principales sistemas operativos del mercado.

       En cuanto al lado cliente tenemos la portabilidad asegurada
       ya que sólo será necesario que el usuario disponga de una
       conexión a \cursiva{internet} y un navegador web compatible con
       los estándares.
\end{itemize}

\section{Descripción de Requisitos Funcionales}

\subsection{Configurar el sitio}
El administrador podrá navegar por la página de configuración web y la
usará para cambiar el comportamiento de la misma.\\

Esta función se realizará raras veces.
\subsection{Crear una cuenta de usuario}
Para que un usuario cualquiera sea capaz de poder usar nuestro
producto ha de tener una cuenta registrada en el servidor que lo
distinga unívocamente de cualquier otro usuario; para ello se le
pedirá introducir un nombre de usuario, y una dirección de e-mail. Una
vez registrado el sistema enviará un e-mail de confirmación de
registro dando la bienvenida a nuestro sistema.\\

Esta función es condición necesaria para que puedan realizarse todos y
cada uno de los requisitos que se expresan en los apartados siguientes
y se realizará una única vez por usuario.

\subsection{Acceder a la cuenta de usuario}
Para que cualquier usuario registrado sea capaz de usar nuestro
producto deberemos de proporcionarle un método de acceso e
identificación mediante el nombre de cuenta y contraseña que han sido
albergados en nuestro servidor.\\

Esta operación se realizará muy a menudo y tantas veces como el
usuario desee.
\subsection{Abandonar la cuenta de usuario}
Cualquier usuario que haya iniciado sesión en nuestra aplicación
deberá ser capaz de abandonar de forma segura la misma.\\

Esta operación se realizará muy a menudo y tantas veces como un
usuario que haya iniciado sesión desee.

\subsection{Abrir equipos}
El administrador será capaz de decidir cuándo los usuarios del sistema
podrán modificar sus alineaciones o tácticas, así como la posibilidad
de interactuar con los sistemas de finanzas y ofertas de jugadores.

Esta función será realizada muy a menudo.

\subsection{Cerrar equipos}
El administrador será capaz de decidir cuándo los usuarios del sistema
no podrán modificar sus alineaciones o tácticas, así como la
posibilidad de interactuar con los sistemas de finanzas y ofertas de jugadores.

Esta función será realizada muy a menudo.

\subsection{Listar ofertas}
El usuario podrá en cualquier momento visionar el listado de ofertas
pendientes, aceptadas o rechazadas que tenga en su historial.

Esta operación podrá ser realizada en cualquier momento.

\subsection{Emitir oferta por futbolista}
El usuario podrá en cualquier momento hacer una oferta por un
futbolista en el que esté interesado, interactuando de este modo con
otros usuarios a través de acuerdos económicos. Para evitar que dos
usuarios puedan realizar acuerdos económicos para hacer trampas y
beneficiarse deportiva y económicamente, éstos no podrán tasar a sus
propios jugadores haciendo de la oferta una especie de pujo, sino que
será el propio sistema el que estipulará el precio del futbolista en
base a la calidad que éste disponga.

Esta operación podrá ser realizada en cualquier momento en el que
los equipos estén abiertos.

\subsection{Aceptar oferta por futbolista}
El usuario podrá en cualquier momento aceptar una oferta de un
futbolista de su propiedad emitida por otro usuario.

Esta operación podrá ser realizada en cualquier momento en el que los
equipos estén abiertos.

\subsection{Rechazar oferta por futbolista}
El usuario podrá en cualquier momento rechazar una oferta de un
futbolista de su propiedad emitida por otro usuario.

Esta operación podrá ser realizada en cualquier momento en el que los
equipos estén abiertos.

\subsection{Cancelar oferta por futbolista}
Un usuario que esté arrepentido de una oferta realizada por un
futbolista podrá cancelarla en cualquier momento.

Esta operación podrá ser realizada en cualquier momento en el que los
equipos estén abiertos.

\subsection{Renovación de futbolistas}
El sistema actualizará los sueldos que reciben los futbolistas al
término de cada temporada, según la ponderación de los atributos
cualitativos que tengan.\\

Esta operación se realizará una vez por temporada y afectará a todos
los futbolistas del sistema.
\subsection{Venta de entradas}
El usuario podrá estipular el precio de venta de las entradas en los
partidos que se disputen en su estadio siendo ésta la forma principal
de obtención de ingresos para la gestión económica del club.\\

Esta función será realizada tantas veces como desee el usuario.

\subsection{Planificación de alineaciones}
El usuario podrá escoger cuál será la alineación que presentará para
el próximo partido que vaya a disputar su equipo, pasando ésta a ser
la alineación por defecto hasta que el usuario decida volver a
cambiarla.\\

Esta operación se realizará a menudo.
\subsection{Planificación de tácticas}
El usuario podrá escoger entre diversas tácticas de juego para
afrontar los distintos partidos. Esta acción repercutirá directamente
en el estilo de juego y por tanto en la simulación resultado del
partido que dispute.\\

Esta operación se realizará tantas veces como el usuario desee.
\subsection{Planificación de entrenamientos}
El usuario podrá personalizar los atributos a entrenar de los
futbolistas integrantes de su equipo. De esta forma el club adquirirá
valor deportivo y económico.\\

Esta función se realizará a menudo.

\subsection{Asignar club}
Una vez que un usuario haya sido registrado en el servidor mediante
una cuenta de usuario el sistema le asignará un nuevo club cuando
todas las ligas hayan concluido y el administrador así desee.\\

Esta operación se realizará una vez por temporada y por cada nuevo
usuario registrado en el sistema.

\subsection{Simular jornadas}
El sistema debe emitir comentarios durante el transcurso del partido para que
cualquier usuario sea capaz de conocer la evolución del mismo.\\

Esta función se realizará a menudo, una vez por jornada.

\subsection{Comenzar jornadas}
El sistema debe simular la disputa de un partido en base a la
ponderación de los atributos cualitativos de los futbolistas
alineados, tácticas empleadas y variables aleatorias confrontadas con las del equipo
rival.\\

Esta función se realizará a menudo, una vez por jornada.

\subsection{Pasar a siguiente jornada}
El sistema debe ser capaz de gestionar el calendario creado de cada
competición. Para ello se le da la oportunidad al administrador de
avanzar en la línea temporal del calendario para hacer discurrir las
jornadas del calendario.

Además, en esta acción se actualizarán los entrenamientos y traspasos
aceptados por futbolistas entre usuarios, así como restar los gastos
de la plantilla y mantenimiento del estadio.

Esta función se realizará a menudo a lo largo de la competición.

\subsection{Promoción y descenso de clubes}
El sistema debe ser capaz de gestionar varias ligas organizadas de
forma jerárquica según el número de usuarios que estén registrados en
nuestro servidor. De esta forma conseguiremos que el usuario desee
mejorar la posición de su club, lo que repercutirá directamente en la
capacidad deportiva y económica del mismo.\\

Esta función se realizará una vez por temporada y una vez por jornada.

\subsection{Composición de calendarios}
El sistema deberá de crear un calendario de partidos por cada liga que
está albergada en el servidor. Los calendarios se crearán una vez por
temporada y será accesible a todos los usuarios.\\

Esta función se realizará una vez por creación de liga.
\subsection{Gestiones de lesiones y sanciones}
El sistema debe ser capaz de llevar a cabo una política realista de
imposición de sanciones a los futbolistas, así como simular lesiones
más o menos duraderas.\\

Esta función se realizará a menudo.

\section{Especificación de los Casos de Uso}
\subsection{Caso de uso crear cuenta de usuario}
\negrita{Descripción: } Crea una cuenta de usuario para poder 
participar y acceder a las acciones provistas por nuestra aplicación.\\
\negrita{Actores: } Usuario no registrado, usuario registrado,
sistema. \\
\negrita{Precondición: } Ninguna. \\
\negrita{Postcondición: } Se da de alta un nuevo usuario en el
sistema. \\
\negrita{Escenario principal:}
\begin{enumerate}
  \item Un usuario no registrado desea registrarse.
  \item El usuario introduce el nombre de usuario, la contraseña
    asociada al mismo, una cuenta de correo electrónico además de repetir la contraseña para evitar un
    error humano al ingresarla y también el idioma en el que desea ver
    por defecto la web y recibir los correos electrónicos.
  \item El sistema registra al usuario.
  \item El sistema avisa al usuario registrado mediante un correo
    electrónico y la propia interfaz web que ha completado su registro con éxito.
\end{enumerate}
\negrita{Escenario alternativo:}
\begin{enumerate}
  \item Si el usuario introduce un nombre de cuenta de usuario o e-mail
    existente o dos contraseñas no coincidentes se le avisará de los errores cometidos
    para que pueda realizar un registro exitoso.
\end{enumerate}

\subsection{Caso de uso acceder a la cuenta de usuario}
\negrita{Descripción: } El usuario provee un sistema para poder
identificar a los usuarios registrados en el mismo y dejarlos acceder
a la funciones principales de nuestro sistema.\\
\negrita{Actores: } Usuario registrado, usuario logueado. \\
\negrita{Precondición: } El usuario ha de estar registrado en nuestro sistema. \\
\negrita{Postcondición: } Se identifica al usuario en el sistema y se
le permite acceder a las funciones principales del mismo. \\
\negrita{Escenario principal:}
\begin{enumerate}
  \item Un usuario registrado desea iniciar sesión.
  \item El usuario introduce el nombre de usuario y la contraseña
    asociada al mismo, además de repetir la contraseña para evitar un
    error humano al ingresarla.
  \item El sistema registra al usuario.
  \item El sistema avisa al usuario registrado mediante un correo
    electrónico y la propia interfaz web que ha completado su registro con éxito.
\end{enumerate}
\negrita{Escenario alternativo:}
\begin{enumerate}
  \item Si el usuario introduce un nombre de cuenta de usuario o dos
    contraseñas no coincidentes se le avisará de los errores cometidos
    para que pueda realizar un registro exitoso.
\end{enumerate}

\subsection{Caso de uso abandonar la cuenta de usuario}
\negrita{Descripción: } El sistema provee un sistema para que un
usuario logueado pueda desconectarse de forma segura de nuestro sistema.\\
\negrita{Actores: } Usuario logueado. \\
\negrita{Precondición: } El usuario ha de haber iniciado sesión en
nuestro sistema. \\
\negrita{Postcondición: } Se cierra la sesión del usuario con nuestro sistema. \\
\negrita{Escenario principal:}
\begin{enumerate}
  \item El usuario logueado desea abandonar su sesión.
  \item Se le pregunta al usuario si está seguro de realizar la
    acción.
  \item El sistema avisa al usuario que acaba de finalizar la sesión a
    través de la interfaz web de nuestra aplicación.
\end{enumerate}
\negrita{Escenario alternativo:}
\begin{enumerate}
  \item Si el usuario decide que no está seguro de abandonar la sesión
    no se hace nada.
\end{enumerate}

\subsection{Caso de uso abrir equipos}
\negrita{Descripción: } El sistema provee un sistema para que el
usuario administrador pueda permitir cambios de jugadores, tácticas o
cualquier tipo de acción que modifique el estado de los equipos.\\
\negrita{Actores: } Usuario administrador. \\
\negrita{Precondición: } Ninguna. \\
\negrita{Postcondición: } Los equipos quedarán abiertos. \\
\negrita{Escenario principal:}
\begin{enumerate}
  \item El administrador cierra los equipos.
  \item El sistema avisa al administrador que acaba de abrir los
    equipos a través de la interfaz web de nuestra aplicación.
\end{enumerate}

\subsection{Caso de uso cerrar equipos}
\negrita{Descripción: } El sistema provee un sistema para que el
usuario administrador pueda evitar cambios de jugadores, tácticas o
cualquier tipo de acción que modifique el estado de los equipos.\\
\negrita{Actores: } Usuario administrador. \\
\negrita{Precondición: } Ninguna. \\
\negrita{Postcondición: } Los equipos quedarán cerrados. \\
\negrita{Escenario principal:}
\begin{enumerate}
  \item El administrador cierra los equipos.
  \item El sistema avisa al administrador que acaba de cerrar los
    equipos a través de la interfaz web de nuestra aplicación.
\end{enumerate}

\subsection{Caso de uso de notificación de registro por e-mail}
\negrita{Descripción: } El sistema notificará por correo electrónico
que un registro asociado al e-mail proporcionado se ha efectuado correctamente.\\
\negrita{Actores: } Sistema y usuario registrado \\
\negrita{Precondición: } El usuario ha de haber realizado un registro exitoso. \\
\negrita{Postcondición: } El sistema remite un correo electrónico en
el idioma que seleccionó en el registro nuestro usuario. \\
\negrita{Escenario principal:}
\begin{enumerate}
  \item El sistema detecta que un nuevo usuario ha sido creado.
  \item Remite un correo electrónico de bienvenida a la aplicación a
    la cuenta de correo asociada al usuario registrado.
\end{enumerate}

\subsection{Caso de uso de listar ofertas emitidas}
\negrita{Descripción: } El sistema ha de proporcionará una interfaz
para poder listar el historial de ofertas emitidas del club asociado.
\negrita{Actores: } Usuario logueado. \\
\negrita{Precondición: } El usuario ha de estar logueado en
nuestra aplicación.
\negrita{Postcondición: } Se muestra un listado del historial de
ofertas emitidas del club.
\negrita{Escenario principal:}
\begin{enumerate}
  \item Se muestra un listado del historial de ofertas emitidas del club.
\end{enumerate}

\subsection{Caso de uso de listar ofertas recibidas}
\negrita{Descripción: } El sistema ha de proporcionará una interfaz
para poder listar el historial de ofertas recibidas del club asociado.
\negrita{Actores: } Usuario logueado. \\
\negrita{Precondición: } El usuario ha de estar logueado en
nuestra aplicación.
\negrita{Postcondición: } Se muestra un listado del historial de
ofertas recibidas del club.
\negrita{Escenario principal:}
\begin{enumerate}
  \item Se muestra un listado del historial de ofertas recibidas del club.
\end{enumerate}

\subsection{Caso de uso de emitir oferta por futbolista}
\negrita{Descripción: } El sistema ha de proporcionará una interfaz
para poder realizar una oferta por un futbolista ajeno al club del que
el usuario es propietario.
\negrita{Actores: } Usuario logueado. \\
\negrita{Precondición: } El usuario ha de estar logueado en
nuestra aplicación, el comprador ha de tener la suficiencia económica
para acometer el fichaje, además los equipos han de estar abiertos. \\
\negrita{Postcondición: } El usuario emite una oferta por un
futbolista ajeno a su club y el propietario del mismo recibe la misma.
\negrita{Escenario principal:}
\begin{enumerate}
  \item Un usuario emite una oferta por un futbolista que posee
    otro usuario.
  \item El usuario propietario del futbolista recibe la oferta.
\end{enumerate}
\negrita{Escenario alternativo:}
\begin{enumerate}
  \item El usuario comprador no puede efectuar la oferta.
  \item El sistema emitirá un aviso a través de la interfaz web para
    informar que es imposible emitir la oferta.
\end{enumerate}

\subsection{Caso de uso de cancelar oferta emitida por futbolista}
\negrita{Descripción: } El sistema ha de proporcionará una interfaz
para poder cancelar una oferta emitida por un futbolista.
\negrita{Actores: } Usuario logueado. \\
\negrita{Precondición: } El usuario ha de estar logueado en
nuestra aplicación, la oferta ha de existir y los equipos han de estar
abiertos. \\
\negrita{Postcondición: } El usuario cancela una oferta por un
futbolista de un equipo ajeno al suyo.
\negrita{Escenario principal:}
\begin{enumerate}
  \item El usuario cancela la oferta previamente realizada.
\end{enumerate}

\subsection{Caso de uso de rechazar oferta por futbolista}
\negrita{Descripción: } El sistema ha de proporcionará una interfaz
para poder rechazar una oferta realizada por un futbolista que
provenga de otro usuario.
\negrita{Actores: } Usuario logueado. \\
\negrita{Precondición: } El usuario ha de estar logueado en
nuestra aplicación, la oferta ha de existir y los equipos han de estar
abiertos. \\
\negrita{Postcondición: } El usuario rechaza una oferta por un
futbolista de un equipo ajeno al suyo.
\negrita{Escenario principal:}
\begin{enumerate}
  \item El usuario rechaza la oferta previamente realizada.
\end{enumerate}

\subsection{Caso de uso de aceptar oferta por futbolista}
\negrita{Descripción: } El sistema ha de proporcionará una interfaz
para poder aceptar una oferta realizada por un futbolista que
provenga de otro usuario.
\negrita{Actores: } Usuario logueado. \\
\negrita{Precondición: } El usuario ha de estar logueado en
nuestra aplicación, la oferta ha de existir, los equipos han de estar
abiertos, la venta del jugador no debe suponer rebasar el límite
mínimo de jugadores en plantilla. \\
\negrita{Postcondición: } El usuario acepta la oferta, el montante
económico de la clausula del jugador es transferido a las arcas del
club y el jugador ya no es perteneciente a la plantilla.
\negrita{Escenario principal:}
\begin{enumerate}
  \item El usuario acepta la oferta.
  \item El futbolista es transferido a la plantilla del usuario emisor
    de la oferta.
  \item Se incrementa la caja del club en el montante económico de la
    clausula del jugador transferido.
\end{enumerate}
\negrita{Primer Escenario alternativo: }
\begin{enumerate}
  \item El usuario emisor no tiene recursos económicos suficientes
    para acometer el fichaje.
  \item El sistema informará al usuario que no puede aceptar la oferta
    y ésta será automáticamente rechazada.
\end{enumerate}
\negrita{Segundo Escenario alternativo: }
\begin{enumerate}
  \item El usuario receptor de la oferta está en el número mínimo de
    jugadores en plantilla.
  \item El sistema informará al usuario que no puede aceptar la oferta
    y la misma será automáticamente rechazada.
\end{enumerate}
\subsection{Caso de uso de renovación de futbolistas}
\negrita{Descripción: } El sistema renovará cada final de temporada a
los jugadores de nuestro equipo actualizando el sueldo que reciben y
su clausula de rescisión.\\
\negrita{Actores: } Sistema.\\
\negrita{Precondición: } Las temporadas han de haber concluido y los
equipos bloqueados. \\
\negrita{Postcondición: } Los jugadores serán renovados en función de
la nueva ponderación de sus atributos así como su clausula de rescisión.\\
\negrita{Escenario principal:}
\begin{enumerate}
  \item El sistema realizará una ponderación de los atributos de los
    futbolistas.
  \item Se asigna una clausula de rescisión en función de la
    ponderación.
  \item Se asigna un sueldo anual en función de la ponderación.
\end{enumerate}

\subsection{Caso de uso de fijar precio de venta de entradas}
\negrita{Descripción: } El usuario podrá modificar el precio de las
entradas de su equipo de fútbol. Este hecho repercutirá directamente
en la afluencia al estadio por parte de nuestros aficionados y a las
arcas del club.\\
\negrita{Actores: } Usuario logueado.\\
\negrita{Precondición: } Los equipos deben estar desbloqueados\\
\negrita{Postcondición: } El precio del ticket de entrada es
modificado.\\
\negrita{Escenario principal:} 
\begin{enumerate}
  \item El usuario desea cambiar el precio del ticket de entrada.
  \item El ticket de entrada es cambiado al precio estipulado por el usuario.
\end{enumerate}
\negrita{Escenario alternativo:}
\begin{enumerate}
  \item El usuario introduce un precio menor de cero.
  \item El sistema informará al usuario que no puede asignar un precio
    negativo a través de la interfaz web.
\end{enumerate}

\subsection{Caso de uso de planificación de alineaciones}
\negrita{Descripción: } El usuario podrá modificar quiénes serán los
futbolistas que jueguen el siguiente partido.\\
\negrita{Actores: } Usuario logueado\\
\negrita{Precondición: } Los equipos deben estar desbloqueados.\\
\negrita{Postcondición: } El usuario dejará formada la alineación que
usará en el siguiente partido.\\
\negrita{Escenario principal:}
\begin{enumerate}
  \item El usuario desea cambiar la posición entre dos jugadores.
  \item El usuario seleccionará dos jugadores.
  \item Efectuará el cambio de posición.
  \item El usuario repetirá este proceso todas las veces que necesite hasta dejar
    la alineación como desee.
\end{enumerate}
\negrita{Escenario alternativo:}
\begin{enumerate}
  \item El usuario selecciona un sólo jugador.
  \item Efectúa el cambio de posición.
  \item El sistema emitirá un aviso para informar que la operación que
    intenta hacer es imposible.
\end{enumerate}

\subsection{Caso de uso de planificación de tácticas}
\negrita{Descripción: } El usuario podrá seleccionar una táctica
distinta para afrontar el próximo partido.\\
\negrita{Actores: } Usuario logueado.\\
\negrita{Precondición: } Los equipos han de estar desbloqueados.\\
\negrita{Postcondición: } Se cambia la táctica por la seleccionada por
el usuario.\\
\negrita{Escenario principal:}
\begin{enumerate}
  \item El usuario desea cambiar la táctica usada en el partido.
  \item El usuario selecciona la táctica que desea usar y la fija como
    táctica por defecto para los siguientes partidos.
\end{enumerate}

\subsection{Caso de uso de planificación de entrenamientos}
\negrita{Descripción: } El usuario podrá mejorar una de las
habilidades que integran un jugador cada vez que lo mande a entrenar.\\
\negrita{Actores: } Usuario logueado.\\
\negrita{Precondición: } Los equipos han de estar desbloqueados y el
jugador no puede estar entrenando otra habilidad previa.\\
\negrita{Postcondición: } El jugador entrenará la habilidad
seleccionada por el usuario las jornadas que el sistema estipule.\\
\negrita{Escenario principal:}
\begin{enumerate}
  \item El usuario desea que un jugador de su plantilla entrene cierta habilidad.
  \item El sistema informará de cuántas jornadas serán necesarias para
    terminar el entrenamiento.
\end{enumerate}
\negrita{Escenario alternativo: }
\begin{enumerate}
  \item Si el usuario intenta poner a entrenar la habilidad de un
    jugador que haya llegado al máximo posible para el atributo el
    sistema le informará a través de la interfaz web de que no es
    necesario realizar dicha acción.
\end{enumerate}

\subsection{Caso de uso asignar club}
\negrita{Descripción: } El sistema proporciona una forma de asignar
nuevos club de fútbol a los nuevos usuarios registrados cuando el administrador lo desee entre
temporada y temporada.\\
\negrita{Actores: } Sistema, usuario registrado y administrador \\
\negrita{Precondición: } Las temporadas han de haber finalizado, debe
haber un mínimo de nuevos usuarios registrados para poder formar la
nueva liga y los equipos han de estar bloqueados. \\
\negrita{Postcondición: } Se les asigna a cada usuario un nuevo club
con varios jugadores y un monto inicial de dinero para poder comenzar
con las transacciones. \\
\negrita{Escenario principal:}
\begin{enumerate}
  \item El administrador desea crear una nueva liga.
  \item El sistema verifica que haya los suficientes nuevos usuarios
    sin clubes asignados para crearla.
  \item El sistema genera una plantilla de futbolistas nuevos a cada
    nuevo club creado.
  \item Se asigna cada club a cada usuario nuevo que formará la nueva
    liga.
\end{enumerate}

\subsection{Caso de uso de simular jornadas}
\negrita{Descripción: } El sistema simulará cada partido de cada
jornada de cada temporada según las alineaciones y tácticas de los
equipos que se enfrenten.\\
\negrita{Actores: } Administrador y Sistema. \\
\negrita{Precondición: } Los equipos deben estar bloqueados.\\
\negrita{Postcondición: } Se generará un conjunto de comentarios
ordenados por cada acción que tenga lugar en el partido.\\
\negrita{Escenario principal: }
\begin{enumerate}
  \item El administrador decidirá simular los partidos de la jornada
    en la que se encuentre el sistema.
  \item El sistema generará una ristra de comentarios por cada acción
    generada en el encuentro usando la ponderación de los equipos
    alineados y las tácticas utilizadas.
\end{enumerate}

\subsection{Caso de uso de comenzar jornadas}
\negrita{Descripción: } El sistema emulará la retransmisión de los
partidos englobados en las jornadas en tiempo real.\\
\negrita{Actores: } Administrador y Sistema. \\
\negrita{Precondición: } Los equipos deben estar bloqueados.\\
\negrita{Postcondición: } Se emitirá a través de la interfaz web el
conjunto de comentarios generados previamente en la simulación de los partidos.\\
\negrita{Escenario principal: }
\begin{enumerate}
  \item El administrador decide que se comiencen a emitir los
    comentarios de los partidos.
  \item El sistema empieza a mostrar los comentarios de cada partido
    de forma ordenada y actualizándose cada minuto.
\end{enumerate}

\subsection{Caso de uso de pasar a siguiente jornada}
\negrita{Descripción: } El sistema proveerá una forma de avanzar a lo
largo del calendario de partidos y actualizar traspasos, finanzas y entrenamientos.\\
\negrita{Actores: } Administrador y Sistema.\\
\negrita{Precondición: } Los equipos deben estar bloqueados\\
\negrita{Postcondición: }Se pasará de jornada en el calendario de las
ligas, se actualizarán los entrenamientos, traspasos y finanzas de los
clubes registrados en el sistema.\\
\negrita{Escenario principal: }
\begin{enumerate}
  \item El administrador decide pasar a la siguiente jornada.
  \item El sistema actualiza los resultados y las clasificaciones de
    cada liga y hace de la jornada siguiente la actual.
  \item Se actualizan las finanzas de los clubes.
  \item Se actualizan los traspasos aceptados.
  \item Se actualizan los entrenamientos de los jugadores.
\end{enumerate}
\negrita{Escenario alternativo: }
\begin{enumerate}
  \item Si la jornada es la última se dará por concluidas las
    temporadas que tenían lugar en cada liga.
\end{enumerate}


\subsection{Caso de uso de promoción y descenso de clubes}
\negrita{Descripción: } El sistema proveerá una forma de organizar las
ligas de forma jerárquica.\\
\negrita{Actores: } Administrador y Sistema.\\
\negrita{Precondición: } Los equipos deben estar bloqueados\\
\negrita{Postcondición: } Se ascenderán y descenderán los clubes que
estén en posiciones de descenso y ascenso según la organización
jerárquica de las ligas.\\
\negrita{Escenario principal: }
\begin{enumerate}
  \item El sistema reorganizará las ligas para reflejar los cambios en
    las ligas debida a las posiciones que obtengan en la tabla de
    clasificación los equipos de nuestro sistema.
\end{enumerate}

\subsection{Caso de uso de composición de calendarios}
\negrita{Descripción: } El sistema proveerá una forma de componer los
calendarios de las ligas que se estén jugando en nuestra aplicación.\\
\negrita{Actores: } Administrador y Sistema.\\
\negrita{Precondición: } Los equipos deben estar bloqueados y las
temporadas finalizadas.\\
\negrita{Postcondición: } Se generará un calendario de jornadas
siguiendo un algoritmo round-robin de creación de calendarios por cada
liga albergada en nuestro sistema.\\
\negrita{Escenario principal: }
\begin{enumerate}
  \item El sistema creará un calendario por liga de las nuevas
    temporadas que van a jugarse en nuestro sistema.
\end{enumerate}

\section{Diagramas de Casos de Uso}

\figura{casos_uso/funciones_principales.png}{scale=0.35}{Diagrama de
  casos de uso de las funciones principales}{casos-uso-funciones-generales}{H}

\negrita{Descripción:} 
\begin{enumerate}
  \item{Gestión de usuarios}
    \begin{itemize}
      \item Configurar el sitio
      \item Crear una cuenta de usuario
      \item Acceder a la cuenta de usuario
      \item Abandonar la cuenta de usuario
      \item Notificación de registro por e-mail
      \figura{casos_uso/gestion_usuarios.png}{scale=0.35}{Diagrama de
        casos de uso para la gestión de usuarios}{casos-uso-gestion-usuarios}{H}
    \end{itemize}
  \item{Gestión económica}
    \begin{itemize}
      \item Listar ofertas emitidas
      \item Listar ofertas recibidas
      \item Emitir oferta por futbolista
      \item Aceptar oferta por futbolista
      \item Rechazar oferta por futbolista
      \item Cancelar oferta emitida por futbolista
      \item Renovación de futbolistas
      \item Fijar precio de venta de entradas
        \figura{casos_uso/gestion_economica.png}{scale=0.35}{Diagrama
          de casos de uso para la gestión económica}{casos-uso-gestion-economica}{H}
    \end{itemize}
  \item{Gestión deportiva}
    \begin{itemize}
      \item Planificación de alineaciones
      \item Planificación de tácticas
      \item Planificación de entrenamientos
      \figura{casos_uso/gestion_deportiva.png}{scale=0.35}{Diagrama de
      casos de uso para la gestión deportiva}{casos-uso-gestion-deportiva}{H}
    \end{itemize}
  \item{Gestión de campeonatos}
    \begin{itemize}
      \item Asignar club
      \item Simular jornadas
      \item Comenzar jornadas
      \item Pasar a siguiente jornada
      \item Promoción y descenso de clubes
      \item Composición de calendarios
      \figura{casos_uso/gestion_campeonatos.png}{scale=0.35}{Diagrama
        de casos de uso para la gestión de campeonatos}{casos-uso-gestion-campeonatos}{H}
     \end{itemize}
\end{enumerate}

\section{Diagrama de clases conceptuales}
En la figura \ref{diagrama-clases-uml} podemos observar el diagrama de clases de
nuestra aplicación según la recolección de requisitos que hemos
obtenido.

\figura{uml.png}{scale=0.30}{Diagrama de Clases
  UML}{diagrama-clases-uml}{H}

\section{Diagramas de secuencia de sistemas}

\subsection{Diagrama de secuencia de crear cuenta de usuario}

\figura{diagramas_secuencia/crear_cuenta_usuario.png}{scale=0.35}{Diagrama de
  secuencia de crear cuenta de usuario}{diagrama-secuencia-crear-cuenta-usuario}{H}

\subsection{Diagrama de secuencia de acceder a cuenta de usuario}

\figura{diagramas_secuencia/acceder_cuenta_usuario.png}{scale=0.35}{Diagrama de
  secuencia de acceder a cuenta de
  usuario}{diagrama-secuencia-acceder-cuenta-usuario}{H}

\subsection{Diagrama de secuencia de abandonar cuenta de usuario}

\figura{diagramas_secuencia/abandona_cuenta_usuario.png}{scale=0.35}{Diagrama de
  secuencia de abandonar cuenta de
  usuario}{diagrama-secuencia-abandonar-cuenta-usuario}{H}

\subsection{Diagrama de secuencia de abrir equipos}

\figura{diagramas_secuencia/abrir_equipos.png}{scale=0.35}{Diagrama de
  secuencia de abrir equipos}{diagrama-secuencia-abrir-equipos}{H}

\subsection{Diagrama de secuencia de cerrar equipos}

Este diagrama es igual al \ref{diagrama-secuencia-abrir-equipos} pero
con la ruta \cursiva{/leagues/close\_teams} y la llamada al modelo con
el método \cursiva{close}.


\subsection{Diagrama de secuencia de notificación de registro por
  e-mail}

\figura{diagramas_secuencia/crear_cuenta_usuario.png}{scale=0.35}{Diagrama de
  secuencia de notificación de registro por e-mail}{diagrama-secuencia-notificacion-registro}{H}


\subsection{Diagrama de secuencia de listar ofertas emitidas}

\figura{diagramas_secuencia/listar_ofertas_emitidas.png}{scale=0.35}{Diagrama de
  secuencia de listar ofertas emitidas}{diagrama-secuencia-listar-ofertas-emitidas}{H}

\subsection{Diagrama de secuencia de listar ofertas recibidas}

Este diagrama es igual al
\ref{diagrama-secuencia-listar-ofertas-emitidas} pero con la ruta
\cursiva{/clubs/*/received\_offers} y la llamada al modelo club con el
método \cursiva{offers\_as\_seller}.

\subsection{Diagrama de secuencia de emitir oferta por futbolista}

\figura{diagramas_secuencia/emitir_oferta_futbolista.png}{scale=0.35}{Diagrama de
  secuencia de emitir oferta por
  futbolista}{diagrama-secuencia-emitir-oferta-futbolista}{H}

\subsection{Diagrama de secuencia de cancelar oferta emitida por
  futbolista}

\figura{diagramas_secuencia/cancelar_oferta_emitida_futbolista.png}{scale=0.35}{Diagrama de
  secuencia de cancelar oferta emitida por
  futbolista}{diagrama-secuencia-cancelar-oferta-emitida-futbolista}{H}

\subsection{Diagrama de secuencia de rechazar oferta recibida por
  futbolista}

\figura{diagramas_secuencia/rechazar_oferta_recibida_futbolista.png}{scale=0.35}{Diagrama de
  secuencia de rechazar oferta recibida por
  futbolista}{diagrama-secuencia-rechazar-oferta-recibida-futbolista}{H}

\subsection{Diagrama de secuencia de aceptar oferta recibida por
  futbolista}

\figura{diagramas_secuencia/aceptar_oferta_recibida_futbolista.png}{scale=0.35}{Diagrama de
  secuencia de aceptar oferta recibida por
  futbolista}{diagrama-secuencia-aceptar-oferta-recibida-futbolista}{H}

\subsection{Diagrama de secuencia de renovación de futbolistas}

\figura{diagramas_secuencia/renovacion_futbolistas.png}{scale=0.35}{Diagrama de
  secuencia de renovación de
  futbolistas}{diagrama-secuencia-renovacion-futbolistas}{H}

El proceso de renovación está integrado en start.

\subsection{Diagrama de secuencia de fijar precio de venta de
  entradas}

\figura{diagramas_secuencia/fijar_precio_venta_entradas.png}{scale=0.35}{Diagrama de
  secuencia de fijar precio de venta de
  entradas}{diagrama-secuencia-fijar-precio-venta-entradas}{H}

\subsection{Diagrama de secuencia de planificación de alineaciones}

\figura{diagramas_secuencia/planificacion_alineaciones.png}{scale=0.35}{Diagrama de
  secuencia de emitir planificación de
  alineaciones}{diagrama-secuencia-planificacion-alineaciones}{H}

\subsection{Diagrama de secuencia de planificación de tácticas}

\figura{diagramas_secuencia/planificacion_tacticas.png}{scale=0.35}{Diagrama de
  secuencia de planificación de tácticas
}{diagrama-secuencia-planificacion-tacticas}{H}

\subsection{Diagrama de secuencia de planificación de entrenamientos}

\figura{diagramas_secuencia/planificacion_entrenamientos.png}{scale=0.35}{Diagrama de
  secuencia de planificación de entrenamientos
}{diagrama-secuencia-planificacion-entrenamientos}{H}

\subsection{Diagrama de secuencia de asignar club}

\figura{diagramas_secuencia/asignar_club.png}{scale=0.35}{Diagrama de
  secuencia de secuencia de asignar club
}{diagrama-secuencia-asignar-club}{H}

\clearpage

\subsection{Diagrama de secuencia de simular jornadas}

\figura{diagramas_secuencia/simular_jornadas.png}{scale=0.32, angle=270}{Diagrama de
  secuencia de simular
  jornadas}{diagrama-secuencia-simular-jornadas}{H}

\subsection{Diagrama de secuencia de comenzar jornadas}

\figura{diagramas_secuencia/comenzar_jornadas.png}{scale=0.43, angle=270}{Diagrama de
  secuencia de comenzar
  jornadas}{diagrama-secuencia-comenzar-jornadas}{H}


\subsection{Diagrama de secuencia de pasar a siguiente jornada}

\figura{diagramas_secuencia/pasar_siguiente_jornada.png}{scale=0.27, angle=270}{Diagrama
 de secuencia de pasar a siguiente
  jornada}{diagrama-secuencia-pasar-siguiente-jornada}{H}

\subsection{Diagrama de secuencia de promoción y descenso de clubes}

\figura{diagramas_secuencia/promocion_descenso_clubes.png}{scale=0.35}{Diagrama
 de secuencia de promocion y descenso de
 clubes}{diagrama-secuencia-promocion-descenso-clubes}{H}

\clearpage

\subsection{Diagrama de secuencia de composición de calendarios}

\figura{diagramas_secuencia/crear_calendario.png}{scale=0.32, angle=270}{Diagrama
 de secuencia de composición de calendarios}{diagrama-secuencia-composicion-calendarios}{H}